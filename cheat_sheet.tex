\documentclass{article}
\usepackage[a4paper, total={6in, 8in}]{geometry}
\usepackage[italian]{babel}
\usepackage[utf8]{inputenc}
\usepackage[T1]{fontenc}
\usepackage{amsmath}
\usepackage{listings}
\usepackage{hyperref}
\usepackage{siunitx}
\usepackage{fancyhdr}
\pagestyle{fancy}
\usepackage{textcomp}
\usepackage{makecell}
\usepackage[font=small,labelfont=bf]{caption} 
\usepackage{pdfpages}
\usepackage{multicol}
\usepackage[ruled,vlined]{algorithm2e}
\usepackage{mhchem}
\usepackage{float}
\usepackage{graphicx}
\pagestyle{fancy}
\fancyhf{}
\lhead{\rightmark}
\cfoot{\leftmark}
\rfoot{\thepage}
\newlength{\depthofsumsign}
\setlength{\depthofsumsign}{\depthof{$\sum$}}
\newlength{\totalheightofsumsign}
\newlength{\heightanddepthofargument}
\setcounter{secnumdepth}{5}
\newcommand{\nsum}[1][1.4]{% only for \displaystyle
    \mathop{%
        \raisebox
            {-#1\depthofsumsign+1\depthofsumsign}
            {\scalebox
                {#1}
                {$\displaystyle\sum$}%
            }
    }
}
\lstset{
    frame=tb, % draw a frame at the top and bottom of the code block
    tabsize=4, % tab space width
    showstringspaces=false, % don't mark spaces in strings
    numbers=none, % display line numbers on the left
    commentstyle=\color{green}, % comment color
    keywordstyle=\color{red}, % keyword color
    stringstyle=\color{blue}, % string color
    breaklines=true,
    postbreak=\mbox{\textcolor{green}{$\hookrightarrow$}\space}
}

\renewcommand{\lstlistingname}{}% Listing -> Algorithm
\renewcommand{\lstlistlistingname}{Algoritmi}% List of Listings -> List of Algorithms
\author{
  Giacomo Fantoni \\
  \small Telegram: \href{https://t.me/GiacomoFantoni}{@GiacomoFantoni} \\[3pt]
  Github: \href{https://github.com/giacThePhantom/AlgoritmiStruttureDati}{https://github.com/giacThePhantom/AlgoritmiStruttureDati}}


\renewcommand*{\listalgorithmcfname}{}
\renewcommand*{\algorithmcfname}{}
\renewcommand*{\algorithmautorefname}{}
\renewcommand{\thealgocf}{}
\newcommand{\resum}[1]{%
    \def\s{#1}
    \mathop{
        \mathpalette\resumaux{#1}
    }
}
\newcommand{\resumaux}[2]{% internally
    \sbox0{$#1#2$}
    \sbox1{$#1\sum$}
    \setlength{\heightanddepthofargument}{\wd0+\dp0}
    \setlength{\totalheightofsumsign}{\wd1+\dp1}
    \def\quot{\DivideLengths{\heightanddepthofargument}{\totalheightofsumsign}}
    \nsum[\quot]%
}

% http://tex.stackexchange.com/a/6424/16595
\makeatletter
\newcommand*{\DivideLengths}[2]{%
  \strip@pt\dimexpr\number\numexpr\number\dimexpr#1\relax*65536/\number\dimexpr#2\relax\relax sp\relax
}
\makeatother


\title{\Huge \textbf{Genetica}}

\author{
  Giacomo Fantoni \\
  \small Telegram: \href{https://t.me/GiacomoFantoni}{@GiacomoFantoni} \\[3pt]
  \small Github: \href{https://github.com/giacThePhantom/Genetica}{https://github.com/giacThePhantom/Genetica}}
\begin{document}
\maketitle
\begin{multicols}{2}
\paragraph*{Mendel}
\begin{itemize}
	\item Incroci monoibridi $3:1$.
	\item Incroci diibridi $9:6:1$.
	\item Testcross $1:2:1$.
	\item $\chi^2$, probabilit\`a che la differenza tra valori osservati e attesi dovuta al caso: $\chi^2 = \nsum\dfrac{(obs-exp)^2}{exp}$.
	\item Gradi di libert\`a: $n-1$, $n$ numero di fenotipi attesi.
\end{itemize}

\paragraph*{Estensione Mendel}
\begin{itemize}
	\item Dominanza incompleta $1:2:1$.
	\item Codominanza.
	\item Allelismo multiplo.
	\item Penetranza.
	\item Espressivit\`a.
	\item Alleli letali $2:3$.
	\item Epistasi recessiva $9:3:4$.
	\item Epistasi dominante $I$ $12:3:1$.
	\item Epistasi dominante $II$ $13:3$.
	\item Epistasi recssiva doppia $9:7$.
	\item Ridondanza genica $15:1$.
	\item Interazione genica dominante $9:6:1$.
\end{itemize}

\paragraph*{Pedigree}
\begin{itemize}
	\item Caratteri autosomici recessivi:
		\begin{itemize}
			\item Stessa frequenza in entrambi i sessi.
			\item Un allele per genitore.
			\item Raro genitori eterozigoti non affetti.
			\item Salta generazioni.
			\item Entrambi i genitori affetti tutta progenie.
			\item Rato partner esterno figli non manifestano, eterozigoti.
		\end{itemize}
	\item Caratteri autosomici dominanti:
		\begin{itemize}
			\item Stessa frequenza in entrambi i sessi.
			\item Trasmissione diretta, individuo ereditato allele da un genitore.
			\item Non saltano generazioni.
			\item Raro individui eterozigoti.
		\end{itemize}
	\item Caratteri recessivi legati al $X$:
		\begin{itemize}
			\item Pi\`u frequenti nei maschi.
			\item Femmine due.
			\item Maschi madri non affette ma portatrici e padri affetti.
			\item Salta generazioni.
			\item Donna eterozigote met\`a figli affetti e met\`a figlie portatrici.
			\item No trasmissione da padre a figlio.
			\item Figlie di uomo affetto portatrici.
			\item Figli di donna omozigote affetti.
		\end{itemize}
	\item Caratteri dominanti legati al $X$:
		\begin{itemize}
			\item Maggiore frequenza in femmine.
			\item Genitore affetto, non saltano generazioni.
			\item Maschi affetti a figlie non figli.
			\item Donne affetti met\`a figli e met\`a figlie.
			\item Maschi solo dalla madre.
		\end{itemize}
	\item Caratteri legati al $Y$:
		\begin{itemize}
			\item Solo i maschi.
			\item Da padre a figlio completamente.
			\item No rapporto dominanza.
			\item Figlie sane.
		\end{itemize}
\end{itemize}

\paragraph*{Determinazione sesso}
\begin{multicols}{2}
\begin{itemize}
	\item $XX$-$X0$.
	\item $ZZ$-$ZW$.
	\item $XX$-$XY$.
	\item $X:A$.
	\item Genica.
	\item Ambiente.
\end{itemize}
\end{multicols}

\paragraph*{Linkage}
\begin{itemize}
	\item Frequenza di ricombinazione e distanza di mappa: $\dfrac{nprog_{ric}}{nprog_{tot}}\cdot 100$.
	\item Test $\chi^2$ indipendenza:
		\begin{itemize}
			\item Tabella valori osservati e totali righe e colonne, generale.
			\item $EXP = \dfrac{TOT_{riga} \cdot TOT_{colonna}}{TOT_{generale}}$.
			\item $\chi^2$.
			\item $gl = (numero\ righe -1)\cdot(numero\ colonne - 1)$.
		\end{itemize}
	\item Mappe genetiche tre punti:
		\begin{itemize}
			\item Ordine dei geni, centrale.
			\item Localizzazione dei crossing over.
			\item Frequenza di ricombinazione.
			\item Coefficiente di coincidenza: $\dfrac{OBS_{2Xcrossover}}{EXP_{2Xcrossover}}$.
			\item Interferenza = $1 - coefficiente\ di\ coincidenza$.
		\end{itemize}
	\item Probabilit\`a associazione: $P(\frac{x}{n}, RF) = \frac{n!}{x!y!}\cdot (\frac{RF}{2})^x\cdot(\frac{1-RF}{2})^y$.
	\item Logarithm of odds = $\log_10 \dfrac{P(\frac{x}{n}, RF)}{P(\frac{x}{n}, 0.5)}$.
	\item Analisi delle tetradi.
		\begin{itemize}
			\item Ditipo parentale no ricombinanti.
			\item Tetratipo 2 parentali due ricombinanti.
			\item Ditipo non parentale tutti ricombinanti, a due a due uguali.
			\item Distanza di mappa $\dfrac{(T-2DNP) + 2(4DNP)}{TOT}\cdot 0.5\cdot 100$.
		\end{itemize}
\end{itemize}

\paragraph*{Polimorfismi}
\begin{itemize}
	\item Aplotipi: siti polimorfici con due varianti alleliche.
		Associazione di nucleotidi polimorfici presenti in una porzione di cromosoma che tendono ad essere ereditati insieme.
	\item Linkage disequilibrium: misura della coereditabilit\`a di elementi genetici.
\end{itemize}

\paragraph*{Genetica di popolazione}
\begin{itemize}
	\item Frequenza genotipica.
	\item Frequenza allelica.
	\item Equilibrio di Hardy-Weinberg:
		\begin{itemize}
			\item $q^2+2pq+p^2 = 1$.
			\item $p = f(A) = \dfrac{2n_{aa} + n_{Aa}}{2N}$.
			\item $q = f(a) = \dfrac{2n_{aa}+n_{Aa}}{2N}$.
			\item $f(AA) = p^2$.
			\item $f(Aa) = 2pq$.
			\item $f(aa) = q^2$.
		\end{itemize}
	\item Fitness genotipi moltiplicati per $W_{AA}$ numero medio di prole generata diviso numero massimo tra i medi.
\end{itemize}

\paragraph*{Genetica quantitativa}
\begin{itemize}
	\item Probabilit\`a genotipi omozigoti parentali $(\dfrac{1}{4})^2$, $n$ geni coinvolti.
	\item Norma di reazione

\end{itemize}
\end{multicols}

\end{document}
