\chapter{Polimorfismi}

\section{Panoramica}
SNP variazione di sequenza fra poszioni di DNA non omologhe, polimorfismo che coinvolge un solo nucleotide.
SI differrnziano dalle mutazioni per la loro frequenza di almeno $1\%$.

\section{Tecniche per la rilevazione di polimorfismi}

	\subsection{Saggio RFLP}
	Bassa densit\`a, enzimi di restrizione specifici, sequenze palindromiche.
	Alleli digeribili e non.
	Elettroforesi diversa banda.
	Southern blot o PCR.
	PCR non digestione per microsatellite, polimorfiche per il numero di ripetizioni.;

	\subsection{Sonda allele-specifica}
	Sonda complementare per l'allele non polimorfico, annealing situazione con solo appaiamento perfetto.
	Fluorescenza, sonde comlementari con fluorescente e quencher.
	RIpiegata su s\`e tessa buia, legata a DNA luminosa.

	\subsection{Illumina}
	PCR per genotipizzazione di diversi alleli.
	Biglia di interazione con streptavidina e biotina.
	DNA biotinilato e appliczione della biglia.
	P1 e P2 aomplementari alla regione e l'ultima con polimorfismo.
	Appariare P1 con T e P2 con C nell'ibridazione allele specifica.
	Un primer locus specifico si lega a valle del polimorfismo con due regioni colorate e una universale P3.
	SI fa appiare il primer allel specifico e lo si estende fino alla regione del primer locus-specifico.
	Una ligasi genera DNA con code universali, P1, P3.
	Si ottengono $1000$ frammenti diversi nella porzione cenrtale e diagnostici per $1000$ polimorfismi e uguali nelle estremit\`a.
	Il locus specifico ha un indirizzo che riconduce l'informazione al locus, legato a microbiglie con microantenne.
	I prodotti di PCR divrersi si appaiano con zolle diverse in base alla complemetnariet\`a.
	Zolla associata con un locus P1 e P2 con sonde fluorescenti valutando omozigosi o eterozigosi di un locus in base ai colori e la qualit\`a dell'allele.

	\subsection{Infium assays}
	Sintesi del DNA con sonda primer fino al nucleotide prima del polimorfico.
	Primo nt incorporato complementare al polimorfico.
	Macro-array, sintesi locale e ibridazioni con analisi di milioni di variazioni.
	QTL.

\section{Aplotipi}
Aplotipo definizione empirica, tutti i siti polimorfici due fvarianti alleliche.
SI riportano le lettere corrispondenti ai polimorfismi.
Aplotipo: associazione di nucleotidi polimorfici presenti in una porzione di cromosoma che tendono ad essere ereditati insieme.

	\subsection{Numero minimo}
	Il numero minimo di polimorfismi che consente di diagnosticare l'appartenenza della sequenza all'aplotipo.
	Questi vengono elevati a tag SNP.

	\subsection{Provenienza di varianti genetiche}
	Codice a barre di polimorfismi vicini all'interesse, origine molecolare con il punto di partenza dell'allel dominante.

	\subsection{PKU}
	Origine stabilita dai polimorfismi coereditati con la mutazione.

	\subsection{Linkage disequilibrium}
	Assortimento indipendente, coereditabilit\`a di elementi genetici, mappare organizzazione del genoma.
	Se linkage con elemento genetico.;j

	\subsection{Genome wide association study}
	Stato genetico normoglicemiche con livello di glucosio pi\`u alto.
	Maggiore rischio di morte.
	Analizzando polimorfismi se ne cercano coereditati.
	Omozigoti per allele della glucosio 6 fosfatasi pi\`u probabilit\`a di avere livelli di glucosio alti.

	\subsection{Colorazione occhio}
	Aplotipo con tre SNP coereditati, OCA2 introne I omozigosi inattivazione OCA2 albinismo.
	SNP colorazioni leggere e graduali.
