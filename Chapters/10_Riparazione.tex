\chapter{Riparazione del DNA}

\section{Panoramica}

	\subsection{Tasso di errore nella replicazione}
	La fedelt\`a della replicazione viene caratterizzata dal tasso di errore della DNA polimerasi.
	La capacit\`a della polimerasi varia in base al complesso specifico.
	Si nota inoltre un'attivit\`a di proofreading svolta da un endonucleasi e un esonucleasi.
	Un sistema di riparazione coaudiva il processo di replicazione.

	\subsection{Rischio di malappaiamento}
	Il rischio di creare un malappaiamento aumenta nel caso di danni al DNA.
	Una volta generato tale appaiamento pu\`o essere corretto.
	Se invece persiste la situazione diventa una mutazione.
	Il sistema di riconoscimento e riparazione dei mismatch elimina il nucleotide errato permettendo alla polimerasi una sintesi corretta.

	\subsection{Checkpoint}
	Esistono dei checkpoint durante la replicazione capaci di riconoscere gravi errori al DNA individuando una forcella di replicazione in stallo.
	Questi innescano sistemi di compensazione come regressione, sintesi traslesione o la riparazione ricombinazionale.

	\subsection{Tipologie di risposte cellulari al danno al DNA}
	\begin{multicols}{2}
		\begin{itemize}
			\item Reversione del danno alla base.
			\item Escissione di basi scorrette, malappaiate o danneggiate:
				\begin{itemize}
					\item Base excision repair \emph{BER}.
					\item Nucleotide excision repair \emph{NER}.
					\item Transcription-coupled nucleotide excision repair \emph{TC-NER}.
					\item Alternative excision repair \emph{AER}.
					\item Mismatch repair \emph{MMR}.
				\end{itemize}
			\item Riparazione di rotture al filamento:
				\begin{itemize}
					\item Single-strand break repair \emph{SSBR}.
					\item Double-strand break repair \emph{DSBR}.
				\end{itemize}
			\item Tolleranza al danno di basi:
				\begin{itemize}
					\item Sintesi del DNA translesione \emph{TLS}.
					\item Gap filling post replicativo.
					\item Replicazione nella progressione della forcella.
				\end{itemize}
			\item Attivazione dei checkpoint del ciclo cellulare.
			\item Apoptosi.
		\end{itemize}
	\end{multicols}

\section{Sintesi del DNA translesione}

	\subsection{Panoramica}
	La sintesi del DNA translesione \emph{TLS} \`e un meccanismo di sintesi del DNA attivato nel caso di danni chimici al DNA come dimeri di pirimidina formati da raggi UV.
	In questi luoghi la polimerasi non riesce a svolgere l'attivit\`a di sintesi a causa di ingombro sterico.

	\subsection{Cambio di polimerasi}
	Il meccanismo consiste nell'utilizzo di una DNA polimerasi poco fedele in grado di copiare il frammento modificato in modo errato.
	Questa \`e poco processiva e si stacca dopo le due basi permettendo il continuo della normale replicazione.

		\subsubsection{Sito catalitico}
		Il sito catalitico di questa DNA polimerasi \`e pi\`u ampio in modo da tollerare appaiamenti errati.

		\subsubsection{Caricamento}
		Il caricamento della DNA polimerasi specializzata avviene attraverso modifiche post-traduzionali della \emph{PCNA}.
		L'assenza di queste modifiche consente un normale processo di replicazione.

	\subsection{Danni drammatici}
	In caso di danni troppo estesi la \emph{PCNA} viene poli-ubiquitinata causando una retrocessione del replisosoma.
	Pu\`o essere usato un nuovo stampo non danneggiato in modo da ottenere una copia \emph{chicken foot}.

\section{Mismatch repair}

	\subsection{Panoramica}
	Il sistema di mismatch repair \emph{MMR} si occupa di trovare e correggere malappaiamenti.

	\subsection{Fissazione dei malappaiamenti}
	Nel caso di replicazione completa con un malappaiamento dopo un secondo ciclo di mutazione diventa complicato correggere questo errore.
	L'errore \`e recuperabile e richiede un sistema specifico che scansiona il DNA.

	\subsection{Sistema di mismatch}
	Il sistema di mismatch si compone di una proteina omodimerica con un sito di legame per \emph{ATP} in grado di misurare la geometria del DNA.
	
		\subsubsection{Attivazione}
		Quando la proteina riconosce una variazione recluta \emph{MutL/H/S} che riconoscono il filamento di nuova sintesi, rompono un legame fosfodiestere formando un nick.
		
		\subsubsection{Riparazione del danno}
		Il taglio permette l'attivit\`a di un'esonucleasi che digerisce il filamento in una direzione.
		Successivamente viene reclutata una DNA polimerasi che riempie il gap e ripara il danno.

		\subsubsection{Riconoscimento del filamento di nuova sintesi}

			\paragraph{Procarioti}
			Negli eucarioti la replicazione porta alla formazione di DNA emimetilato: il filamento non metilato \`e quello di nuova sintesi.
			A questo si lega \emph{MutH}.
			Questo avviene in quanto la metilazione avviene successivamente alla replicazione, creando un intervallo di tempo di emimetilazione.
			Questo permette a \emph{MutS} di riconoscere il malappaiameento.
			La distanza del taglio \`e influenzata dalla metilazione.
			L'emitelazione si crea grazie a due proteine che competono per il legame al DNA.

			\paragraph{Eucarioti}
			Negli eucarioti il filamento di nuova sintesi lagging pu\`o essere riconosciuto grazie ai numerosi nick che intervallano i frammenti.
			L'altro filamento pu\`o essere riconosciuto da nick creati dall'errato inserimento di ribonucleotidi rimossi da \emph{ribonucleasiH}.
			Gli analoghi di \emph{MutS} e \emph{MutL} sono \emph{MSH2/6}, omodimeri che creano un sistema pi\`u preciso e specializzato.
			Questi reclutano una polimerasi processiva per risintetizzare il filamento.

\section{Nucleotide excision repair}

	\subsection{Panoramica}
	Il nucleotide excision repair \emph{NER} \`e un sistema di riparazione che agisce prima dell'arrivo della forcella di replicazione.
	Il processo consiste di un escissione di un piccolo frammento di DNA.
	Percepisce e rimuove distorsioni dell'elica come quelle dovute ai raggi UV.
	Sorveglia il filamento trascritto dai geni attivi ed \`e una parte del global genome repair \emph{GGR}.

	\subsection{Escissione}
	Il meccanismo riconosce addotti ingombranti nella doppia elica, punti in cui \`e distortaa.
	Complessi proteici \emph{AAB} la legano e cambiano la conformazione della doppia elica.
	Questi reclutano un'endonucleasi \emph{CC} che rompe un legame fosfodiestere in due punti del filamento.
	Le elicasi svolgono il filamento che viene escisso.
	A questo punto una DNA polimerasi fedele e processiva pu\`o subentrare e sintetizzare il filamento mancante.

	\subsection{Eucarioti}

		\subsubsection{Reclutamento}
		Il reclutamento del sistema \emph{NER} avviene grazie allo stallo di una RNA polimerasi che incontra una distorsione.
		Questa recluta \emph{CSA} e \emph{CSB} che rimuovono la RNA polimerasi.

		\subsubsection{Processo}
		\emph{XPCHR23B} riconosce \emph{CSA} e \emph{CSB} e recluta \emph{TFIIH} associato a \emph{XPD} e \emph{XPB}, elicasi con polarit\`a opposta.
		\emph{XPG} e \emph{XPF} incidono a monte e a valle uno dei due filamenti escindendone un frammento.
		Una polimerasi pu\`o pertanto ora resintetizzare il filamento.

	\subsection{Confronto con \emph{MMR}}
	\emph{NER} e \emph{MMR} si differenziano per la lunghezza del frammento escisso, pi\`u lungo per \emph{MMR} e per il fatto che \emph{NER} non discrimina secondo l'et\`a del filamento.

		\subsubsection{Fotoriattivazione}
		La fotoriattivazione \`e un metodo di riparazione diretto dei dimeri di timina.
		I batteri possiedono un enzima fotoliasi in grado di riconoscere e riparare il dimero utilizzando energia luminosa.
		In queste cellule la luce attiva anche il \emph{NER} meno specifico.

\section{Base excision repair}

	\subsection{Panoramica}
	Il base excision repair \emph{BER} \`e un processo di riparazione per escissione di basi.

		\subsubsection{Rimozione delle basi azotate}
		Le glicolasi si occupano di rimuovere le basi azotate modificate chimicamente.
		Lo fanno idrolizzando il legame glicosidico tra la base e lo zucchero.
		IL taglio lascia un sito apurinico o apirimidinico.
		\emph{AP-endonucleasi}  riconoscono il sito senza base e rimuovono i due legami fosfodiestere permettendo l'attivit\`a della DNA polimerasi $\beta$.

		\subsubsection{Riconoscimento dell'errore}
		Le glicolasi riconoscono l'errore di basi danneggiate grazie alla percezione della stabilit\`a del legame tra due basi di diversi filamenti.
		Se la base \`e giusta e debolmente associata alla complementare viene rimossa.

		\subsubsection{Long patch \emph{BER}}
		Pu\`o essere che venga reclutata una polimerasi pi\`u processiva che formi dei flap, tagliati da un'endonucleasi e sigillati da una DNA ligasi nella long patch \emph{BER}.

	\subsection{Guanina ossidata}
	La guanina ossidata pu\`o essere fonte di un errore in quanto la $8$-idrossiguanosina incorpora un'adenina durante la replicazione del DNA.
	Questo evento pu\`o non essere corretto dalla DNA polimerasi.

		\subsubsection{Glicosilasi fail-safe}
		Una glicosilasi fail safe pu\`o intervenire in questo caso andando a rimuovere la base $A$ normale con una $C$ riducendo il tasso di errore, complementare alla guanosina ossidata.
		Viene anche detta glicosilasi di guardia.

		\subsubsection{$\mathbf{8}$-idrossiguanina}
		La $8$-idrossiguanina o dGTPasi si occupa di salvaguardare i GTP impedendo loro di inserire guanine ossidate nella sintesi del DNA.

		\subsubsection{Glicosilasi \emph{OGG1}}
		\emph{OGG1} \`e una glicosilasi deputata a staccare le guanine ossidate.

	\subsection{Isole CpG}
	Le isole CpG sono isole substrato per enzimi di modifiche epigenetiche come metiltrasferasi che metilano la citosina.
	La modifica chimica pu\`o portare alla variazione di una $C$ in una $T$.

		\subsubsection{Sistema di riparazione}
		Il sistema di riparazione coinvolge la glicosilasi \emph{MBI4}.
		Questa stacca le $T$ appaiate con $G$ nonostante siano entrambe basi naturali e avvia il sistema di riparazione.

\section{Rottura di entrambi i filamenti della doppia elica}

	\subsection{Panoramica}
	La rottura di entrambi i filamenti della doppia elica o \emph{DSB} pu\`o essere riparata attraverso diversi meccanismi.
	Tutti questi hanno in comune il substrato: un estremit\`a che espone il \emph{$3'$-OH} complementare a uno integro usato come stampo.

		\subsubsection{Formazione delle rotture}
		Le rotture a doppia elica possono generarsi in caso di replicazione incompleta durante la segregazione dei cromosomi o durante la replicazione in base al danno alle basi.

	\subsection{Non homologous end joining}
	Non homnologous end joining \emph{NHEJ} \`e un processo in cui nel punto di rottura vengono reclutate \emph{Ku70/80} alle terminazioni in modo da evitare la loro digesione.
	Queste reclutano \emph{DNA-PKcs} che tiene unite le due estremit\`a che vengono sigillate da una ligasi.
	Le due estremit\`a possono essere generate attraverso endonucleasi.

		\subsubsection{\emph{DNA-PKcs}}
		\emph{DNA-PKcs} \`e una grande proteina con molte eliche in grado di associarsi alle estremit\`a del DNA che devono essere vicine.

	\subsection{Ricombinazione omologa}
	La ricombinazione omologa \`e un processo attivo solo quando \`e presente un filamento omologo, ovvero successivamente alla replicazione del genoma.
	Se \emph{Ku70/80} non arrivano in tempo alle estremit\`a endonucleasi espongono \emph{$3'$-OH} su cui si aggregano proteine che attivano una strand invasion al cromatidio fratello.
	In questo modo viene recuperata l'informazione pera a causa della digestione.
	Avviene uno strand displacement che sigilla il doppio filamento.
	Il processo \`e preciso e fedele, ma pu\`o avere errori quando a monte e valle del sito di taglio sono presenti sequenze parzialmente ripetute che portano a appaiamenti di sequenze complementari erronee che diventano flaps e rimossi.
	I filamenti vengono uniti da ligasi.

	\subsection{Metodi di mantenimento della rottura}
	Ibridi a RNA-DNA o molecole di RNA sono utili per riparare le rotture di DNA.
	Il complesso \emph{MRN} \`e un'alternativa a \emph{Ku70/80}.
	\`E formato da tre subunit\`a con un dominio esonucleasico che digerisce una delle due molecole di DNA rotto.
	Alla rottura viene attivato un processo di trascrizione generando un ibrido DNA-RNA grazie a \emph{RNAPII}.
	\emph{RNAasiH} rimuove i ribonucleotidi che possono essere sostituiti dal DNA.
