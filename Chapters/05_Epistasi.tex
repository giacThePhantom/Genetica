\chapter{Epistasi}
\section{introduzione}
\subsection{Fenocopia}
Esistono pi\`u genetiche a cui corrisponde un fenotipo. Si deve pertanto introdurre la fenocopia. Fenotipicamente si osserva lo stesso tratto ma facendo l'analisi genetica si osserva
che allo stesso fenotipo corrispondono potenziali genotipi diversi. 
\subsubsection{Colorazione dell'occhio di Drosophila}
Esistono due percorsi biosintetici controllati da alcuni geni, uno $b$ e uno $v$ che producono pigmenti: il vermiglio e il marrone. Questa \`e la genetica dell'esempio precedente. 
Esiste poi un altro gene $w$ responsabile del trasporto o della possibilit\`a che i pigmenti prodotti dalle vie biosintetiche arrivino nel posto e momento giusto per produrre il 
colore rosso. Il fenotipo dell'occhio bianco pu\`o pertanto comparire quando i $b$ e $v$ sono in omozigosi recessiva (esempio precedente), ma anche quando si trova in omozigosi
recessiva nel gene $w$ in quanto questo non \`e pi\`u funzionale e non trasporta i pigmenti al punto richiesto. 
\subsection{Pathway genetici}
I geni non lavorano in modo isolato e questo fa s\`i che si instaurino delle interazioni tra geni che possono complicarsi quando alcuni alleli sono generati. Le collaborazioni tra i 
geni portano alla creazione di pathway che possono essere di tipi diversi:
\begin{itemize}
	\item Pathway biosintetici: geni che producono un composto molecolare alla loro fine.
	\item Pathway di trasduzione del sengale: fanno piccole modifiche con la capacit\`a di trasferire segnali e modifiche a livello trascrizionale. 
	\item Pathway di sviluppo che sono legati al ruolo di geni che influenzano alcuni aspetti di crescita, differenziamento, parti del corpo o strutture. 
\end{itemize}
L'intuizione delle fenocopie riporta all'esempio. Ci pu\`o essere una sorpresa negli incroci mendeliani. Si considerino due fiori con fenotipo recessivo bianco. Incrociandoli si ottiene
una progenie composta da fiori con colorazione selvatica. Questo fenomeno si descrive come un fenomeno di complementazione. L'incrocio di due genomi con un fenotipo recessivo porta
alla ricomparsa del fenotipo dominante o selvatico. Un modo per interpretare questa contraddizione \`e quella di proporre che i due recessivi siano fenocopie, una genetica complementare:
si incrociano $CCpp$ con $ccPP$: due geni controlano il fenotipo bianco. Incrociando tra di loro piante $ccPP$ dovrebbero dare sempre lo stesso fenotipo. Accettando questa 
interpretazione si impone che tutta la $F_1$ sia eterozigote doppia $CcPp$. Autofecondando ora $F_1$ i risultati dovrebbero seguire la stessa regola di incrocio di ibrido con 
assortimento indipendente. Scomponendo l'incrocio di ibrido ci si attende in $F_2$ $\frac{3}{4}$ con almeno un allele dominante e $\frac{1}{4}$ recessivo. La stessa cosa con $P$ e
se i fenomeni sono indipendeti si ha un ipotetico $9:3:3:1$ e si deve vedere la risultante delle combinazioni per chiedersi le classi fenotipiche. I $\frac{9}{16}$ $C-P-$ hanno
fenotipo viola, i $\frac{3}{16}$ di $C-pp$ dovrebbero avere fiori bianchi cos\`i come $ccP-$ e $ccpp$. La combinazione dei tre eventi mutualmente esclusivi ci si aspetta un rapporto
$9:7$ tra i fenotipi. La deviazione \`e dovuta all'interazione funzionale di alelli di geni diversi. Interpretando questo fenomeno come mascheramento (un gene controlla e maschera il 
fenotipo dell'altro) ci si deve chiedere qual \`e il gene che maschera. Questo \`e un caso di epistasi recessiva doppia e senza comprensione del percorso biosintetico non si \`e in
grado di capire quale gene domina sull'altro. 

