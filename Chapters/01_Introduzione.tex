\chapter{Introduzione}
Si intende per genetica lo studio della trasmissione dell'informazione genetica da genitori a figli, come questa viene organizzata e decodificata per portare ad un fenotipo rilevante. 
Si considera la sua organizzazione, divisione, frammentazione utilizzo e modifica nel corso delle varie divisioni di cellule somatiche o germinali. Il DNA svolge il fondamentale
ruolo di contenere e permettere la decodifica di informazioni essenziali per la vita. Si indica con mutazione de novo una mutazione che accade durante l'osservazione di un sistema.
Il soggetto di studio della genetica sono sistemi modello animali o cellulari utilizzati in quanto pi\`u semplici da coltivare e studiare. Lo studio si svolge sulle correlazioni 
genotipo-fenotipo, analisi dei pedigree e loro ricostruzione per capire da dove viene il rischio di portare un fenotipo. L'utilizzo dei sistemi modello per lo studio di patologie o 
eventi caratteristici dell'uomo \`e legittimato dal fatto che l'uomo e il sistema modello sono imparentati tra di loro. I modello sono pertanto predittivi di fenomeni e dell'effetto di 
mutazioni nella specie umana. I sistemi modello si dividono in:
\begin{itemize}
	\item Unicellulari con vita coloniale.
	\item Pluricellulari semplici (poche cellule).
	\item Pluricellulari complessi. 
\end{itemize}
\section{Il DNA come base molecolare dell'ereditariet\`a}
Nei primi anni $50$ si tentava di determinare quale molecola contenesse il materiale genetico. Per determinarla si sono evidenziate alcune sue caratteristiche:
\begin{itemize}
	\item Deve contenere informazioni complesse e variegate in modo da essere in grado di dare origine alle molteplici forme viventi: generare ovvero fenotipi diversi. 
	\item Deve essere presente in tutte le forme viventi.
	\item Deve essere stabile e capace di replicazione fedele. 
	\item Deve essere capace di subire modificazioni permanenti o mutazioni.
	\item Deve trovarsi nel nucleo e far parte dei cromosomi.
	\item Deve essere in grado di esprimesi, definire e codificare un fenotipo. 
\end{itemize}
Fino all'inizio degli anni $50$ si riteneva che fossero le proteine queste molecole in quanto possedevano la complessit\`a di sequenza e funzione necessaria alle caratteristiche 
elencate. Il tardivo riconoscimento del DNA si deve alla mancanza di conoscenze precise sulla sua composizione e struttura. 
\subsection{Isolamento del DNA}
Il DNA viene scoperto tra il $1868$ e il $1869$ da Miescher, medico sperimentale che durante il suo studio a Tubingen si dedica a esplorare il nucleo delle cellule. Per farlo utilizza
bende usate piene di pus, materiale di scarto da cui isola prima le cellule e il loro nucleo. Le bende vengono lavate in acqua, una soluzione di solfato di magnesio consente l'estrazione
del nucleo da cui vengono rimossi i lipidi con acqua ed etere. Utilizzano acidi blandi si nota la formazione di un precipitato  che pu\`o essere risospeso usando una soluzione lievemente
alcalina. Attraverso saggi alla fiamma Miescher nota come sia presente molto fosforo e lo zolfo sia assente. Miescher riesce pertanto ad isolare una nuova molecola che chiama nucleina. 
Utilizzando pepsina determina che non \`e una proteina. I suoi colleghi successivamente confermano il risultato estraendo la nucleina da eritrociti nucleati di pesce confermando la 
sua natura pervasiva. Un altro ricercatore nel $1889$ segue il protocollo e ritiene di aver isolato una sottocomponente della nucleina che chiama acido nucleico. Successivamente 
Miescher riconferma l'esperimento con lo sperma di salmone isolando da esso la nucleina e la sua presenza in cellula germinali fa sorgere domande sul suo ruolo nell'ereditariet\`a. 
\subsection{Caratterizzazione chimica e prima teorizzazione della struttura}
Albrecht Kossel determina che la nucleina \`e composta da basi azotate, zucchero e fosfato e nella prima decade del $900$ Levene e Steudel studiano la struttura della macromolecola e 
il primo propone una struttura a tetranucleotidi, il DNA come una macromolecola formata da tetrameri contenenti le quattro basi legate tra di loro poste una sopra l'altra. Questo modello
viene ampiamente accettato ma l'omogeneit\`a della struttura rende improbabile che questa possa codificare informazioni complesse e pertanto prevale l'idea che il DNA abbia una funzione
principalmente strutturale e non si occupi di trasferire le informazioni. 
\subsection{Determinazione del rapporto tra le basi}
Chargaff, biochimico, studia il DNA utilizzando cromatografia su carta: prendendo il DNA da diverse sorgenti passa alla cromatografia le molecole caratterizzando il rapporto quantitativo
relativo tra le componenti e nota come le basi siano in percentuali diverse (rapporti variabili tra gli organismi), pertanto la struttura a tetranucleotide non pu\`o essere quella 
corretta, riscoprendo il valore del DNA. 
\subsection{Studi del pneumococco}
\subsubsection{Griffith}
Griffith nel $1928$ studia a Londra il comportamento del pneumococco e ne osserva due tipi, un primo liscio di tipo IIIS (formano una sovrastruttura di zuccheri) e uno rugoso di tipo 
IIR. La forma IIIS \`e aggressiva e in grado di infettare topi con la polmonite. Successivamente compone un esperimento con tre beute di controllo e una di studio. Nella prima beuta
fa crescere dei batteri di tipo IIIS virulenti e li fa crescere, iniettandoli poi nel topo nota come questo soffre e muore, nel suo sangue si trova crescita batterica. Nella prima 
beuta fa crescere batteri di tipo IIR non virulento, iniettandoli poi nel topo questo non soffre e non si trovano batteri nel suo sangue. Nella terza beuta pone i batteri virulenti e
li lisa attraverso il calore: iniettando nel topo i corpi cellulari questo non soffre e non si trovano batteri nel suo sangue. Nella quarta beuta mischia i batteri non virulenti di 
tipo IIR con il lisato di IIIS, mettendoli in coltura e iniettando il topo questo soffre e muore e si trova il batterio di tipo IIIS nel suo sangue. Griffith scopre pertanto un 
principio trasformante, una caratteristica permanente che pertanto non \`e dovuta al trasferimento della capsula ma che \`e diventato patrimonio dei batteri. Non riesce a determinare
la natura chimica del principio.
\subsubsection{Dawson e Sia}
Dawson e Sia ripetono l'esperimento di Griffith senza iniettare le cellule ma mischiando il lisato IIIS e IIR in coltura e piastrando le cellule sulle piastra di coltura e determinano
che il fenomeno \`e esterno al topo: la cocoltura \`e sufficiente per far comparire colonie IIR virulente. Si ha la conferma del principio trasformante. 
\subsubsection{Avery}
Nel $1944$ nel laboratorio diretto da Avery al Rockfeller institute viene svolto un esperimento per determinare la molecola responsabile del principio trasformante. Si fanno crescere
i batteri virulenti in coltura e li si uccide. Il lisato viene diviso in tre provette separate nelle quali vengono introdotte rispettivamente RNAasi, proteasi e DNAasi. Mischiando 
i lisati con IIR si osserva quando si ottiene la manifestazione fenotipica del principio trasformante: soltanto il lisato trattato con DNAasi non dimostra il passaggio di informazione
e pertanto la molecola trasformante \`e il DNA. 
\subsection{Studi dei batteriofagi}
La phage church in particolare Hershey e Chase, un gruppo di microbiologi appassionato alla ricerca dei batteriofagi, entit\`a visibili al tempo solo indirettamente osservando la loro 
capacit\`a di uccidere batteri, sviluppa un esperimento per visualizzare la molecola responsabile del passaggio genico. Per farlo si sfrutta il fatto che la molecola \`e ricca di 
fosforo ma mancante zolfo, elemento presente invece nelle proteine. Pertanto si usa un terreno contenente un isotopo radioattivo dello zolfo dove vengono fatti crescere i batteri e 
infettati con il batteriofago T2 in modo da avere una progenie di fagi marcata. Recuperando i fagi radioattivi e infettando cellule di E. coli non radioattive le si fa infettare, si 
separano con un frullatore e si centrifuga per ottenere un pellet di cellule batteriche. Osservando dove si trova la radioattivit\`a si nota come questa rimane nel surnatante e non \`e 
nei batteri che hanno subito l'infezione: questi successivamente subiscono lisi e la progenie fagica non \`e radioattiva. Si conclude che le macromolecole marcate con lo zolfo non sono 
importanti per produrre progenie fagica. Si ripete l'esperimento con fosforo radioattivo e si nota come la radioattivit\`a in questo caso si trova nel pellet e non nel surnatante: il 
tracciante informa che molto probabilmente il DNA \`e entrato nel batterio e la popolazione di fagi da esso derivante \`e parzialmente radioattiva. Unendo i due risultati si determina 
che le proteine non partecipano all'infezione mentre l'acido nucleico viene trasmesso nei batteri e riproposto nella progenie: l'elemento importante per la produzione di nuovi fagi \`e 
il DNA e non le proteine che avranno ruolo di rivestimento, di formare il capside che rimane fuori dalla cellula. Le proteine hanno pertanto la funzione primaria di proteggere e 
trasportare il materiale genetico. 
\subsection{Alternative al DNA a doppio filamento}
Fraenkel-Conrat e Singer studiando il virus del mosaico del tabacco notano come questo sia formato da una struttura proteica molto regolare che forma un un barilotto contenente una 
singola molecola di RNA. Si trovano due varianti del virus A e B e si riesce a smontare e rimontare i virus in provetta in modo da scambiare il capside tra una popolazione A e una B. 
Gli ibridi chimerici ora vengono utilizzati per infettare delle foglie e studiando la progenie si determina che il tipo \`e determinato dall'RNA e non dalle proteine. Per questo tipo 
di virus \`e l'RNA la molecola della vita. Gierer e Shramm completano e confermano gli esperimenti notando come l'evoluzione ramificandosi ha scelto diverse strategie per la molecola
della vita: DNA a filamento singolo (ssDNA) e doppio (dsDNA) o RNA a filamento singolo (ssRNA) e doppio (dsRNA). 



\section{La rivoluzione del DNA}
Si intende per rivoluzione del DNA la capacit\`a di leggere, contare e scrivere il DNA iniziata con il nuovo millennio. 
\subsection{Leggere il DNA}
La tecnica di lettura del DNA viene scoperta da Sanger (da cui prende il nome) che inventa e migliora metodi per leggere il DNA: dei nucleotidi modificati in $3'$ in modo che blocchino
la sintesi da parte della DNA polimerasi quando vengono aggiunti alla catena nascente e accoppiandoli con fluorocromi si pu\`o ricostruire la sequenza del DNA in modo lineare in quanto
i colori permettono di determinare la posizione delle basi. Questo processo \`e veloce ed automatizzabile. Il metodo Sanger accoppiato con la reazione a catena della polimerasi (PCR)
permette di scegliere un segmento di DNA e crearne moltissime copie  in modo da aumentare il segnale e la capacit\`a di riconoscerne la sequenza. All'inizio del $2000$ si annuncia il 
primo draft del genoma umano ottenuto dallo studio del DNA di diversi individui. Il genoma umano \`e composto da \numprint{3200000000} di basi. Ora si necessita di decodificarlo 
andando a perseguire la funzione del gene. Lo sviluppo tecnologico ha abbassato drasticamente i costi di sequenziamento rendendo meno importanti lo studio del modello e dei pedigree e 
la ricerca dei tratti rari. Sono nati genome browser, repository contenenti informazioni generali sul genoma dell'uomo e di altri organismi. 
\subsection{Contare il DNA}
Il next generation sequencing permette di contare ed annotare informazioni sul DNA, diventa uno strumento quantitativo per capire come l'informazione viene usata e decodificata. Si 
osserva come alcuni frammenti si sono spostati, la scomparsa di un frammento e le sequenze invertite (inversioni). Grazie alla sua lettura e quantificazione si capisce la struttura della
cromatina, capire i punti di inizio di trascrizione da parte dell'RNA polimerasi, tentando di capire l'organizzazione del genoma, numerando i geni e separando la frazione codificante
da quella non funzionale e ingombrante. Si noti come la trascrizione genera plasticit\`a in quanto ci sono molte possibilit\`a di splicing alternativo e siti di poliadenilazione 
diversi. Si nota come studiando il genoma umano invece dei teorizzati \numprint{100000} geni le analisi iniziali ne hanno trovati \numprint{35000} e quel numero \`e stato diminuito fino 
a \numprint{21000} e molti geni codificano solo per RNA e non per proteine come prodotto finale. 
\subsection{Scrivere il DNA}
