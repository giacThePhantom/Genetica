\chapter{Genetica batterica}

\section{Coniugazione}

	\subsection{Lederberg e Tatum}
	Lederberg e Tatum dimostrano lo scambio di materiale genetico tra i batteri attraverso un esperimento che utilizza due ceppi di E. coli:
	\begin{multicols}{2}
		\begin{itemize}
			\item \emph{Y10} auxotrofo per treonina, leucina e tiamina.
			\item \emph{Y24} auxotrofo pre cisteina, fenilanalina e biotina.
		\end{itemize}
	\end{multicols}
	Notano come mischiando i due ceppi si formano colonie in grado di crescere su terreno minimo.
	L'aggunta di materiale genetico esogeno compensa l'assenza di tre geni.

	\subsection{Davies}
	Davies dimostra la non filtrabilit\`a degli agenti responsabili di ricombinazione genetica attraverso un tupo a $U$.
	
	\subsection{Hayes}
	Hayes dimostra \`e il trasferimento genico \`e unidirezionale.
	Il donatore possiede un fattore di fertilit\`a che permette la coniugazione che pu\`o essere perso nel processo di trasferimento.
	Il fattore di fertilit\`a $F$ \`e un plasmide con una serie di geni che consentono al batterio di sintetizzare un pilo che mette in contatto e avvicina le due cellule.

	\subsection{Processo}
	Durante la coniugazione il fattore $F$ rende il donatore capace di sintetizzare il pilo, un poro tra le due cellule.
	Un filamento di $F$ passa nel poro e viene convertito in dsDNA.
	A completamento della coniugazione il ponte viene rimosso e i batteri contengono una copia di $F$.
	Il fattore $F$ \`e un plasmide eposomale e che pu\`o integrarsi con il genoma permettendo il trasferimento di diversi geni presenti sul genoma.
	Questa coniugazione pu\`o essere interrotta andando a definire l'ordine dei geni nel genoma.

\section{Trasformazione}
La trasformazione \`e un fenomeno in cui alcune cellule batteriche sono in grado di assorbire DNA dalla soluzione.
Questi frammenti possono ricombinare con sequenze omologhe.
La proporzione di trasformanti con pi\`u informazioni contemporanee decresce con la distanza relativa tra i geni.
La percentuale di cotrasformanti stabilisce la distanza relativa di due geni.

\section{Trasduzione}

	\subsection{Trasduzione generalizzata}

		\subsubsection{Lederberg e Zinder}
		Lederberg e Zinder fanno un esperimento con un tubo a $U$ in cui vengono posti due ceppi di salmonella.
		Dal lato $A$ uno auxotrofo per alcuni metaboliti, dal lato $B$ un altro auxotrofo per altri.
		Il filtro lascia passare solo la sospensione liquida.
		Questo esperimento mostra come nascono cellule capaci di crescere cellule capaci di crescere in terreno minimo selettivo nel ceppo $A$.
		Questo processo \`e avvenuto grazie a virus e si dice trasduzione.

		\subsubsection{Meccanismo}
		Alcune particelle fagiche entrate nel batterio sono in grado di catturare frammenti del genoma batterico digerito ottenendo una progenie di fagi con geni del batterio.
		Questi quando infettano un batterio non ne provocano la lisi ma lo arricchiscono con il DNA batterico trasdotto.
		I geni vicini presentano una maggiore probabilit\`a di essere cotrasdotti.

	\subsection{Trasduzione specializzata}

		\subsubsection{Ciclo lisogenico}
		Il ciclo lisogenico dei batteriofagi permette loro di iniettare il genoma all'interno di batteri e integrarsi con il genoma batterico.
		Questo \`e tipico di fagi come il $\lambda$ e genera la trasduzione specializzata.
		L'inizio del ciclo lisogenico consiste di una ricombinazione sito-specifica attraverso sequenze parzialmente omologhe.
		Dopo questa il genoma viene integrato nel batterico fra il gene \emph{gal} e il \emph{bio} distanziandoli.
		Il processo \`e reversibile attraverso escissione.
		L'escissione pu\`o creare un fago contenente genoma batterico adiacente al sito di escissione, favorendo il trasferimento di geni adiacenti al sito di inserzione.

		\subsubsection{Hershey e Rotman}
		Hershey e Rotman studiano elementi genetici del genoma fagico attraverso mutanti di $T2$: $h^{+\-}$ $r^{+\-}$.
		$h^+$ \`e in grado di infettare e lisare E. coli $B$ ma non $B/2$, mentre $h^-$ entrambe.
		$r^+$ provoca piccole placche, mentre $r^-$ placche molto grandi.
		Per stimare la posizione dei geni fagici vengono usate le differenze fenotipiche.
		Partono da due fagi con caratteristiche complementari facendo un infezione mista in una miscela di $B$ e $B/2$.
		In una placca confluente si misura l'attivit\`a litica attraverso la presenza di placche.
		L'evento di interesse \`e la ricombinazione tra i due geomi fagici con informazioni complementari.

		\subsubsection{Benzer}
		Benzer si interessa alla dimensione fisica dei geni dimostrando che non tute le porzioni di un gene sono uguali.
		Utilizza $T4$ selvatico in grado ci lisare E. coli $B$ e un mutante $rII$ capace di crescere su $B$ con piccole placche ma incapace di lisare $K$.
		L'esperimento usa due mutanti di $rII$ con stesso fenotipo ma ottenuti in modo indipendente.
		Un infezione mista dovrebbe portare il genoma di entrambi i mutanti.
		L'infezione mista dovrebbe pertanto essere in grado di produrre molecole diverse per la presenza di sequenze geniche distinte.
		Le due informazioni potrebbero complementare e lisare $K$.
		Nota per\`o due gruppi di complementazione che non complementano in quanto con mutazioni nella stessa unit\`a funzionale.
		Benzer conclude pertanto che il gene ha una dimensione fisica e due mutanti possono avere mutazioni diverse per l'allele coinvolto o mutazioni eteroalleliche.
