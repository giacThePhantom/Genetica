\chapter{Mendel}
\section{Vita}
Si noti come Mendel costruisce i suoi esperimenti senza sapere che cosa fossero i geni e che controllano i caratteri, che sono localizzati sui cromosomi e che le cellule hanno la
possibilt\`a di segregare i cromosomi attraverso meiosi. Mendel non parte da zero ma \`e un uomo del suo tempo e beneficia della cultura del tempo: anche in famiglia \`e a contatto con
una tradizione di chi lavorando sui campi tenta di migliorare la produzione incrociando le piante. Mendel nasce nel $1822$ in Repubblica Ceca, in un piccolo villaggio, centro agricolo. 
Nasce da una famiglia non molto agiata di contadini, in un ambiente dove l'ibridazione fra piante \`e pane quotidiano. Viene suggerito alla famiglia di farlo studiare nel ginnasio dove
avr\`a dei buoni risultati e quando dovr\`a spostarsi per poter accedere all'universit\`a si trova in un ambiente difficile e da questa esperienza torna e decide di entrare nel $1843$ 
in un convento di Agostiniani, studiosi che abitavano in un convento a Brno ed erano ben visti per la loro competenza. Non ha successo come insegnante. Conosceva molto la sistematica
e la nomenclatura e descrizione delle forme viventi con il sistema di Linneo e aveva idee radicali sul concetto di ereditariet\`a. Completa i suoi studi e prende i voti nel $1847$. 
L'abate gli permette di continuare gli studi facendolo spostare a Vienna all'universit\`a. Mendel matura l'interesse per dimostrare di trovare delle regole nell'analisi dei fenotipi 
di piante che vengono incrociate in modo opportuno per trovare la comparsa di varianti. Mendel inizia gli esperimenti incrociando topi per cercare di capire se poteva trovare
qualche pattern, ma viene scoraggiato dall'abate. Come seconda scelta lavora sulle piante. Ultimati questi esperimenti viene scritta ''Esperimenti nell'incrocio delle piante" dove Mendel
racconta le sue scoperte dove viene pubblicato il suo articolo che non ha una grossa diffusione. Le scoperte di Mendel rimangono nascoste per alcuni anni. Mendel per discutere della sua 
scoperta con le persone pi\`u influenti scrive a N\"ageli per intavolare una discussione e continuare la sua opera. Lo scambio di lettere \`e molto saltuario. N\"ageli dice a Mendel di 
ricreare l'esperimento con un altro modello sperimentale e suggerisce di usare un modello senza la tendenza di generare ibridi. In parte anche per questo Mendel abbandona l'attivit\`a 
sperimentale anche perch\`e diventa responsabile del monastero e avr\`a interazioni difficili con l'autorit\`a locale.
\subsection{Esperiment}
L'approccio vincente di Mendel \`e quello di aver costruito l'esperimento con pazienza scegliendo il modello sperimentale: prova prima con pi\`u piante e alla fine sceglie il pisello da 
giardino. Inizia a selezionare i caratteri da investigare per valutare cosa succede durante gli incroci. \`E un lavoro metodico che necessita di alcuni anni: scarta diverse 
caratteristiche fino ad arrivare a $7$. Mendel seleziona il colore, forma e rivestimento del seme, colore e forma del baccello, colore e forma del baccello e altezza del fusto. La pianta
\`e comoda in quanto gli stami e le antere sono confinati in un astuccio, nonostante la pianta tende a fare autofecondazione, Mendel pu\`o tagliare le antere prima che maturi il polline 
e poi andare a fecondare manualmente in modo da realizzare l'incrocio desiderato. La prima cosa che Mendel scopre \`e che un carattere \`e dominante e l'altro recessivo: la forma del 
seme liscia viene detta dominante rispetto alla forma rugosa per esempio. I primi esperimenti si occupano dell'incrocio di piante che differiscono per una sola caratteristica. Le piante 
da cui Mendel parte devono essere linee pure: autofecondate per diverse generazioni in modo da dimostrare che tutta la progenie abbia il fenotipo scelto. Mendel introduce le lettere del 
linguaggio mendeliano ancora in utilizzo. La lettera maiuscola determina il carattere dominante. Oltre a preparare il sistema sperimentale genera esperimenti con numerologia 
significativa. Nel primo esperimento scopre pertanto che una delle due caratteristiche \`e dominante: nella generazione $F_1$ generata dall'ibridazione la popolazione torna omogenea e 
presenta una sola caratteristica. Mendel conduce gli esperimenti fino a che fosse necessario per chiarire l'ipoteso: si deve superare la prima generazione per capire dove finisce il 
fenotipo recessivo: si autofeconda $F_1$ e riappare il carattere recessivo e contando i numeri si rende conto che il rapporto tra carattere dominante e recessivo approssima il $3:1$. 
Dopo questo comincia a combinare caratteri, con incroci che coinvolgono due fenotipi e riesce di nuovo a stabilire delle regole di segregazione: nella $F_1$ rimangono visibili solo i 
fenotipi dominicani e nella $F_2$ ricompaiono i caratteri recessivi in combinazioni non presenti nella linea parentale: i caratteri segregano e sono collegati a qualcosa che si separa e 
si assortiscono e nota come l'assortimento sia indipendente. 
\subsection{L'eredit\`a Mendeliana}
La scoperta di Mendel rimane sepolta in qualche biblioteca ma viene riscoperta verso il $1900$ quando si vede la 
comparsa nella letteratura scientifica del nome di Mendel e delle sue scoperte grazie a De Vries, Tschermak e Correns che fanno esperimenti di ibridazione citando Mendel. De Vries 
inoltre deduce che nuove varianti possono essere generate dai mutanti grazie ai quali si pu\`o generare l'evoluzione. Correns che lavora a T\"ubingen, studente di N\"ageli che riprende
il suo lavoro. Si aggiungono ad essi Bateson e Punnet, il primo \`e quello che fa pi\`u degli altri traducendo l'articolo di Mendel in inglese e diffonde il suo lavoro, il 
secondo lo affianca e sostituisce creando i quadrati di Punnet che consente di prevedere la formazione di gameti e il fenotipo di una generazione. Viene scoperta anche un'eccezione. 
\section{Principi mendeliani}
\subsection{Primo esperimento}
Si parte da due caratteristiche diverse: seme rotondo e grinzoso. Si vuole studiare cosa succede incrociando i fenotipi nella progenie. Lo sperimentatore separa polline dalla parte 
femminile, lo prende dalla pianta con la caratteristica che ha scelto. Guardando il fenotipo del seme non si deve aspettare molto: si aprono i baccelli e si osserva. Non si pu\`o 
prendere una pianta a caso ma deve essere una linea pura omogenea e si pu\`o fare l'incrocio. La generazione $P$ o parentale sono le linee pure che vengono incrociate. La prima
generazione o $F_1$ presenta una progenie omogenea con semi lisci. Le piante poi vengono lasciate libere di autofecondarsi e si osserva poi $F_2$ e si nota come la caratteristica 
grinzosa non era scomparsa ma ancora presente, inoltre si deduce con una proporzione come i semi rotondi e grinzosi stanno in rapporto $3:1$. Mendel comincia a pensare che i caratteri
siano legati a fattori discreti ma comincia ad usare un simbolismo attraverso lettere: una linea pura con fenotipo dominante e in omozigosi $RR$ e per l'altra linea pura con fenotipo
rugoso $rr$. La prima ipotesi di Mendel, per noi formazione di gameti attraverso la meiosi, \`e che i caratteri discreti alla base del fenotipo si separano durante la formazione di
una nuova generazione e nella fecondazione singole lettere si incontrano: le piante $F_1$ saranno tutte eterozigoti $Rr$, la dominanza fa s\`i che tutti i semi abbiano fenotipo liscio, 
autofecondando $F_1$, la produzione attraverso la separazione di alleli. Pertanto nella fecondazione di due eterozigoti si hanno tre possibilit\`a: $RR$, $Rr$ e $rr$. Gli eventi sono
equiprobabili e sommando per fenotipo si ha un rapporto $3:1$
\subsubsection{Reintrerpretazione in tempi moderni}
Con la cellula diploide e due cromosomi omologhi, eterozigote per il gene legato al carattere forma del seme. Questa cellula pu\`o andare incontro a meiosi per formare nuovi gameti e 
per farla, si duplica il DNA, i cromosomi sono composti da due cromatidi fratelli con $4$ lettere, la cellula viene divisa meioticamte: vengono separati i due omologhi e poi i cromatidi
fratelli formando cellule aploidi con solo un cromosoma e una lettera. Partendo da un'eterozigote si formano cellule aploidi in pari numero dominanti e recessivi, vero anche se avviene
un crossing-over meiotico che scambia porzioni dei cromatidi non fratelli. 
\subsubsection{Quadrati di Punnet}
\subsubsection{Ipotesi di Mendel}
Ci sono dei fattori responsabili della trasmissione ereditaria dei caratteri e sono unit\`a discrete (geni) che compaiono in coppie, esistono in forme alternative e si separano 
(segregano) durante la formazione dei gameti. Le piante possono avere due alleli equivalenti (omozigoti) o due alleli diversi: uno dominante e uno recessivo (eterozigoti). Mendel 
conclude che facendo l'incrocio tra individui che differiscono per un solo carattere possono differire solo per una coppia di alleli e se si hanno delle linee pure $AA$ e $aa$ la 
progenie $F_1$ sar\`a obbligatoriamente eterozigote $Aa$. Facendo autofecondare $F_1$ $Aa\times Aa$ si pu\`o ottenere $F_2$ dove si ritrova il fenotipo scomparso in proporzione di 
$\frac{1}{4}$ e il resto presenta fenotipo dominante di classe omozigote $AA$ ed eterozigote $Aa$. 
\subsection{Reincrocio}
Un altro incrocio di Mendel: si ha un problema con una pianta che mostra il fenotipo dominante si pu\`o non essere sicuri di avere una linea pura omozigote. Per chiarire se la pianta
\`e omozigote od eterozigote il fenotipo non ci pu\`o aiutare, Mendel si inventa un incrocio: incrocio di controllo o reincrocio: si prende una pianta a fenotipo dominante e genotipo 
non noto e incrociarla con una pianta a fenotipo recessivo. Osservando il risultato e la proporzione della progenie si nota il fenotipo della pianta di controllo. 
\subsection{Secondo esperimento}
Il secondo esperimento \`e l'incrocio tra due coppie di caratteri. La generazione parentale presenta omozigosi di due tratti: giallo e liscio $RRYY$ e l'altra \`e verde e rugoso
$rryy$. La pianta con il fenotipo dominante e linea pura si separano le coppie di elementi discreti $RY$ e il fenotipo recessivo $ry$. Questi gameti vengono fatti reincontrare e 
$F_1$ possiede un genotipo $RrYy$ con fenotipo dominante giallo liscio. L'esperimento continua e si autofeconda $F_1$ creando $F_2$. Si formano diversi gameti: $RY$, $Ry$, $rY$, $ry$. 
Non si sa se $R$ e $Y$ si muovono in modo causale o con delle regole precise, non si sa se si formano davvero o con la stessa proporzione. Aprendo il baccello si nota
che si trovano semi gialli e lisci $R-Y-$, gialli e rugosi $rrY-$, verdi e lisci $R-yy$ e verdi e rugosi $rryy$. Si nota la comparsa di fenotipi che non si trovavano nella linea 
parentale. Si dimostra come si assortiscono in modo casuale le coppie di caratteri diversi e i numeri dicono che il processo \`e casuale: $\frac{9}{16}$ giallo e liscio, $\frac{3}{16}$
per giallo e rugoso e verde e liscio e $\frac{1}{16}$ per verde e rugoso. Si nota come i geni stanno sui cromosomi e la coppia allelica $Yy$ sta su un cromosoma e quella $Rr$ un
altro e l'assortimento indipendente \`e il movimento indipendente del movimento dei cromosomi. 
\subsubsection{Assortimento indipendente}
\`E vero che ci si deve sempre aspettare assortimento indipendente. Se i geni sono sullo stesso cromosoma ci si aspetta che siano ereditati insieme. 
\section{Comprensione molecolare degli esperimenti di Mendel}
L'identificazione dei geni responsabili per i tratti di studio di Mendel richiedono dimostrazione che:
\begin{itemize}
	\item Alleli diversi devono essere responsabili di variazione morfologica.
	\item La differenza tra alleli \`e legata a una di DNA. 
	\item I prodotti proteici tra diversi alleli hanno diversa struttura e funzione. 
	\item Le differenze funzionali tra le varianti di proteine hanno effetto sulle variazioni morfologiche.
\end{itemize}
\section{Esempi di tratti mendeliani nell'uomo}
\subsection{Definizioni}
\begin{itemize}
	\item Ogni cromosoma \`e costituito da una successione lineare di geni o loci.
	\item Ogni coppia di cromosomi contiene gli stessi geni nello stesso ordine ma non necessariamente in forma identica.
	\item Il locus \`e la posizione occupata da un gene su un cromosoma.
	\item Gli alleli sono forme diverse di uno stesso gene. 
	\item Il genotipo \`e la costituzione genetica di un individuo, \`e riferito sia ad un singolo gene che al loro insieme.
	\item Il fenotipo \`e la manifestazione fisica di un carattere genetico che dipende dal genotipo specifico e dalla sua interazione con l'ambiente. 
	\item Un carattere \`e una caratteristica di un organismo rilevabile con un qualsiasi mezzo di indagine. 
\end{itemize}
\subsection{Beta-Talassemia}
La beta-talassemia pu\`o essere causata da mutazioni in omozigosi o eterozigosi-composta nel gene della beta-globina $11p15$. Pu\`o originarsi dalla delezione dell'intero cluster di 
geni della beta-globina o delle sequenze $5'$ dal cluster o regione di controllo del locus beta. 
\subsubsection{Descrizione}
La beta-talassemia \`e caratterizzata da una produzione ridotta dell'emoglobina $A$ ($HbA$, $\alpha-2/\beta-2$), che risulta in sintesi ridotta delle catene di beta-globina rispetto
a quelle di alfa-globina causando uno sbilanciamento e eritropoiesi anormale. Il disordine \`e clinicamente eterogeneo. L'assenza di beta-globina causa beta-zero-talassemia, mentre
ridotte quantit\`a di beta-globina individuabile causa beta-pi\`u-talassemia. La beta-talassemia si divide in talassemia maggiore (dipendente dalle trasfusioni), intermedia e minore
(asintomatica). La diversit\`a fenotipica riflette l'eterogeneit\`a delle mutazioni al locus $HBB$, l'azione di molti modificatori secondari e terziari e un grande intervallo di 
fattori ambientali. 
\subsubsection{Caratteristiche cliniche}
\paragraph{Talassemia maggiore}
Infanti affetti da talassemia maggiore non si sviluppano correttamente e sono pallidi con problemi di diarrea, irritabilit\`a, numerosi episodi di febbre e allargamento dell'addome
causato da splenomegalia. Attraverso trasfusioni crescita e sviluppo sono normali fino a $10$ o $11$ anni, successivamente gli individui sviluppano rischi legati al sovraccarico di 
ferro imposto dalle trasfusioni.
\paragraph{Talassemia intermedia}
Pazienti con la talassemia intermedia subiscono effetti eterogenei: pallore, allargamento di ittero, fegato e milza, cambi scheletrici da moderati a 
severi, ulcere nelle gambe, masse extramidollari di midollo eritroide, una tendenza a sviluppare osteopenia e osteoporosi. Le trasfusioni non sono richieste e il sovraccarico di ferro
accade principalmente dal suo assorbimento aumentato causato da eritropoiesi inefficace. 
\paragraph{Talassemia minore} 
I portatori di beta-talassemia sono clinicamente asintomatici. 
\paragraph{Caratteristiche geniche}
La coeredit\`a di alfa-talassemia con omozigote beta-talassemia risulta in un miglioramento della beta-talassemia. Inoltre la beta-talassemia eterozigote \`e associata con manifestazioni
cliniche severe quando coereditata con un gene di alfa-globina in pi\`u: in ognuno dei $5$ casi un cromosoma $16$ trasportava $3$ geni di alfa-globina. Lo stesso aggravamento si trova
con loci alfa triplicati (esempio di interazione genica). 
\subsubsection{Cluster genici della globina}
Esistono diversi geni della globina, prodotti durante diversi stadi della vita di un individuo. Questi si trovano raggruppati in cluster su diversi cromosomi: 
\paragraph{Cromosoma $\mathbf{11}$}
Globina epsilon $\varepsilon$, gamma $\gamma$ G e A a formare $Hb\ F$, delta $\delta$ a formare $Hb\ A2$ e beta $\beta$ a formare $Hb\ A$.
\paragraph{Cromosoma $\mathbf{16}$}
Globina zeta $\zeta\ 2$, zeta $\zeta\ 1$, alfa $\alpha\ 2$  e alfa $\alpha\ 1$.
\paragraph{Composizione emoglobina}
L'emoglobina \`e un tetramero formato da due subunit\`a proveniente dal cromosoma $11$ e due dal cromosoma $16$. La composizione varia in base allo stato di sviluppo:
\begin{itemize}
	\item Embrionica: $\zeta\zeta\varepsilon\varepsilon$, $\alpha\alpha\varepsilon\varepsilon$ o $\zeta\zeta\gamma\gamma$.
	\item Fetale: $\alpha\alpha\gamma\gamma$ ($HbF$).
	\item Postnatale: $\alpha\alpha\delta\delta$ ($HbA_2$) o $\alpha\alpha\beta\beta$ ($HbA$). 
\end{itemize}
\paragraph{Formazione del cluster genico}
Probabilmente il gene cluster si \`e originato da un gene primordiale della globina che si \`e duplicato formando nel cromosoma $22$ il gene della mioglobina e un gene precursore
della $\alpha/\beta$-globina. Quest'ultimo si \`e poi duplicato e diviso nel gene primordiale della $\alpha$-globina e della $\beta$-globina. Questi due poi sono successivamente andati
incontro a duplicazioni multiple che hanno portato la formazione dei cluster genici rispettivamente nei cromosomi $16$ e $11$.
\subsubsection{Gestione clinica}
\paragraph{Trattamento con \emph{5-azacitidina}}
Nel $1982$ la beta-pi\`u-talassemia \`e stata trattata in un uomo di $42$ anni con \emph{5-azacitidina}. Si \`e registrato un aumento di concentrazione di emoglobina. Si nota
l'ipometilazione della $\gamma$-globina e della $\epsilon$-globina oltre a un aumento di mRNA per $\gamma$-globina.
\paragraph{Trapianto di midollo}
Lo studio di pazienti a cui era stato svolto il trapianto del midollo $BMT$ per la cura della talassemi. Il midollo allogenico proveniente da donatori $HLA$-identici e i pazienti 
avevano $\beta$-talassemia ed erano sotto i $16$ anni. Conclusero che il trapianto di midollo offriva un'alta probabilit\`a di sopravvivenza senza complicazioni se il recipiente non
soffriva di epatomegalia o di fibrosi portale. 
\paragraph{Terapia genica}
La terapia genica per la $\beta$-talassemia \`e difficile in quanto richiede produzione massiva di emoglobina in maniera specifica al lignaggio e dalla mancanza di vantaggio selettivo
per le cellule staminali ematopoietiche corrette. In ogni caso dopo la terapia un paziente \`e diventato indipendente dalle trasfusioni e la maggior parte dei benefici deriva da
un clone cellulare a base mieloide dominante. Viene suggerito che la dominanza clonale che accompagna l'efficacia pu\`o essere coincidentale e stocastica o il risultato da un
espansione di una cellula benigna causata dalla mal-regolazione del gene $HMGA2$ in cellule staminali o progenitrici. 
\subparagraph{Background}
La disponibilit\`a dei donatori e i rischi del trapianto limitano il suo uso nei pazienti, pertanto dopo aver stabilito che il trasferimento lentivirale di un gene di $\beta$-globina
marcato potrebbe sostituire la trasfusione di pazienti affetti da $\beta$-talassemia si vuole valutare la sicurezza ed efficacia della terapia genica nei pazienti. 
\subparagraph{Metodi}
Nello studio si ottengono cellule mobilizzate autologhe $CD34+$ da pazienti e si trasducono le cellule in vivo con il vettore LentiGlobin $BB305$ che codifica l'emoglobina adulta
$HbA$ con una sostituzione amminoacida. Le cellule sono state reinfuse nel paziente. Si monitorano poi gli effetti avversi, l'integrazione del vettore e i livelli di replicazione del
lentivirus. 
\subparagraph{Risultati}
Dopo un intervallo di $26$ mesi tutti i pazienti tranne uno hanno smesso di ricevere trasfusioni e i livelli di emoglobina erano vicini al normale. La terapia genica ha pertanto 
eliminato la necessit\`a di trasfusioni a lungo termine senza eventi avversi importanti. 
\paragraph{Trattamento con CRISPR}
$CTX001$ \`e una terapia in ex vivo in cui cellule autologhe sono raccolte dal paziente, CRISPR applica poi la tecnologia di editing genomico alle cellule per fare un cambio genetico
progettato per aumentare l'aumento di livelli di emoglobina fetale. Le cellule sono poi reinfuse e dovrebbero produrre cellule di globuli rossi con emoglobina fetale nel paziente
superando le deficienze di emoglobina. L'edit di CRISPR crea una delezione in $BCL11A$ che codifica un fattore di trascrizione che altrimenti reprime la sintesi di emoglobina fetale. 
Questa terapia \`e efficace anche contro l'anemia falciforme. 
\subsubsection{Genetica della popolazione}
La $\beta$-talassemia \`e uno dei disordini recessivi e autosomiali pi\`u comuni. \`E prevalente nelle popolazioni di mediterraneo, medio oriente, transcaucaso, Asia centrale, 
subcontinente indiano e l'est. \`E comune in popolazioni di discendenza africana. 
\subsection{Anemia falciforme}
\subsubsection{Descrizione}
L'anemia falciforme \`e una malattia multisistema associata con episodi di malattia acuta e danno agli organi progressivo. La polimerizzazione dell'emoglobina che porta 
alla rigidit\`a dell'eritrocita e all'occlusione dei vasi \`e centrale nella fisiopatologia della malattia. La causa pi\`u comune \`e una variante di $HbS$ con la malattia di emoglobina
$SS$ prevalente negli africani. 
\subsubsection{Caratteristiche cliniche}
L'anemia falciforme comporta tosse, sudore notturno, dolori nelle gambe e nelle articolazioni, dolori addominali, poco appetito e fatica. Si mostra una correlazione tra tensione 
dell'ossigeno e la forma a falce dei globuli rossi. La forma diventa pi\`u pronunciata con bassa pressione dell'ossigeno e le grandi aggregazioni delle cellule viste nei capillari 
e negli organi riflettono una bassa tensione dell'ossigeno che porta alla morte. I pazienti che esprimono $\gamma$-globina tra il $10$ e il $20\%$ del livello di globina a falce 
hanno migliorato le prognosi cliniche. La sindrome di anemia falciforme prodotta da Antille $HbS$ ha un fenotipo pi\` severo rispetto a quella prodotta da $HbS$. Gli eterozigoti umani
per $HbS$  hanno globuli rossi che contengono il $40\%$ $HbS$ ma non esibiscono sintomi clinici, mentre gli eterozigoti per Antille $HbS$ esibiscono sintomi clinici simili agli 
omozigoti. Questo in quanto $HbS$ Antille \`e meno solubile e favorisce deossigenazione e polimerizzazione di Antille $HbS$. Gli eventi sono dovuti alla polimerizzazione della d
deossiemoglobina $S$ quando i globuli rossi trasportano ossigeno. La cellule a falce possono bloccarsi nei capillari e causare dolori. Possono inoltre lisarsi e l'emoglobina legare
ossido nitrico \emph{NO} causando vasocostrizione. Possono inoltre essere fagocitate dai macrofagi causando anemia grave. Possono inoltre aderire all'endotelio causando infiammazione
associata con leucocitosi neutrofila. La formazione di $HbS$ \`e dovuta a una mutazione puntiforme di $GAG$ in $GUG$ che sostituisce un acido glutammico con una valina. In caso di
genotipo $HbA/HbA$ si ha un fenotipo normale, con $HbA/HbS$ non si manifesta a livello clinico ma si trovano alcuni globuli rossi falciformi visibili, $HbS/HbS$ presenta anemia grave
con globuli rossi a falce. 
\subsubsection{Gestione clinica}
\paragraph{Terapia genica}
La terapia genica per l'anemia falciforme coinvolge una singola infusione di midollo autologo derivato dalle cellule emopoietiche staminali $HSC$ $CD34+$ trasdotte con un vettore 
lentivirale contenente un ansa di RNA con target $BCL11A$. Questa terapia abbassa l'espressione di $BCL11A$ che normalmente reprime la produzione di emoglobina fetale. 
\paragraph{CRISPR per combattere l'anemia falciforme}
I ricercatori hanno mostrato del successo nel correggere la mutazione nei topi, ma l'applicazione umana \`e lontana anni, l'efficienza del processo \`e bassa per usi pratici. 
\paragraph{Serendipity}
L'uso di idrossiurea per la prevenzione dell'avvenimento di infarti in bambini affetti da anemia falciforme pu\`o essere utilizzata nei paesi in via di sviluppo in quanto i costi di cura
sono contenuti. L'uso di idrossiurea \`e associato con eventi avversi come neutropenia se utilizzata ad alte dosi. L'utilizzo di idrossiurea causa un aumento della concentrazione 
intracellulare di emoglobina fetale $HbF$ che interferisce con la formazione del polimero di deossiemoglobina $S$ e riduce la conta neutrofila riducendo lo stato di infiammazione 
cronico.
\subparagraph{meccanismo di azione dell'idrossiurea}
L'obiettivo dell'idrossiurea \`e l'enzima ribonucletodide riduttasi e agisce come un radicale libero per i gruppi tirosile dell'enzima, essenziale per la sintesi di DNA. La sua 
inibizione causa un arresto nella fase $S$ del ciclo cellulare. L'efficacia nel trattamento dell'anemia falciforme \`e associata all'abilit\`a di aumentare i livelli di emoglobina
fetale che diminuisce la concentrazione di $HbS$ diminuendo la polimerizzazione dell'emoglobina anormale. Il meccanismo non \`e chiaro e potrebbe essere citotossica per i precursori 
eritrcoidi tardivi, un effetto che porta al reclutamento di precursori primordiali con una capacit\`a maggiore di produrre $HbF$. Un altro meccanismo \`e che potrebbe riprogrammare i 
precursori tardivi per fargli produrre $HbF$. Alternativamente potrebbe interrompere i fattori di trascrizione che si legano a regioni promotrici intorno ai geni di globina alterando
il rapporto tra $HbA$ e $HbF$. 
\subsubsection{Resistenza alla malaria}
Le cellule infettate da falciparum malaria sviluppano gobbe sulla superficie che le portano ad attaccarsi all'endotelio di capillari come quelli nel cervello. In questi siti avviene
la falcificazione a causa della bassa concentrazione di ossigeno. Perforazione della membrana dei parassiti come risultato di stressi fisico avviene con perdita di postassio. Inoltre
la cellula infetta \`e pi\`u acida, aumentando il tasso di falcificazione. I parassiti intraeritrocitici si sviluppano pi\`u lentamente in eritrociti $HbF$ che pertanto fornisce
protezione dalla plasmodium falciparum malaria ritardando la crescita del parassita. Il meccanismo coinvolge la resistenza alla digestione da emoglobinasi malariali basata sulla 
super stabilit\`a del tetramero $HbF$. 
