\chapter{Mendel}
\section{Vita}
Si noti come Mendel costruisce i suoi esperimenti senza sapere che cosa fossero i geni e che controllano i caratteri, che sono localizzati sui cromosomi e che le cellule hanno la
possibilt\`a di segregare i cromosomi attraverso meiosi. Mendel non parte da zero ma \`e un uomo del suo tempo e beneficia della cultura del tempo: anche in famiglia \`e a contatto con
una tradizione di chi lavorando sui campi tenta di migliorare la produzione incrociando le piante. Mendel nasce nel $1822$ in Repubblica Ceca, in un piccolo villaggio, centro agricolo. 
Nasce da una famiglia non molto agiata di contadini, in un ambiente dove l'ibridazione fra piante \`e pane quotidiano. Viene suggerito alla famiglia di farlo studiare nel ginnasio dove
avr\`a dei buoni risultati e quando dovr\`a spostarsi per poter accedere all'universit\`a si trova in un ambiente difficile e da questa esperienza torna e decide di entrare nel $1843$ 
in un convento di Agostiniani, studiosi che abitavano in un convento a Brno ed erano ben visti per la loro competenza. Non ha successo come insegnante. Conosceva molto la sistematica
e la nomenclatura e descrizione delle forme viventi con il sistema di Linneo e aveva idee radicali sul concetto di ereditariet\`a. Completa i suoi studi e prende i voti nel $1847$. 
L'abate gli permette di continuare gli studi facendolo spostare a Vienna all'universit\`a. Mendel matura l'interesse per dimostrare di trovare delle regole nell'analisi dei fenotipi 
di piante che vengono incrociate in modo opportuno per trovare la comparsa di varianti. Mendel inizia gli esperimenti incrociando topi per cercare di capire se poteva trovare
qualche pattern, ma viene scoraggiato dall'abate. Come seconda scelta lavora sulle piante. Ultimati questi esperimenti viene scritta ''Esperimenti nell'incrocio delle piante" dove Mendel
racconta le sue scoperte dove viene pubblicato il suo articolo che non ha una grossa diffusione. Le scoperte di Mendel rimangono nascoste per alcuni anni. Mendel per discutere della sua 
scoperta con le persone pi\`u influenti scrive a N\"ageli per intavolare una discussione e continuare la sua opera. Lo scambio di lettere \`e molto saltuario. N\"ageli dice a Mendel di 
ricreare l'esperimento con un altro modello sperimentale e suggerisce di usare un modello senza la tendenza di generare ibridi. In parte anche per questo Mendel abbandona l'attivit\`a 
sperimentale anche perch\`e diventa responsabile del monastero e avr\`a interazioni difficili con l'autorit\`a locale.
\subsection{Esperiment}
L'approccio vincente di Mendel \`e quello di aver costruito l'esperimento con pazienza scegliendo il modello sperimentale: prova prima con pi\`u piante e alla fine sceglie il pisello da 
giardino. Inizia a selezionare i caratteri da investigare per valutare cosa succede durante gli incroci. \`E un lavoro metodico che necessita di alcuni anni: scarta diverse 
caratteristiche fino ad arrivare a $7$. Mendel seleziona il colore, forma e rivestimento del seme, colore e forma del baccello, colore e forma del baccello e altezza del fusto. La pianta
\`e comoda in quanto gli stami e le antere sono confinati in un astuccio, nonostante la pianta tende a fare autofecondazione, Mendel pu\`o tagliare le antere prima che maturi il polline 
e poi andare a fecondare manualmente in modo da realizzare l'incrocio desiderato. La prima cosa che Mendel scopre \`e che un carattere \`e dominante e l'altro recessivo: la forma del 
seme liscia viene detta dominante rispetto alla forma rugosa per esempio. I primi esperimenti si occupano dell'incrocio di piante che differiscono per una sola caratteristica. Le piante 
da cui Mendel parte devono essere linee pure: autofecondate per diverse generazioni in modo da dimostrare che tutta la progenie abbia il fenotipo scelto. Mendel introduce le lettere del 
linguaggio mendeliano ancora in utilizzo. La lettera maiuscola determina il carattere dominante. Oltre a preparare il sistema sperimentale genera esperimenti con numerologia 
significativa. Nel primo esperimento scopre pertanto che una delle due caratteristiche \`e dominante: nella generazione $F_1$ generata dall'ibridazione la popolazione torna omogenea e 
presenta una sola caratteristica. Mendel conduce gli esperimenti fino a che fosse necessario per chiarire l'ipoteso: si deve superare la prima generazione per capire dove finisce il 
fenotipo recessivo: si autofeconda $F_1$ e riappare il carattere recessivo e contando i numeri si rende conto che il rapporto tra carattere dominante e recessivo approssima il $3:1$. 
Dopo questo comincia a combinare caratteri, con incroci che coinvolgono due fenotipi e riesce di nuovo a stabilire delle regole di segregazione: nella $F_1$ rimangono visibili solo i 
fenotipi dominicani e nella $F_2$ ricompaiono i caratteri recessivi in combinazioni non presenti nella linea parentale: i caratteri segregano e sono collegati a qualcosa che si separa e 
si assortiscono e nota come l'assortimento sia indipendente. 
\subsection{L'eredit\`a Mendeliana}
La scoperta di Mendel rimane sepolta in qualche biblioteca ma viene riscoperta verso il $1900$ quando si vede la 
comparsa nella letteratura scientifica del nome di Mendel e delle sue scoperte grazie a De Vries, Tschermak e Correns che fanno esperimenti di ibridazione citando Mendel. De Vries 
inoltre deduce che nuove varianti possono essere generate dai mutanti grazie ai quali si pu\`o generare l'evoluzione. Correns che lavora a T\"ubingen, studente di N\"ageli che riprende
il suo lavoro. Si aggiungono ad essi Bateson e Punnet, il primo \`e quello che fa pi\`u degli altri traducendo l'articolo di Mendel in inglese e diffonde il suo lavoro, il 
secondo lo affianca e sostituisce creando i quadrati di Punnet che consente di prevedere la formazione di gameti e il fenotipo di una generazione. Viene scoperta anche un'eccezione. 
\section{Principi mendeliani}
\subsection{Primo esperimento}
Si parte da due caratteristiche diverse: seme rotondo e grinzoso. Si vuole studiare cosa succede incrociando i fenotipi nella progenie. Lo sperimentatore separa polline dalla parte 
femminile, lo prende dalla pianta con la caratteristica che ha scelto. Guardando il fenotipo del seme non si deve aspettare molto: si aprono i baccelli e si osserva. Non si pu\`o 
prendere una pianta a caso ma deve essere una linea pura omogenea e si pu\`o fare l'incrocio. La generazione $P$ o parentale sono le linee pure che vengono incrociate. La prima
generazione o $F_1$ presenta una progenie omogenea con semi lisci. Le piante poi vengono lasciate libere di autofecondarsi e si osserva poi $F_2$ e si nota come la caratteristica 
grinzosa non era scomparsa ma ancora presente, inoltre si deduce con una proporzione come i semi rotondi e grinzosi stanno in rapporto $3:1$. Mendel comincia a pensare che i caratteri
siano legati a fattori discreti ma comincia ad usare un simbolismo attraverso lettere: una linea pura con fenotipo dominante e in omozigosi $RR$ e per l'altra linea pura con fenotipo
rugoso $rr$. La prima ipotesi di Mendel, per noi formazione di gameti attraverso la meiosi, \`e che i caratteri discreti alla base del fenotipo si separano durante la formazione di
una nuova generazione e nella fecondazione singole lettere si incontrano: le piante $F_1$ saranno tutte eterozigoti $Rr$, la dominanza fa s\`i che tutti i semi abbiano fenotipo liscio, 
autofecondando $F_1$, la produzione attraverso la separazione di alleli. Pertanto nella fecondazione di due eterozigoti si hanno tre possibilit\`a: $RR$, $Rr$ e $rr$. Gli eventi sono
equiprobabili e sommando per fenotipo si ha un rapporto $3:1$
\subsubsection{Reintrerpretazione in tempi moderni}
Con la cellula diploide e due cromosomi omologhi, eterozigote per il gene legato al carattere forma del seme. Questa cellula pu\`o andare incontro a meiosi per formare nuovi gameti e 
per farla, si duplica il DNA, i cromosomi sono composti da due cromatidi fratelli con $4$ lettere, la cellula viene divisa meioticamte: vengono separati i due omologhi e poi i cromatidi
fratelli formando cellule aploidi con solo un cromosoma e una lettera. Partendo da un'eterozigote si formano cellule aploidi in pari numero dominanti e recessivi, vero anche se avviene
un crossing-over meiotico che scambia porzioni dei cromatidi non fratelli. 
\subsubsection{Quadrati di Punnet}
\subsubsection{Ipotesi di Mendel}
Ci sono dei fattori responsabili della trasmissione ereditaria dei caratteri e sono unit\`a discrete (geni) che compaiono in coppie, esistono in forme alternative e si separano 
(segregano) durante la formazione dei gameti. Le piante possono avere due alleli equivalenti (omozigoti) o due alleli diversi: uno dominante e uno recessivo (eterozigoti). Mendel 
conclude che facendo l'incrocio tra individui che differiscono per un solo carattere possono differire solo per una coppia di alleli e se si hanno delle linee pure $AA$ e $aa$ la 
progenie $F_1$ sar\`a obbligatoriamente eterozigote $Aa$. Facendo autofecondare $F_1$ $Aa\times Aa$ si pu\`o ottenere $F_2$ dove si ritrova il fenotipo scomparso in proporzione di 
$\frac{1}{4}$ e il resto presenta fenotipo dominante di classe omozigote $AA$ ed eterozigote $Aa$. 
\subsection{Reincrocio}
Un altro incrocio di Mendel: si ha un problema con una pianta che mostra il fenotipo dominante si pu\`o non essere sicuri di avere una linea pura omozigote. Per chiarire se la pianta
\`e omozigote od eterozigote il fenotipo non ci pu\`o aiutare, Mendel si inventa un incrocio: incrocio di controllo o reincrocio: si prende una pianta a fenotipo dominante e genotipo 
non noto e incrociarla con una pianta a fenotipo recessivo. Osservando il risultato e la proporzione della progenie si nota il fenotipo della pianta di controllo. 
\subsection{Secondo esperimento}
Il secondo esperimento \`e l'incrocio tra due coppie di caratteri. La generazione parentale presenta omozigosi di due tratti: giallo e liscio $RRYY$ e l'altra \`e verde e rugoso
$rryy$. La pianta con il fenotipo dominante e linea pura si separano le coppie di elementi discreti $RY$ e il fenotipo recessivo $ry$. Questi gameti vengono fatti reincontrare e 
$F_1$ possiede un genotipo $RrYy$ con fenotipo dominante giallo liscio. L'esperimento continua e si autofeconda $F_1$ creando $F_2$. Si formano diversi gameti: $RY$, $Ry$, $rY$, $ry$. 
Non si sa se $R$ e $Y$ si muovono in modo causale o con delle regole precise, non si sa se si formano davvero o con la stessa proporzione. Aprendo il baccello si nota
che si trovano semi gialli e lisci $R-Y-$, gialli e rugosi $rrY-$, verdi e lisci $R-yy$ e verdi e rugosi $rryy$. Si nota la comparsa di fenotipi che non si trovavano nella linea 
parentale. Si dimostra come si assortiscono in modo casuale le coppie di caratteri diversi e i numeri dicono che il processo \`e casuale: $\frac{9}{16}$ giallo e liscio, $\frac{3}{16}$
per giallo e rugoso e verde e liscio e $\frac{1}{16}$ per verde e rugoso. Si nota come i geni stanno sui cromosomi e la coppia allelica $Yy$ sta su un cromosoma e quella $Rr$ un
altro e l'assortimento indipendente \`e il movimento indipendente del movimento dei cromosomi. 
\subsubsection{Assortimento indipendente}
\`E vero che ci si deve sempre aspettare assortimento indipendente. Se i geni sono sullo stesso cromosoma ci si aspetta che siano ereditati insieme. 
\section{Comprensione molecolare degli esperimenti di Mendel}
L'identificazione dei geni responsabili per i tratti di studio di Mendel richiedono dimostrazione che:
\begin{itemize}
	\item Alleli diversi devono essere responsabili di variazione morfologica.
	\item La differenza tra alleli \`e legata a una di DNA. 
	\item I prodotti proteici tra diversi alleli hanno diversa struttura e funzione. 
	\item Le differenze funzionali tra le varianti di proteine hanno effetto sulle variazioni morfologiche.
\end{itemize}
