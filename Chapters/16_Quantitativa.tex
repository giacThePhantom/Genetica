\chapter{Genetica quantitativa - caratteri poligenici}

\section{Caratteri continui e norma di reazione}
I caratteri continui sono caratteri che dipendono da pi\`u geni e danno origine a una vasta diversit\`a di fenotipi che mostrano una distribuzione continua nella popolazione.
Si definiscono come caratteri che mostrano ampia e continua distribuzione del fenotipo.
Possono essere caratteri poligenici con moltigeni e multifattoriale.
Le code della gaussiana stesso genotipo delle linee pure.
\[\biggl(\dfrac{1}{4}\biggr)^n = probabilita genotipi omozigoti parentali\]
Dove $n$ \`e il numero di geni coinvolti, per genomi diploidi e che segregano in maniera indipendente.

	\subsection{Norma di reazione}
	La norma di reazione definisce il limite fino a cui l'ambiente influisce sulla determinazione di un tratto genico.
	L'ambiente pu\`o cambiare la varianza dei fenotipi.

\section{Ipotesi poligenica per l'ereditariet\`a continuativa}
L'ipotesi poligenica per l'ereditariet\`a continuativa molti locus controllano.
Proporzionalemnte all'incremento del numero di alleli si crea un effetto additivo sul fenotipo.
Si studia attraverso modelli con numero crescente di geni.
Gli alleli possono essere classificati funzionali o additivi o non funzionali non contribuiscono rispetto alla determinazione del carattere.
Gli allei devono scontribuire in modo uguale e che i loro contributi devono sommarsi senza epistasi.
Chicchi di grano Ehle.

\section{Descrizione della distribuzione dei fenotipi di un carattere continuo}
La distribuzione dei fenotipi pu\`o essere:
\begin{itemize}
	\item Normale simmetrica.
	\item Asimmetrica.
	\item Bimodale a due picchi.
\end{itemize}
Il valore medio della curva con frequenza massima \`e la media su $x$.
La varianza quantifica la differenza di ciascun punto dal grafico e dalla media.
Maggiore la varianza maggiore la dispersione della curva.
\[s^2=\dfrac{\sum x_i - x)^2}{n-1}=deviazione\ quadratica\ della\ media\]
\[ s = \sqrt{s^2} = deviazione\ standard\]

\section{Segregazione allelica nella produzione dei tratti quantitativi}
East geni multipli Nicotiana longiflora corolla.
Due linee pure con altezza media:
\begin{itemize}
	\item Lunghezza della corolla si basa sualla segregazione di geni multipli, i geni devono essere $5$ per la frequenza di fenotipo parentale.
	\item L'espressione fenotipica di ogni genotipo \`e influenzata da fattori ambientali giustificando la variazione osservata nelle linee pure.
\end{itemize}
Con generazioni succcessive aumenta l'ampiezza della distribuzione fenotipica o varianza e la media rimane costante.
Genotipi diversi.
Incrociando coda di $F2$ con la parentale si riduce la varianza in quando si tendono ad incrociare genotipi omogenei.

\section{Ipotesi multifattoriale di Fisher}
Descrive la curva di distribuzione noramle.
La varianza misura la dispersione.
La varianza totale \`e la combinazione additiva di tutte le componenti che contribuiscono al fenotipo continuo: geni e ambiente 
\[V_{TOT} = V_A + V_G\]
La covarianza mette in relazioni due variabili e le valuta moltiplicando gli scostamenti di due fattori indipendenti
\[cov = \dfrac{\sum(x_i -x)(y_i - y)}{n-1}\]
Dalla covarianza si arriva al coefficiente di correlazione:
\[coefficiente\ correlazione = \dfrac{cov_{xy}}{s_xs_y}\]
Misura la forza di associazione tra due variabili nella stessa unit\`a sperimentale.

	\subsection{Retta di regressione}
	Esprime la relazione  tra due variabili, definita da una retta
	\[y = a + bx\]
	Dove $y$ e $x$ sono le variabili \`e $a$ \`e l'intercetta sull'asse $y$.
	\[b = \dfrac{covarianza_{xy}}{s^2_x}\]

\section{Effetti soglia ed espressione di caratteri discontinui multifattoriali}
Non tutti i caratteri multifattoriali mostrano una distribuzione continua.
Si trova un equilibrio tra alleli negativi e positivi.
Un carattere soglia presenta una retta verticale di predisposizione o suscettibilit\`a.
Distribuita in modo continuo ma quando raggiugne un valore soglia esprime il carattere.
Ereditabilit\`a del carattere soglia concordanza tra gemelli monozigoti, affetti dal fenomeno e quante volte hanno lo stesso fenotipo.
Se i geni sono sufficienti si ha concordanza del $100\%$.

\section{Analisi dei caratteri quantitativi}
Si devono dividere le variazioni osservate del carattere nella componente genetica e ambientale.
La varianza interazione geni-ambiente $V_{GA}$
\[V_{TOT}  = V_A + V_G + V_{GA}\]
La varianza genetica pu\`o essere di dominanza o interazione genica;

	\subsection{Ereditabilit\`a di un carattere}
		
		\subsubsection{Senso lato}
		Variazione fenotipica attribuibile a fattori genetici
		\[H^2 = \dfrac{V_G}{V_{TOT}}\]

		\subsubsection{Senso stretto}
		Proporzione della varianza fenotipica dovuta agli effetti addiivi degli alleli nella popolazione.
		\[h^2 = \dfrac{V_{GA}}{V_{TOT}}\]

\section{Regressione verso la media}
Tendenza delle generazioni ad avere media che tende verso la popolazione generale.

	\subsection{Selezione artificiale}
	La media del fenotipo per i genitori selezionati o differenziale di selezione.
	Genitori distanti dalla media genotipo verso direzione desiderata.
	Sia $\mu$ la media di popolazione, $T_P$ il differenziale di selezione e $T_0$ il fenotipo del figlio.
	\[T_0 = \mu + h^2(T_P - \mu)\]
	\[(T_0 - \mu) = h^2(t_P - \mu)\]
	Dove $T_0-\mu$ \`e la risposta alla selezione $R$ o la distanza nel fenotipo tra i figli rispetto alla media della popolazione.
	La componente additiva che permette di guidare una selezione diminuisce con l'omogeneit\`a genica.

	\subsection{Drosophila}
	Numero di setole.

\section{Quantitative trait loci}

	\subsection{Ereditabilit\`a dell'obesit\`a}
	Componente genetica additiva alta.
	Modifica recettore accumulo cellule adipose bianche.
	All'aumentare dell'indice di massa corporea dei genitori si ha la tendenza ad avere figli co massa corporea maggiore.
	Un ruolo genetico \`e evidente.

	\subsection{QTL}
	I QTL sono i quantitative trait loci, regioni genomiche contenenti un elemento genico che contribuisce in modo quantitativo ad un fenotipo.
	Obesit\`a $408$.


	\subsection{Identificazione dei QTL}
	Identificazione omozigoti marcatori molecolari.
	Pomodori code della popolazione LL SS.
	F1 intermedi.
	Incrocio di controllo F2 variabile pi\`u verso LL.
	Crecse varianza.
	Correlare marcatori molecolari.
	Correlare per ogni pianta al peso del frutto analisi molecolare.
	Per ogni marcatore una media per il peso in omozigosi e l'altra perso in eterozigosi.
	Differeza delle due medie, se vicine marcatore.

	\subsection{Eterosi}
	L'eterosi \`e la dissezione dell'architettura di un QTL di un lievito.
	Correlazione con vita a temperature.
	Ibrido tra due ceppi.
	Crescita tra tuttie tre studia vitalit\`a diversa.
	L'eterozigosi vantaggio: combinazione di alleli diversi contro-arresta un effetto negativo o eterosi.
	Nessun aploide resistente come ibrido.
	Tra tre e quattro geni.
	$(\frac{1}{2})^n$, aploide.
	Marcatori biallelici, posizioni dove i due genomi hanno alleli diversi.
