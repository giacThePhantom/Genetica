\chapter{Genetica del cancro}

\section{Comprensione delle cause}

	\subsection{Caratteristiche}
	Alterazioni funzionali di numerosi geni.

	\subsection{Tipologia cause}

		\subsubsection{Virali}

			\paragraph{RSV}
			Polli.
			Umani Src.

		\subsubsection{Iperplasia cellualre}

		\subsubsection{Divisione cromosomica anormale}
		Flemming e Boveri alterazione cromosomica dovuta a esposizione di sostanze chimiche dannose.

		\subsubsection{Fattore genetico endogeno}
		Retinoblastoma.
		Inibitore del tumore RB1, perso in entrambe le copie.
		Perdita parte cromosoma $13$.

		\subsubsection{Cromosoma Philadelphia}
		Traslocazione cromosomica di due frammenti fusi in modo aberrate.

	\subsection{Oncogeni}
	Capaci di trasformare cellule in tumorali come ras e myc su topi.

\section{Hallmarks of cancer}

	\subsection{DInamico cambiamento del genoma}
	Cellule con fisiologia influenzata dal contesto locale e cominicazione.

	\subsection{Nascita dei tumori}
	Proliferazione, procurarsi sostane nutritive, ignorare segnali inibitori della crescita.
	Abolire funzioni autonome di apoptosi.
	Cellule specializzate forniscono nutrimento e crescono verso la stessa massa per angiogenisi.
	Sono capaci di lasciare il luogo di origine.

	\subsection{Autonomia nei segnali di crescita}
	Il tumore pu\`o auto prodursi il segnale attraverso stimolazione autocrina, o rendendo il recettore iperattivo.
	Non necessiti del segnale di crescita \emph{RTK} fosforilate.
	Caderine e integrine modificate rendendo tumore indipendente.

	\subsection{Insensibilit\`a ai segnali inibitori della crescita}
	Stato di quiescenza e stato post-mitotico permanente prevengono la crescita tumorale.
	I segnali antiproliferativi vengono incanalati verso \emph{pRb}.
	Questi ipofosforilati bloccano la proliferazione sequestreando la funzione dei fattori di trascrizione E2F per i geni della progressione nella fase S.
	Tumore rompe il pathway pRb liberando E2F, permettendo proliferazione.
	TGF$\beta$ smads, inibitori Rb.
	HPV E7 inibisce Rb.

	\subsection{Evasione della morte cellulare programmata}
	Death factor attivano Caspasi, bilanciati con fattori di sopravvivenza, tumore sconvolge equilibrio.

	\subsection{Immortalit\`a}
	Erosione della porzione terminale dei cromosomi.
	Limite di Hayflick.
	Celule tumorali attivano telomerasi allungando telomeri.

	\subsection{Angiogenesi}
	Nuova angiogenesi per irrorare tumore primario.
	Ipossia p segnale iniziatore di espressione genica che gemma capillari a avvicinarsi ai tumori attraverso \emph{VEGF}.
	Spinge le cellule endoteliali a prliverare in direzione di un gradiente.
	Fattore sui bersaglio aiutandole a crescere verso il tumore.

	\subsection{Invasione tissutale e metastasi}
	Metastasi: tumore lascia la sede originaria e invade e attraversa una matrice di collagene e proteina, raggiunge via di trasporto, sopravvive e trova un terreno fertile.
	Produzione di proteasi nella matrice per renderla pi\`u permissiva.

	\subsection{Evasione del sistema immunitario}

	\subsection{Infiammazione}
	Cronica e di basso livello alterando microambiente.

	\subsection{Ecosistema}
	DI comunicazione.

	\subsection{Plasticit\`a epigenetica}
	Variazione di informazioni, alternative con regolazione per ambiente permissivo.

