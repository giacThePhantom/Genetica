\chapter{Il linkage, la ricombinazione e mappatura del gene eucariote}

\section{Citogenetica - Bandeggio e mappatura fisica dei cromosomi}

	\subsection{Tipi di bandeggio}
	\begin{multicols}{2}
		\begin{itemize}
			\item Bandeggio $G$: i cromosomi sono soggetti a digestione controllata con trispsina prima di essere colorati con \emph{Giemsa}, un colorante legante il DNA.
				Le bande colorate sono bande $G$, quelle non colorate $G$ negative.
			\item Bandeggio $Q$: i cromosomi sono colorati con un colorante fluorescente che si lega preferenzialmente a DNA ricco in $AT$ come \emph{Quinacrina}, \emph{DAPI} o \emph{Hechst 33258} e osservati in fluorescenza.
				Le bande fluorescenti sono dette bande $Q$ e marcano gli stessi segmenti delle bande $G$.
			\item Bandeggio $R$: \`e l'inverso del bandeggio $G$.
				I cromosomi sono denaturati con calore in soluzione salina prima di essere colorati con \emph{Giemsa}.
				Le bande $R$ e $Q$ sono negative.
				Lo stesso pattern pu\`o essere riprodotto da coloranti specifici a $GC$.
			\item Bandeggio $T$: identifica un sottoinsieme delle bande $R$ concentrate ai telomeri.
				Sono pi\`u intensamente colorate e visualizzati con un trattamento a calore del cromosoma prima della colorazione.
			\item Bandeggio $C$: mostra eterocromatina costitutiva ai centromeri.
				I cromosomi sono esposti a denaturazione con una soluzione saturata di idrossido di bario prima della colorazione con \emph{Giemsa}.
		\end{itemize}
	\end{multicols}

	\subsection{Categorizzazione dei cromosomi}
	I cromosomi possono essere divisi in categorie in base alla posizione relativa del centromero:
	\begin{multicols}{2}
		\begin{itemize}
			\item Metacentrico: il centromero si trova a met\`a delle due braccia.
			\item Submetacentrico: il centromero si trova leggermente spostato rispetto al centro.
			\item Acrocentrico: il centromero si trova vicino ai telomeri.
			\item Telocentrico: il centromero si trova sui telomeri.
		\end{itemize}
	\end{multicols}

	\subsection{Strutture dei cromosomi}
	Il cromosoma pu\`o essere diviso in braccia corte $p$ e lunghe $q$, divise in bande e sottobande.
	\[\overbrace{<p,q>}^{\text{braccio}}\overbrace{<num>}^{\text{banda}}\overbrace{.<num>}^{\text{sottobanda}}\]

		\subsubsection{Colorazione \emph{Giemsa} - le bande $\mathbf{G}$}
		Le bande $G$ sono utili per identificare cromosomi.
		Le bande si trovano infatti a luoghi specifici e micrografi a scansione di elettroni mostrano costrizioni a siti dove appaiono le bande o inserti.
		Regioni di lunghezza variabile sono comuni nella regione dei centromeri.
		Per convenzione $p$ denota il braccio corto, mentre $q$ il lungo.
		Ogni braccio \`e diviso in sezioni principali e sottosezioni.
		Si identifica una particolare sottosezione precedendola con il numero della propria sezione.

	\subsection{Cromosomi politenici di Drosophila}
	Nelle ghiandole salivari di Drosophila sono presenti moltissime bande e sono detti pertanto politenici.
	Si intende per politenici cromosomi che presentano moltissimi filamenti.
	I centromeri dei quattro cromosomi presenti appaiono fusi al centromero.
		
		\subsubsection{Formazione di un cromosoma politenico}
		Nel cromosoma politenico delle ghiandole salivari di Drosophila il bandeggio \`e la conseguenza del preciso allineamento di DNA e proteine amplificato da successive replicazioni difettive di centromero e telomero.
		In questi cromosomi ciascun cromosoma parentale si replica $10$ volte e si notano $1024$ filamenti appaiati in modo ordinato.

	\subsection{Ibridazione in situ con sonde fluorescenti}
	L'ibridazione in situ con sonde fluorescenti o \emph{FISH} \`e una tecnica che permette di visualizzare specifiche sequenze di DNA.
	Un cromosoma metafasico viene denaturato e avviene un anneal con una sonda fluorescenza.
	Questa colorazione permette di rivelare la presenza di sequenza specifiche all'interno del cromosoma.

		\subsubsection{Cromosoma Philadelphia}
		Nowell e Hungerford studiano la leucemia mieloide cronica attraverso il cariotipo di cellule tumorali.
		Notano pertanto la presenza di cromosomi alternativi associati alla patologia detto cromosoma Philadelphia \emph{Ph}.
		In seguito si nota come questo sia il risultato della fusione di due cromosomi tramite una traslocazione bilanciata, in cui due cromosomi scambiano porzioni specifiche di DNA.
		Questo processo viene osservato attraverso \emph{FISH} a due colorazione.
		Le due sonde pertanto sono:
		\begin{multicols}{2}
			\begin{itemize}
				\item Rossa: gene del cromosoma $9$.
				\item Verde: gene del cromosoma $22$.
			\end{itemize}
		\end{multicols}
		La ricostruzione del cariotipo delle cellule tumorali evidenzia lo scambio di un frammento fra due cromosomi: una porzione subtelomerica del cromosoma $9$ si scambia con una porzione del piccolo cromosoma $22$ in modo reciproco.

			\paragraph{Processo molecolare}
			Lo scambio reciproco causa la fusione del gene \emph{BCR} verde sul cromosoma $22$ e del gene \emph{ABL} rosso sul cromosoma $9$.
			\emph{ABL} \`e una proteina chinasi che fosforila tirosine e con funzione oncogenica in quanto sostiene la proliferazione cellulare, mentre \emph{BCR} \`e un gene attivo nei linfociti.
			Fondendo questi geni si sovra-esprime grazie al promotore \emph{BCR} una proteina oncogenica.

	\subsection{Mappatura per delezione}
	La mappatura per delezione pu\`o essere utilizzata per determinare la localizzazione cromosomica di un gene.
	
		\subsubsection{Generazione parentale}
		Si prenda in considerazione un individuo omozigote per un allele mutante recessivo \emph{aa} e un individuo \emph{A$+$} che presenta una delezione parziale del cromosoma interessato.
		Questo \`e pertanto un emizigote funzionale per la funzione associata alla porzione di interesse deleta.
	
		\subsubsection{Prima generazione}
		Un incrocio tra il mutante omozigote recessivo e l'eterozigote per delezione si osservano:
		\begin{multicols}{2}
			\begin{itemize}
				\item Eterozigoti con la porzione di cromosoma con l'allele dominante \emph{A$+$} e $a$, con il fenotipo selvatico \emph{A$+$a}.
				\item Eterozigoti in cui il cromosoma con l'allele recessivo \`e in coppia con la delezione parziale del braccio senza allele dominante con fenotipo recessivo \emph{a$-$}.
			\end{itemize}
		\end{multicols}

		\subsubsection{Risultati}
		La prima generazione indica come l'allele recessivo risiede fisicamente nella regione mancante del cromosoma e si pu\`o dedurre come l'allele si trovi nella porzione subtelomerica di questo cromosoma.
	
		\subsubsection{Ibridazione}
		L'ibridazione \`e un approccio sperimentale con cui si generano ibridi somatici in modo da avere linee cellulari con numeri diversi di cromosomi umani per studiare la residenza in essi di geni.
		Incrociando un fibroblasto umano e una cellula tumorale murina.

			\paragraph{Generare ibridi}
			Il terreno presenta il polietilen glicole che facilita la fusione tra le cellule, che passando per un intermedio a eterocarion genera una cellula ibrida a singolo nucleo.
			Clonando le cellule si nota come hanno cromosomi pi\`u piccoli con un ciclo mitotico tarato su dimensioni inferiori.
			Questo produce una perdita selettiva di cromosomi umani in quanto fanno pi\`u fatica ad essere segregati correttamente.

			\paragraph{Utilizzo degli ibridi}
			Si possono cos\`i ottenere $n$ cloni con pochi cromosomi umani.
			In questo modo si pu\`o operare sui cloni indagando la presenza di un gene attraverso test enzimatici.

			\paragraph{Selezione degli ibridi}
			Per selezionare gli ibridi viene utilizzato un terreno specifico \emph{HAT} che contiene un inibitore della diidro-folato-reduttasi \emph{aminopterina} che blocca la generazione di tetraidrofosfato essenziale per una produzione de novo dei nucleotidi.
			Le cellule con aminopterina non riescono pertanto a crescere.
			Fornendo \emph{ipoxantina} e \emph{timidina}, via alternativa per la sintesi di nucleotidi si salvano le cellule.
			Questa via alternativa richiede per\`o dei geni specifici.
			Si rendono pertanto le cellule umane incapaci di metabolizzare l'ipoxantina \emph{$HPRT^-$} e le cellule murine incapaci di metabolizzare timidina \emph{$TK^-$}.
			In \emph{HAT} crescono pertanto unicamente le cellule ibride che complimentano \emph{$TH^+$} e \emph{$HPRT^+$}.

			\paragraph{Fenilchetonuria}
			L'ibridazione tra cellule umane e di topo ha permesso di osservare come il gene della fenilanalina idrossilasi si trova sul cromosoma $12$ dell'uomo.
			In quanto si ipotizza come la fenilchetonuria classica si provocata da mutazioni strutturali del gene si nota come il locus della malattia nell'uomo si trovi in tale cromosoma.

			\paragraph{Posizione di un gene all'interno di un cromosoma}
			L'ibridazione permette di creare una perdita causale di cromosomi o di loro parti.
			Si nota pertanto come se il prodotto genico \`e presente in una linea cellulare con un cromosoma intatto ma assente da una linea cellulare che mostra una delezione cromosomica il gene per tale prodotto genico deve essere localizzato nella regione deleta.

\section{Mappatura genetica e geni associati}

	\subsection{Panoramica}
	La teoria cromosomica dell'ereditariet\`a richiede una riformulazione dei principi di Mendel.

		\subsubsection{Principio della segregazione}
		Il principio della segregazione afferma che un organismo diploide possiede due alleli per un determinato carattere ciascuno dei quali collocato nello stesso locus su entrambi i cromosomi.
		Questi segregano nella meiosi alla fine della quale ogni gamete riceve un omologo.

		\subsubsection{Principio dell'assortimento indipendente}
		Il principio dell'assortimento indipendente afferma che durante la meiosi ogni coppia di cromosomi omologhi si assortisce in maniera indipendente rispetto alle altre coppie omologhe.

		\subsubsection{Geni associati}
		In quanto nella maggior parte degli organismi il numero di cromosomi \`e limitato alcuni geni devono essere collocati sullo stesso cromosoma e non dovrebbero assortire in modo indipendente.
		I geni localizzati vicini su un cromosoma vengono detti geni associati e fanno parte le gruppo di linkage o associazione.
		Questi si muovono insieme durante la meiosi e giungono alla stessa destinazione.

			\paragraph{Fiori dei piselli odorosi}
			Un incrocio in stile mendeliano svolto da Punnet e Bateson prende in considerazione una generazione parentale $P$:
			\begin{multicols}{2}
				\begin{itemize}
					\item Fiore viola con polline allungato dominante.
					\item Fiori rossi con polline rotondo recessivo.
				\end{itemize}
			\end{multicols}
			In $F_1$ si nota una scomparsa dei geni recessivi con un fiore viola con polline allungato in accordo con le leggi di Mendel.
			In $F_2$ ricompaiono i fenotipi parentali dominante e recessivo e la combinazione dei due.
			\begin{table}[H]
				\centering
				\begin{tabular}{|c|c|c|c|c|}
					\hline
			 		& & observed & expected & expected ratio\\
					\hline
					\makecell{Viola \\ allungato} & $P\_L\_$ & $284$ & $215$ & $9$ \\
					\hline
					\makecell{Viola \\ rotondo} & $P\_ll$ & $21$ & $71$ & $3$ \\
					\hline
					\makecell{Rosso \\ allungato} & $ppL\_$ & $21$ & $71$ & $3$ \\
					\hline
					\makecell{Rosso \\ rotondo} & $ppll$ & $55$ & $24$ & $1$ \\
					\hline
				\end{tabular}
			\end{table}
			Si nota come il test del $\chi^2$ dimostra come questi non sono una deviazione accettabile dei risultati attesi.
			Pertanto le classi dell'omozigosi recessiva e dominante sono sovra-rappresentate.
		
				\subparagraph{Ipotesi}
				L'ipotesi avanzata visti questi risultati \`e che i due geni potrebbero trovarsi sullo stesso cromosoma.
				La prossimit\`a fisica fungerebbe da barriera all'assortimento indipendente in quanto vincola i due geni.
		
	\subsection{I geni associati segregano insieme e il crossing-over produce ricombinazione fra loro}
	I geni che si trovano vicini sullo stesso cromosoma normalmente segregano insieme e sono trasmessi insieme.
	Possono comunque passare da un cromosoma a un omologo tramite il processo di crossing-over che determina la ricombinazione rompendo l'associazione tra geni.
	Il linkage e il crossing-over si possono pertanto considerare come protessi con effetti opposti.

		\subsubsection{Notazione}
		La notazione per gli incroci con linkage deve tenere in considerazione:
		\begin{multicols}{2}
			\begin{itemize}
				\item L'ordine con cui gli alleli associati sono disposti sul cromosoma.
				\item Gli alleli si trovano sulla stessa colonna.
			\end{itemize}
		\end{multicols}
		\[\dfrac{A\quad\quad B}{a\quad\quad b}\]

		\subsubsection{Confronto tra linkage completo ed assortimento indipendente}
		Si prendano in considerazione geni in linkage completo, ovvero geni cos\`i vicini che non sono soggetti a crossing-over.
		Questa \`e una situazione ideale e modello.
		
			\paragraph{Testcross}
			Un testcross rende evidenti gli effetti del linkage: incrociando
			\begin{multicols}{2}
				\begin{itemize}
					\item $\frac{A\quad\quad B}{a\quad\quad b}$.
					\item $\frac{a\quad\quad b}{a\quad\quad b}$.
				\end{itemize}
			\end{multicols}
			Si ottiene una progenie:
			\begin{multicols}{2}
				\begin{itemize}
					\item $\frac{A\quad\quad B}{a\quad\quad b}$.
					\item $\frac{a\quad\quad b}{a\quad\quad b}$.
				\end{itemize}
			\end{multicols}
			In cui si manifestano unicamente gli alleli presenti nel genitore doppio eterozigote

			\paragraph{Tipologie di gameti e progenie}

				\subparagraph{Gameti non ricombinanti}
				I gameti non ricombinanti o gameti parentali sono gameti che contengono unicamente le combinazioni originali degli alleli presenti nei genitori.

				\subparagraph{Progenie non ricombinante}
				La progenie non ricombinante o progenie parentale nasce grazie all'acquisizione dei gameti non ricombinanti.

				\subparagraph{Gameti ricombinanti}
				Si definiscono gameti ricombinanti gameti in cui si notano nuove combinazioni di alleli nate dopo un evento di rottura del linkage attraverso crossing over.
				Presentano pertanto combinazioni fenotipiche assenti nella generazione parentale.
				
				\subparagraph{Progenie ricombinante}
				La progenie ricombinante nasce grazie all'acquisizione di gameti ricombinanti.

		\subsubsection{Crossing-over con geni associati}
		Si verifica un certo numero di crossing-over tra i geni posti sullo stesso cromosoma che producono nuovi combinazioni di caratteri.
		Questi geni sono associati in maniera incompleta.

			\paragraph{Descrizione}
			Il crossing-over durante la profase $I$ consiste nello scambio di materiale genetico fra cromatidi non fratelli.
			Dopo un singolo crossing-over i due cromatidi non coinvolti rimangono inalterati e presentano gameti non ricombinanti.
			I cromatidi coinvolti invece contengono nuove combinazioni di alleli e contengono gameti ricombinanti.
			Quando in una meiosi si verifica un crossing-over fra due loci si ottiene un risultato analogo all'assortimento indipendente.

				\subparagraph{Geni strettamente associati}
				Per i geni strettamente associati il crossing-over non si verifica in ogni meiosi.
				Nelle meiosi in cui non avviene si producono solo gameti non ricombinanti, mentre quando avviene met\`a saranno ricombinanti.
				La percentuale totale di gameti ricombinanti \`e pari alla met\`a della percentuale delle meiosi in cui avviene crossing over.

				\subparagraph{Risultati}
				Si nota pertanto come la frequenza dei gameti ricombinanti \`e la met\`a della frequenza dei crossing-over e la quantit\`a massima di gameti ricombinanti \`e il $50\%$.

			\paragraph{Individuare geni associati}
			Quando avviene crossing-over tra geni associati il risultato \`e una progenie in prevalenza non ricombinante con una frazione relativamente piccola di ricombinanti.
			Si nota pertanto come i rapporti numerici evidenziano il linkage tra i due geni e un grado di crossing-over.

		\subsubsection{Calcolo della frequenza di ricombinazione}
		Si indendi per frequenza di ricombinazione la percentuale di progenie ricombinante prodotta.
		\[\dfrac{numero\ progenie\ ricombinante}{numero\ progenie\ totale}\cdot 100\]

		\subsubsection{Configurazione di geni associati}
		La configurazione particolare dei geni associati determina i fenotipi pi\`u frequenti nella progenie del reincrocio.

			\paragraph{Configurazione in accoppiamento}
			La configurazione in accoppiamento o \emph{cis} avviene quando gli alleli selvatici dei geni associati si trovano su un cromosoma e gli alleli mutanti sull'altro.

			\paragraph{Configurazione in repulsione}
			La configurazione in repulsione o \emph{trans} avviene quando i cromosomi presentano un allele selvatico e uno mutante dei geni associati.

				
		\subsection{Predire l'esito degli incroci nei geni associati}
		Per predire l'esito degli incroci nei geni associati \`e fondamentale conoscere:
		\begin{multicols}{2}
			\begin{itemize}
				\item La disposizione degli alleli sul cromosoma in modo da determinare le classi di progenie.
				\item La frequenza di ricombinazione per determinare le frequenze relative delle classi.
			\end{itemize}
		\end{multicols}

			\subsubsection{Cetrioli}
			Nei cetrioli il frutto liscio $t$ \`e recessivo rispetto al bitorzoluto $T$ e quello lucido $d$ recessivo rispetto all'opaco $D$.
			\`E stato determinato che questi geni mostrano una frequenza di ricombinazione del $16\%$.
			Si incroci una pianta omozigote bitorzoluta opaca con una liscia e lucida e di reincrociare $F_1$ con il doppio omozigote recessivo:
			\[\dfrac{T\quad\quad D}{t\quad\quad d}\times\dfrac{t\quad\quad d}{t\quad\quad d}\]

				\paragraph{Gameti}

					\subparagraph{Non ricombinanti}\mbox{}\\
					\begin{multicols}{2}
						\begin{itemize}
							\item $\underline{T\quad\quad D}$.
							\item $\underline{t\quad\quad d}$.
						\end{itemize}
					\end{multicols}
					Tutti i gameti non ricombinanti sono $1-0.16=0.84$ pertanto la probabilit\`a di avere un gamete non ricombinante sar\`a $\frac{0.84}{2}=0.42$
					
					\subparagraph{Ricombinanti}\mbox{}\\
					\begin{multicols}{2}
						\begin{itemize}
							\item $\underline{T\quad\quad d}$.
							\item $\underline{t\quad\quad D}$.
						\end{itemize}
					\end{multicols}
					Questi saranno il $16\%$ in totale, pertanto la probabilit\`a di ottenere un gamete ricombinante sar\`a $\frac{0.16}{2}=0.08$.
		
					\subparagraph{Gameti dell'omozigote recessivo}
					Dall'omozigote recessivo si ottiene con probabilit\`a $1$ $\underline{t\quad\quad d}$.

				\paragraph{Risultati}
				\begin{multicols}{2}
					\begin{itemize}
						\item $\frac{T\qquad D}{t\qquad d}$ con probabilit\`a $0.42\cdot 1 = 0.42$.
						\item $\frac{t\qquad d}{t\qquad d}$ con probabilit\`a $0.42\cdot 1 = 0.42$.
						\item $\frac{T\qquad d}{t\qquad d}$ con probabilit\`a $0.08\cdot 1 = 0.08$.
						\item $\frac{t\qquad D}{t\qquad d}$ con probabilit\`a $0.08\cdot 1 = 0.08$.
					\end{itemize}
				\end{multicols}

	\subsection{Test dell'assortimento indipendente}
	Per determinare se due geni sono associati si deve distinguere tra gli scarti delle attese dovuti al caso o ad altri fattori.
	Il problema pu\`o essere risolto attraverso il testo del chi-quadro dell'indipendenza, che porta a dimostrare se l'ereditariet\`a degli alleli posti su un locus \`e indipendente dall'ereditariet\`a degli alleli posti su un secondo locus.
	Una possibile verifica \`e fornita dal calcolo delle probabilit\`a attese per ogni classe di progenie.
	Si possono trovare scostamenti dalle probabilit\`a se i geni sono associati in quanto l'eredit\`a dei genotipi non \`e indipendente o nel caso in cui la probabilit\`a di ogni genotipo a un locus non \`e $\frac{1}{2}$ in caso di minori probabilit\`a di sopravvivenza o penetranza incompleta.

		\subsubsection{Test del $\mathbf{\chi^2}$ dell'indipendenza}
		Il test del $\chi^2$ dell'indipendenza consente di valutare se la segregazione degli alleli \`e indipendente rispetto alla segregazione degli alleli in un altro.
		Per farlo si costruisca una tabella con i valori osservati delle frequenze delle quattro classi di progenie.
		Si calcoli i totali delle righe e delle colonne e il totale generale.
		Questi servono a calcolare i valori attesi per il test.
		Successivamente si calcolano i valori attesi per ogni combinazione di genotipi assumendo che la segregazione dei due alleli sia indipendente:
		\[valore\ atteso = \dfrac{totale\ di\ riga\cdot totale\ di\ colonna}{totale\ generale}\]
		Si deve calcolare il valore del $\chi^2$
		\[\chi^2 = \sum\dfrac{(valori\ osservati - valori\ attesi)^2}{valori\ attesi}\]
		Si consideri i gradi di liber\`a:
		\[gl = (numero\ righe - 1)\cdot(numero\ colonne - 1)\]
		Si utilizza ora la tabella per trovare la probabilit\`a associata ad esso.
		Un risultato inferiore a $0.05$ si scosta significativamente dai valori attesi e rende evidente che i geni non si sono assortiti in modo indipendente.

			\paragraph{Blatte germaniche}
			Nelle blatte germaniche il corpo giallo $y$ \`e recessivo rispetto al marrone $y'$ e le ali curve $cv$ sono recessive rispetto alle dritte $cv'$/
			Si esegue un reincrocio $y'ycv'cv\times yycvcv$ e si ottiene:
			\begin{multicols}{2}
				\begin{itemize}
					\item $y'ycv'cv$ $63$ con corpo marrone e ali dritte.
					\item $y'ycvcv$ $28$ con corpo marrone e ali curve.
					\item $yycv'cv$ $33$ con corpo giallo e ali dritte.
					\item $yycvcv$ $77$ con corpo giallo e ali curve.
				\end{itemize}
			\end{multicols}
			\begin{multicols}{2}
				\begin{table}[H]
					\centering
					\begin{tabular}{|c|c|c|c|}
						\hline
					 		& $y'y$ & $yy$ & totale righe\\
						\hline
						$cv'cv$ & $63$ & $33$ & $96$\\
						\hline
						$cvcv$  & $28$ & $77$ & $105$\\
						\hline
						totale colonne & $91$ & $110$ & $201$\\
						\hline
					\end{tabular}
					\caption{Valori osservati}
				\end{table}
	
				\begin{table}[H]
					\centering
					\begin{tabular}{|c|c|c|}
						\hline
						Genotipo & Valori osservati & Valori attesi\\
						\hline
						$y'ycv'cv$ & $63$ & $\frac{96\cdot 91}{201} = 43.46$\\
						\hline
						$y'ycvcv$ & $28$ & $\frac{105\cdot 91}{201} = 47.54$\\
						\hline
						$yycv'cv$ & $33$ & $\frac{96\cdot 110}{201} = 52.46$\\
						\hline
						$yycvcv$ & $77$ & $\frac{105\cdot 110}{201} = 57.46$\\
						\hline
					\end{tabular}
					\caption{Valori attesi}
				\end{table}
			\end{multicols}
			Si calcola ora $\chi^2$.
			\[\chi^2 = \dfrac{(63-43.46)^2}{43.46} + \dfrac{(28-47.54)^2}{47.54} + \dfrac{(33-52.46)^2}{52.46} + \dfrac{(77-57.46)^2}{57.46} = 30.68\]
			Ora i gradi di libert\`a:
			\[gl = (2 - 1)\cdot(4 - 1) = 3\]
			Confrontando la tabella si ottiene $p<0.005$.

	\subsection{Mappatura genetica basata sulla frequenza di ricombinazione}
	Essendo che i crossing-over avvengono in modo casuale su un cromosoma, pertanto due geni lontani sono pi\`u facilmente soggetti a crossing-over di due geni vicini.
	Le frequenze di ricombinazione possono pertanto determinare l'ordine dei geni.
	Questo permette la creazione di mappe genetiche.
	La distanza \`e misurata in unit\`a di mappa $um$, equivalente a $1\%$ di ricombinazione.
	Le distanze sono approssimativamente additive.
		
		\subsubsection{Limitazioni}
		Questa misurazione:
		\begin{multicols}{2}
			\begin{itemize}
				\item Non si \`e in grado di distinguere tra geni posti si cromosomi diversi e localizzati molto lontano nello stesso cromosoma.
				\item Un reincrocio tende a sottostimare la reale distanza fisica in quanto non rileva possibili doppi crossing-over nello spazio intermedio.
			\end{itemize}
		\end{multicols}

		\subsubsection{Costruzione di mappe genetiche con reincroci a due punti}
		Si possono costruire mappe genetiche effettuando una serie di reincroci.
		In ognuno di questi un genitore \`e eterozigote per idverse coppie di geni e le frequenze di ricombinazione vengono calcolate fra coppie di geni.
		Questo viene detto reincrocio a due punti.

	\subsection{Incroci a tre punti per mappare geni associati}
	Il reincrocio a tre punti permette di considerare simultaneamente $3$ geni.
	Permette di identificare tre geni e mostra gli effetti dei doppi crossing over.

		\subsubsection{Eventi di crossing over}
		Si consideri i cromosomi:
		\begin{multicols}{2}
			\begin{itemize}
				\item $\underline{A\qquad B\qquad C}$.
				\item $\underline{a\qquad b\qquad c}$.
			\end{itemize}
		\end{multicols}
		\begin{multicols}{2}
			\begin{itemize}
				\item Un singolo crossing over tra $A$ e $B$ produce: $\underline{A\qquad b\qquad c}$ e $\underline{a\qquad B\qquad C}$.
				\item UN singolo crossing over tra $B$ e $C$ produce: $\underline{A\qquad B\qquad c}$ e $\underline{a\qquad c\qquad C}$.
				\item Un doppio crossing over produce: $\underline{A\qquad b\qquad C}$ e $\underline{a\qquad B\qquad c}$.
			\end{itemize}
		\end{multicols}
		Si nota pertanto come il doppio crossing over causa cambiamenti nell'allele centrale fornendo un indizio riguardo l'ordine dei geni.

		\subsubsection{Costruzione di mappe genetiche con incroci a tre punti}
		Per costruire mappe genetiche con incroci a tre punti si prende in considerazione un reincrocio tra:
		\begin{multicols}{2}
			\begin{itemize}
				\item $\dfrac{A\qquad B\qquad C}{a\qquad b\qquad c}$.
				\item $\dfrac{a\qquad b\qquad c}{a\qquad b\qquad c}$.
			\end{itemize}
		\end{multicols}

			\paragraph{Determinazione dell'ordine dei geni}
			Per determinare l'ordine dei geni si identifica per primo il locus intermedio.
			Si esamina in primo luogo le categorie non ricombinanti: queste sono le pi\`u numerose.
			Successivamente si individua la progenie del doppio crossing-over.
			Questa sar\`a quella con i due fenotipi meno numerosi.
			Per stabilire il gene intermedio si disegnano i cromosomi del genitore eterozigote con le tre disposizioni possibili e si verifica se un doppio crossing-over produce la combinazione di geni osservata nella progenie.
			Si identifica successivamente confrontando con le non ricombinanti l'unico allele ricombinato che si trover\`a in mezzo.

			\paragraph{Determinazione della localizzazione dei crossing-over}
			Dopo aver determinato l'ordine dei loci su un cromosoma si deve riscrivere gli alleli nell'ordine giusto per determinare dove hanno avuto luogo i crossing over.

			\paragraph{Calcolo delle frequenze di ricombinazione}
			Per calcolare le frequenze di ricombinazione si calcola sommando la progenie ricombinante e dividendo per la progenie totale.
			Questo risultato moltiplicato per $100$ d\`a le unit\`a di mappa.

			\paragraph{Interferenza e coefficiente di coincidenza}
			Si indendi per interferenza il grado in cui un crossing-over tende a interferire con altri eventi analoghi.
			Per calcolarla si deve determinare il coefficiente di coincidenza, il rapporto tra i doppi crossing-over osservati e attesi:
			\[coefficiente\ di\ coincidenza = \dfrac{numero\ doppi\ crossing\ over\ osservati}{numero\ doppi\ crossing\ over\ attesi}\]
			L'interferenza invece:
			\[interferenza = 1 - coefficiente\ di\ coincidenza\]
			Questo determina la percentuale di progenie attesa derivante da doppi crossing-over non osservata a causa di interferenza.
