\chapter{Il linkage, la ricombinazione e mappatura del gene eucariote}

\section{Citogenetica - Bandeggio e mappatura fisica dei cromosomi}

	\subsection{Tipi di bandeggio}
	\begin{multicols}{2}
		\begin{itemize}
			\item Bandeggio $G$: i cromosomi sono soggetti a digestione controllata con trispsina prima di essere colorati con \emph{Giemsa}, un colorante legante il DNA.
				Le bande colorate sono bande $G$, quelle non colorate $G$ negative.
			\item Bandeggio $Q$: i cromosomi sono colorati con un colorante fluorescente che si lega preferenzialmente a DNA ricco in $AT$ come \emph{Quinacrina}, \emph{DAPI} o \emph{Hechst 33258} e osservati in fluorescenza.
				Le bande fluorescenti sono dette bande $Q$ e marcano gli stessi segmenti delle bande $G$.
			\item Bandeggio $R$: \`e l'inverso del bandeggio $G$.
				I cromosomi sono denaturati con calore in soluzione salina prima di essere colorati con \emph{Giemsa}.
				Le bande $R$ e $Q$ sono negative.
				Lo stesso pattern pu\`o essere riprodotto da coloranti specifici a $GC$.
			\item Bandeggio $T$: identifica un sottoinsieme delle bande $R$ concentrate ai telomeri.
				Sono pi\`u intensamente colorate e visualizzati con un trattamento a calore del cromosoma prima della colorazione.
			\item Bandeggio $C$: mostra eterocromatina costitutiva ai centromeri.
				I cromosomi sono esposti a denaturazione con una soluzione saturata di idrossido di bario prima della colorazione con \emph{Giemsa}.
		\end{itemize}
	\end{multicols}

	\subsection{Categorizzazione dei cromosomi}
	I cromosomi possono essere divisi in categorie in base alla posizione relativa del centromero:
	\begin{multicols}{2}
		\begin{itemize}
			\item Metacentrico: il centromero si trova a met\`a delle due braccia.
			\item Submetacentrico: il centromero si trova leggermente spostato rispetto al centro.
			\item Acrocentrico: il centromero si trova vicino ai telomeri.
			\item Telocentrico: il centromero si trova sui telomeri.
		\end{itemize}
	\end{multicols}

	\subsection{Strutture dei cromosomi}
	Il cromosoma pu\`o essere diviso in braccia corte $p$ e lunghe $q$, divise in bande e sottobande.
	\[\overbrace{<p,q>}^{\text{braccio}}\overbrace{<num>}^{\text{banda}}\overbrace{.<num>}^{\text{sottobanda}}\]

		\subsubsection{Colorazione \emph{Giemsa} - le bande $\mathbf{G}$}
		Le bande $G$ sono utili per identificare cromosomi.
		Le bande si trovano infatti a luoghi specifici e micrografi a scansione di elettroni mostrano costrizioni a siti dove appaiono le bande o inserti.
		Regioni di lunghezza variabile sono comuni nella regione dei centromeri.
		Per convenzione $p$ denota il braccio corto, mentre $q$ il lungo.
		Ogni braccio \`e diviso in sezioni principali e sottosezioni.
		Si identifica una particolare sottosezione precedendola con il numero della propria sezione.

	\subsection{Cromosomi politenici di Drosophila}
	Nelle ghiandole salivari di Drosophila sono presenti moltissime bande e sono detti pertanto politenici.
	Si intende per politenici cromosomi che presentano moltissimi filamenti.
	I centromeri dei quattro cromosomi presenti appaiono fusi al centromero.
		
		\subsubsection{Formazione di un cromosoma politenico}
		Nel cromosoma politenico delle ghiandole salivari di Drosophila il bandeggio \`e la conseguenza del preciso allineamento di DNA e proteine amplificato da successive replicazioni difettive di centromero e telomero.
		In questi cromosomi ciascun cromosoma parentale si replica $10$ volte e si notano $1024$ filamenti appaiati in modo ordinato.

	\subsection{Ibridazione in situ con sonde fluorescenti}
	L'ibridazione in situ con sonde fluorescenti o \emph{FISH} \`e una tecnica che permette di visualizzare specifiche sequenze di DNA.
	Un cromosoma metafasico viene denaturato e avviene un anneal con una sonda fluorescenza.
	Questa colorazione permette di rivelare la presenza di sequenza specifiche all'interno del cromosoma.

		\subsubsection{Cromosoma Philadelphia}
		Nowell e Hungerford studiano la leucemia mieloide cronica attraverso il cariotipo di cellule tumorali.
		Notano pertanto la presenza di cromosomi alternativi associati alla patologia detto cromosoma Philadelphia \emph{Ph}.
		In seguito si nota come questo sia il risultato della fusione di due cromosomi tramite una traslocazione bilanciata, in cui due cromosomi scambiano porzioni specifiche di DNA.
		Questo processo viene osservato attraverso \emph{FISH} a due colorazione.
		Le due sonde pertanto sono:
		\begin{multicols}{2}
			\begin{itemize}
				\item Rossa: gene del cromosoma $9$.
				\item Verde: gene del cromosoma $22$.
			\end{itemize}
		\end{multicols}
		La ricostruzione del cariotipo delle cellule tumorali evidenzia lo scambio di un frammento fra due cromosomi: una porzione subtelomerica del cromosoma $9$ si scambia con una porzione del piccolo cromosoma $22$ in modo reciproco.

			\paragraph{Processo molecolare}
			Lo scambio reciproco causa la fusione del gene \emph{BCR} verde sul cromosoma $22$ e del gene \emph{ABL} rosso sul cromosoma $9$.
			\emph{ABL} \`e una proteina chinasi che fosforila tirosine e con funzione oncogenica in quanto sostiene la proliferazione cellulare, mentre \emph{BCR} \`e un gene attivo nei linfociti.
			Fondendo questi geni si sovra-esprime grazie al promotore \emph{BCR} una proteina oncogenica.

	\subsection{Mappatura per delezione}
	La mappatura per delezione pu\`o essere utilizzata per determinare la localizzazione cromosomica di un gene.
	
		\subsubsection{Generazione parentale}
		Si prenda in considerazione un individuo omozigote per un allele mutante recessivo \emph{aa} e un individuo \emph{A$+$} che presenta una delezione parziale del cromosoma interessato.
		Questo \`e pertanto un emizigote funzionale per la funzione associata alla porzione di interesse deleta.
	
		\subsubsection{Prima generazione}
		Un incrocio tra il mutante omozigote recessivo e l'eterozigote per delezione si osservano:
		\begin{multicols}{2}
			\begin{itemize}
				\item Eterozigoti con la porzione di cromosoma con l'allele dominante \emph{A$+$} e $a$, con il fenotipo selvatico \emph{A$+$a}.
				\item Eterozigoti in cui il cromosoma con l'allele recessivo \`e in coppia con la delezione parziale del braccio senza allele dominante con fenotipo recessivo \emph{a$-$}.
			\end{itemize}
		\end{multicols}

		\subsubsection{Risultati}
		La prima generazione indica come l'allele recessivo risiede fisicamente nella regione mancante del cromosoma e si pu\`o dedurre come l'allele si trovi nella porzione subtelomerica di questo cromosoma.
	
		\subsubsection{Ibridazione}
		L'ibridazione \`e un approccio sperimentale con cui si generano ibridi somatici in modo da avere linee cellulari con numeri diversi di cromosomi umani per studiare la residenza in essi di geni.
		Incrociando un fibroblasto umano e una cellula tumorale murina.

			\paragraph{Generare ibridi}
			Il terreno presenta il polietilen glicole che facilita la fusione tra le cellule, che passando per un intermedio a eterocarion genera una cellula ibrida a singolo nucleo.
			Clonando le cellule si nota come hanno cromosomi pi\`u piccoli con un ciclo mitotico tarato su dimensioni inferiori.
			Questo produce una perdita selettiva di cromosomi umani in quanto fanno pi\`u fatica ad essere segregati correttamente.

			\paragraph{Utilizzo degli ibridi}
			Si possono cos\`i ottenere $n$ cloni con pochi cromosomi umani.
			In questo modo si pu\`o operare sui cloni indagando la presenza di un gene attraverso test enzimatici.

			\paragraph{Selezione degli ibridi}
			Per selezionare gli ibridi viene utilizzato un terreno specifico \emph{HAT} che contiene un inibitore della diidro-folato-reduttasi \emph{aminopterina} che blocca la generazione di tetraidrofosfato essenziale per una produzione de novo dei nucleotidi.
			Le cellule con aminopterina non riescono pertanto a crescere.
			Fornendo \emph{ipoxantina} e \emph{timidina}, via alternativa per la sintesi di nucleotidi si salvano le cellule.
			Questa via alternativa richiede per\`o dei geni specifici.
			Si rendono pertanto le cellule umane incapaci di metabolizzare l'ipoxantina \emph{$HPRT^-$} e le cellule murine incapaci di metabolizzare timidina \emph{$TK^-$}.
			In \emph{HAT} crescono pertanto unicamente le cellule ibride che complimentano \emph{$TH^+$} e \emph{$HPRT^+$}.

			\paragraph{Fenilchetonuria}
			L'ibridazione tra cellule umane e di topo ha permesso di osservare come il gene della fenilanalina idrossilasi si trova sul cromosoma $12$ dell'uomo.
			In quanto si ipotizza come la fenilchetonuria classica si provocata da mutazioni strutturali del gene si nota come il locus della malattia nell'uomo si trovi in tale cromosoma.

			\paragraph{Posizione di un gene all'interno di un cromosoma}
			L'ibridazione permette di creare una perdita causale di cromosomi o di loro parti.
			Si nota pertanto come se il prodotto genico \`e presente in una linea cellulare con un cromosoma intatto ma assente da una linea cellulare che mostra una delezione cromosomica il gene per tale prodotto genico deve essere localizzato nella regione deleta.
