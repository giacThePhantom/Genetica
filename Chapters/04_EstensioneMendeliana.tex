\chapter{Estensione dell'analisi genetica mendeliana}
Se il genetista classico si occupava di analizzare il fenotipo degli organismi che studiava quello moderno si occupa di studiare i meccanismi allelici come la dominanza, la 
recessivit\`a e le loro interazioni come il mascheramento. La cellula verr\`a considerata come una rete globale di interazione genica. Le interazioni tra i geni vengono studiati 
attraverso somiglianze fenotipiche tra i mutanti che permettono di aggregare i geni secondo la loro funzione. Per fare questo uno studio analizza il Saccharomyces cerevisiae costruendo
$23$ milioni di doppi mutanti e identificando \numprint{550000} interazioni positive e \numprint{350000} negative. I geni vengono poi raggruppati per funzione in un metagenoma che
evidenzia come i geni essenziali sono densamente connessi. 
\section{Rapporti tra gli alleli}
L'estensione naturale dell'analisi genetica mendeliana \`e considerare che dato un certo locus non \`e detto che vi si trova un solo gene. Inoltre \`e naturale come i geni non 
lavorano in modo isolato e per capire in un fenotipo si devono considerare pi\`u geni. I geni inoltre si confrontano con l'ambiente cellulare, extracellulare ed esterno. Questi tre
fattori devono essere tutti considerati quando si studiano i geni. Gli esperimenti di Mendel usavano alleli che mostravano dominanza completa, ma esistono altre interazioni.
\subsection{Dominanza incompleta}
In eterozigosi si trova un fenotipo intermedio rispetto a quello dell'omozigote dominante e recessivo: si ha la possibilit\`a di vedere fenotipicamente l'eterozigote.  Avviene in quanto
si ha per alleli diversi una diversa quantit\`a di prodotto. Ogni allele conta nel contribuire a un tratto fenotipico, importante nella genetica multifattoriale per mappare il contributo
quantitativo di alleli sparsi nel genoma  Nella dominanza incompleta ci si aspettano gradazioni intermedie ai due estremi di omozigosi dominante e omozigosi dell'allele meno funzionale. 
\subsubsection{Esempi}
\begin{itemize}
	\item In alcuni fiori il dominante puro ha fenotipo rosso il recessivo puro bianco, mentre l'eterozigote \`e rosa.
	\item Dimensione del pomodoro.
	\item Colore del piumaggio di polli e galline.
\end{itemize}
Si noti come le differenze nei sistemi di modello indicano come probabilmente si tratti di un principio universale. 
\subsection{Codominanza}
La codominanza avviene quando due alleli mostrano contemporaneamente la loro presenza. A differenza della dominanza incompleta non si produce un fenotipo intermedio ma l'effetto di 
ogni gene \`e discreto e in codominanza sono presenti entrambe le caratteristiche intermedie.
\subsubsection{Esempi}
\begin{itemize}
	\item Lenticchie con chiazze e puntini, in eterozigosi si mostrano sia chiazze che puntini. 
	\item Gruppi sanguigni nel modello $AB0$: 
		\begin{itemize}
			\item Omozigosi recessiva $ii$: non si hanno antigeni e si producono anticorpi $A$ e $B$.
			\item Eterozigosi $I^Ai$ ed omozigosi $I^AI^A$: si possiede l'antigene $A$ e si producono anticorpi $B$.
			\item Eterozigosi $I^Bi$ ed omozigosi $I^BI^B$: si possiede l'antigene $B$ e si producono anticorpi $A$.
			\item Eterozigosi $I^BI^A$: si possiedono gli antigeni $A$ e $B$ e non si producono anticorpi.
		\end{itemize}
	\item Sempre nel modello $MN$ per il sangue succeda la stessa cosa. 
\end{itemize}
\subsection{Allelismo multiplo}
Nell'allelismo multiplo i geni esistono in pi\`u di due forme alleliche e ogni individuo pu\`o possedere solo due alleli diversi di uno stesso gene. In questo caso si distinguono
tre categorie tra le quali esiste un ordine di dominanza:
\begin{itemize}
	\item Alleli amorfici o recessivi non funzionanti o nulli.
	\item Alleli parzialmente funzionanti o ipomorfici recessivi rispetto a quelli pi\`u funzionali.
	\item Alleli funzionanti o dominanti rispetto agli altri.
\end{itemize}
\subsubsection{Esempi}
\paragraph{Conigli}
Considerando i conigli come organismi modello e incrociandoli si nota la distribuzione di fenotipi e pertanto le caratteristiche degli alleli. In un certo cromosoma si trova un locus
con un'informazione genetica. Di questo gene ne esistono diverse forme che si caratterizzano come selvatica (associata con una colorazione marrone del pelo). Nella popolazione ci sono 
quattro forme di questo gene o alleli $C$ diversi con mutazioni diverse in punti specifici. Un coniglio albino \`e omozigote $cc$ di un allele non funzionante con assenza di 
pigmentazione. Un coniglio himalaiano \`e albino tranne che nella parte del muso, una porzione delle orecchie e nelle zampe $c^hc^h$, omozigote ma variante allelica di Agouti 
(selvatico). Il cincill\`a presenta chiazze grigie, con alleli e fenotipi diversi. Per capire che sono alleli multipli con interazioni l'uno con l'altro si fanno incroci. Con gli incroci
si nota che l'albino \`e un allele recessivo: deve essere omozigote $cc$ ma l'himalaiano \`e dominante sull'allele $c$ albino: esiste un himalaiano $c^hc$. Questa cosa \`e estensibile: 
il cincill\`a pu\`o essere eterozigote di $c^{ch}c^h$ e presentare codominanza. Il cincill\`a chiaro pu\`o essere eterozigote con albino $c^{ch}c$. Il selvatico oltre a essere omozigote 
pu\`o essere eterozigote con gli altri e presenta dominanza completa. Nella serie allelica ci sono relazioni complesse di dominanza e $C$ domina tutti gli altri e si trova un'ordine di 
dominanza: $c^+>c^{ch}\ge c^h>c$. 
\paragraph{Eterozigosi composita}
L'eterozigosi composita \`e un esempio di allelismo multiplo: due alleli mutanti si trovano in un individuo e la gravit\`a del fenotipo dipende dall'ordine di ipomorfismo. Inoltre 
possono intervenire anche aspetti ambientali. 
\subsection{Alleli letali}
Si definiscono geni essenziali quei geni che se malfunzionanti portano alla morte anticipata dell'organismo. Quando un allele per tale gene \`e presente in forma mutata e porta
alla comparsa del fenotipo drammatico si dice allele letale. Per un allele letale dominante sia in omo che in eterozigosi si presenta il fenotipo e pertanto tende ad autolimitarsi nella
sua diffusione. Se invece \`e recessivo il fenotipo non si presenta in eterozigosi e pertanto possono rimanare nella popolazione protetti dall'allele selvatico. 
\subsubsection{Esempi}
\paragraph{Gatto di Man}
Il gatto dell'isola di Man in eterozigosi presenta una deformazione della spina dorsale e assenza di coda. In eterozigosi non \`e letale ma riduce l'aspettativa di vita in quanto
ha effetto su altri fenotipi, mentre in omozigosi si ha un aborto spontaneo. 
\paragraph{Topo giallo}
Gli esperimenti sul topo giallo vengono svolti nei primi del $900$ da Morgan. Svolge un incrocio tra topi Agouti e selvatici e altri che presentano una colorazione del pelo gialla per
studiare i principi di Mendel. Prende una popolazione Agout $AA$ e una gialla $AA_y$. Il risultato \`e quello aspettato con $\frac{1}{2}$ della popolazione Agouti e l'altra gialla. 
Incrociando successivamente due topi gialli di questa generazione nota come $\frac{2}{3}$ siano gialli e $\frac{1}{3}$ siano Agouti, che sembrerebbe andare in contraddizione con le
regole mendeliane. La spiegazione \`e che una categoria sia invisibile, si sia formata a livello di gametogenesi e fecondazione ma avviene un aborto spontaneo in quanto l'allele in 
omozigosi \`e letale.
\subparagraph{Letalit\`a di un gene responsabile della colorazione del pelo}
Nel $1993$ si studia la ragione per cui una mutazione di un gene responsabile della colorazione del pelo porta alla morte dell'embrione se in omozigosi. Il gene si trova nel cromosoma
$2$ e ha come vicino il gene Raly, che ha tra i trascritti uno per la produzione di una proteina responsabile della corretta espressione genica e legante RNA. La mutazione del topo 
giallo non \`e puntiforme ma \`e una delezione del gene Raly, dello spazio intergenico e del promotore per il gene Agouti. Il colore giallo \`e dovuto a una sovraespressione del gene,
ora controllato dal promotore di Raly e la letalit\`a in omozigosi \`e dovuta alla mancanza della proteina che porta a un deragliamento dello sviluppo embrionale. 
\paragraph{Alleli letali umani}
Esempi di alleli letali per l'uomo sono la sindrome di Tay-Sachs, la fibrosi cistica e la fenilchetonuria (\emph{PKU}), non tutti portano a un aborto ma causano un effetto visibile in
et\`a infantile riducendo l'aspettativa di vita. La qualit\`a della vita delle persone affette \`e migliorabile grazie alle interazioni dei geni con l'ambiente in modo da modulare
con esse la letalit\`a. 



