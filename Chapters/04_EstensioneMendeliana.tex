\chapter{Estensione dell'analisi genetica mendeliana}
Se il genetista classico si occupava di analizzare il fenotipo degli organismi che studiava quello moderno si occupa di studiare i meccanismi allelici come la dominanza, la 
recessivit\`a e le loro interazioni come il mascheramento. La cellula verr\`a considerata come una rete globale di interazione genica. Le interazioni tra i geni vengono studiati 
attraverso somiglianze fenotipiche tra i mutanti che permettono di aggregare i geni secondo la loro funzione. Per fare questo uno studio analizza il Saccharomyces cerevisiae costruendo
$23$ milioni di doppi mutanti e identificando \num{550000} interazioni positive e \num{350000} negative. I geni vengono poi raggruppati per funzione in un metagenoma che
evidenzia come i geni essenziali sono densamente connessi. 
\section{Pathway genetici}
Si nota pertanto come i geni non lavorano in modo isolato: nascono delle interazioni complicate dalle diversit\`a tra gli alleli. Le collaborazioni tra i geni che portano alla
creazione dei pathway possono essere di tipi diversi:
\begin{itemize}
	\item Pathway biosintetici: geni che producono pathway di enzimi che portano alla sintesi di un composto molecolare.
	\item Pathway di trasduzione del segnale: geni che apportano piccole modifiche a molecole permettendo il loro trasporto e localizzazione creando vie di segnalazione.
	\item Pathway di sviluppo: geni che influenzano aspetti di crescita e differenziamento di parti del corpo o strutture intercellulari.
\end{itemize}
\subsection{Lievito gemmante S. cerevisiae e Neurospora crassa}
I lieviti sono organismi eucarioti che possono esistere sia in forma aploide che diploide. Nel secondo caso sono in grado di fare meiosi producendo spore molto resistenti alle condizioni ambientali.
Per S. cerevisiae la meiosi \`e particolare in quanto i suoi prodotti sono confinati in un asco. Micromanipolando gli aschi si pu\`o seguire la meiosi in diretta. Una cosa analoga avviene nella 
Neurospora crassa, in cui la meiosi viene ordinata spazialmente: le cellule si organizzano in un astuccio compatto occupando posizioni specifiche. In questo caso si ha una mitosi dopo la seconda 
meiosi ($8$ gameti totali) disposti ordinatamente in base alla loro origine. Le spore, quando appoggiate singolarmente su un terreno di coltura sono in grado di crescere e vivere come aploidi. L'ottade
ordinata permette di dedurre eventi di crossing over dalla disposizione delle spore. Nell'asco si nota ha un'alternanza di alleli dominanti e recessivi in modo ordinato. 
\subsubsection{Mappatura del pathway biosintetico}
L'esperimento svolto da Beadle e Tatum cerca di mappare la genetica di un percorso biosintetico legato alla produzione e sintesi dell'amminoacido arginina. In una provetta con Neurospora si 
utilizzano raggi X per mutagenizzare nella zona dei corpi fruttiferi. Dopo la mutagenizzazione si incrociano gli individui e si microdisseziona l'asco, si pongono le spore in terreno di coltura e si
osserva quali sopravvivono. In particolare il terreno \`e in assenza di arginina e si isola un mutante \emph{arg}: si escludono prima le spore che hanno subito mutazioni ai geni essenziali facendo 
crescere in terreno completo i mutanti. Da questo pool si trasferiscono gli aploidi in terreno minimo in modo da isolare i mutanti di interesse. Per determinare la natura dei mutanti si creano quattro
tereni: uno di controllo, uno minimo, uno in assenza di vitamine e uno in assenza di amminoacidi. Si osserva come i mutanti non crescono in assenza di amminoacidi. La mutazione pertanto deve coinvolgere
la produzione di amminoaicdi. Per determinare quale amminoacido non riesce a sintetizzare si pone in coltura in terreni in cui manca un unico amminoacido e si nota che il mutante \`e per arginina. 
\subsubsection{Studio dei mutanti}
Dopo aver isolato i mutanti si cerca di ricostruire la loro storia molecolare. Si prendono pertanto cloni dei mutanti isolati precedentemente e si fa un test di complementazione per prototrofia: 
facendo accoppiare due aploidi si cerca di verificare se complementano. In caso di complementazione si ha un riacquisto della prototrofia e il diploide deve poter crescere su un terreno di arginina. In
questo caso pertanto i mutanti devono essere mutati su due geni diversi recessivi che fanno parte di un pathway biosintetico. Per capire dove si trova la mutazione nel pathway biosintetico i mutanti 
vengono fatti crescere su terreni contenenti precursori dell'arginina nel pathway: \emph{Ornitina}, \emph{Citrullina} e \emph{Arginina} e si nota come: 
\begin{itemize}
	\item \emph{arg7} cresce sia in presenza di ornitina, citrullina e arginina, pertanto la mutazione deve trovarsi a valle di questi tre prodotti nel pathway.
	\item \emph{arg3} cresce unicamente in presenza di citrullina e arginina, ma non di ornitina: la mutazione si trova nell'enzima che trasforma ornitina in citrullina.
	\item \emph{arg1} cresce unicamente in presenza di arginina, ma non di ornitina e citrullina: la mutazione si trova nell'enzima che  trasforma citrullina in arginina.
\end{itemize}
Si nota pertanto come il gene \emph{arg7} si trova a monte del pathway rispetto agli altri due, mentre \emph{arg1} \`e quello pi\`u a valle.
\section{Fattori addizionali su un singolo locus}
L'estensione naturale dell'analisi genetica mendeliana \`e considerare che dato un certo locus non \`e detto che vi si trova un solo gene. Inoltre \`e naturale come i geni non 
lavorano in modo isolato e per capire in un fenotipo si devono considerare pi\`u geni. I geni inoltre si confrontano con l'ambiente cellulare, extracellulare ed esterno. Questi tre
fattori devono essere tutti considerati quando si studiano i geni. Gli esperimenti di Mendel usavano alleli che mostravano dominanza completa, ma esistono altre interazioni.
\subsection{Tipi di dominanza}
\subsubsection{Dominanza incompleta}
Si ha dominanza incompleta quando in eterozigosi si trova un fenotipo intermedio rispetto a quello dell'omozigote dominante e recessivo: si ha la possibilit\`a di vedere fenotipicamente l'eterozigote.  
Avviene in quanto si ha per alleli diversi una diversa quantit\`a di prodotto. Ogni allele conta nel contribuire a un tratto fenotipico, importante nella genetica multifattoriale per mappare il contributo
quantitativo di alleli sparsi nel genoma  Nella dominanza incompleta ci si aspettano gradazioni intermedie ai due estremi di omozigosi dominante e omozigosi dell'allele meno funzionale. 
\paragraph{Esempi}
\begin{itemize}
	\item In alcuni fiori il dominante puro ha fenotipo rosso il recessivo puro bianco, mentre l'eterozigote \`e rosa.
	\item Dimensione del pomodoro.
	\item Colore del piumaggio di polli e galline.
	\item Colore del frutto della melanzana: vira da viola a bianco.
\end{itemize}
Il carattere in eterozigosi non si deve necessariamente trovare a met\`a tra le omozigosi ma anche a diverse gradazioni. Inoltre i rapporti fenotipici e genotipici sono identici in $1:2:1$. 
Si noti come le differenze nei sistemi di modello indicano come probabilmente si tratti di un principio universale. 
\subsubsection{Codominanza}
La codominanza avviene quando due alleli mostrano contemporaneamente la loro presenza. A differenza della dominanza incompleta non si produce un fenotipo intermedio ma l'effetto di 
ogni gene \`e discreto e in codominanza sono presenti entrambe le caratteristiche intermedie.
\paragraph{Esempi}
\begin{itemize}
	\item Lenticchie con chiazze e puntini, in eterozigosi si mostrano sia chiazze che puntini. 
	\item Gruppi sanguigni nel modello $AB0$: 
		\begin{itemize}
			\item Omozigosi recessiva $ii$: non si hanno antigeni e si producono anticorpi $A$ e $B$.
			\item Eterozigosi $I^Ai$ ed omozigosi $I^AI^A$: si possiede l'antigene $A$ e si producono anticorpi $B$.
			\item Eterozigosi $I^Bi$ ed omozigosi $I^BI^B$: si possiede l'antigene $B$ e si producono anticorpi $A$.
			\item Eterozigosi $I^BI^A$: si possiedono gli antigeni $A$ e $B$ e non si producono anticorpi.
		\end{itemize}
	\item Gruppi sanguigni nel modello $MN$:
		\begin{itemize}
			\item Omozigosi recessiva $ll$: non si hanno antigeni e non hi producono anticorpi $M$ e $N$. 
			\item Eterozigosi $L^Ml$ ed omozigosi $L^ML^M$: si possiede l'antigene $M$ e si producono anticorpi $N$.
			\item Eterozigosi $L^Nl$ ed omozigosi $L^NL^N$: si possiede l'antigene $N$ e si producono anticorpi $M$.
			\item Eterozigosi $L^NL^M$: si possiedono gli antigeni $M$ e $N$ e non si producono anticorpi.
		\end{itemize}
\end{itemize}
\subsubsection{Allelismo multiplo}
Nell'allelismo multiplo i geni esistono in pi\`u di due forme alleliche e ogni individuo pu\`o possedere solo due alleli diversi di uno stesso gene. In questo caso si distinguono
tre categorie tra le quali esiste un ordine di dominanza:
\begin{itemize}
	\item Alleli amorfici o recessivi non funzionanti o nulli.
	\item Alleli parzialmente funzionanti o ipomorfici recessivi rispetto a quelli pi\`u funzionali.
	\item Alleli funzionanti o dominanti rispetto agli altri.
\end{itemize}
\paragraph{Esempi}
\subparagraph{Conigli}
Considerando i conigli come organismi modello e incrociandoli si nota la distribuzione di fenotipi e pertanto le caratteristiche degli alleli. In un certo cromosoma si trova un locus
con un'informazione genetica. Di questo gene ne esistono diverse forme:
\begin{itemize}
	\item Selvatica: associata con una colorazione marrone del pelo: $c^+\_$.
	\item Cincill\`a: chiazze grigie: $c^{ch}c^{ch}$, $c^{ch}c$, $c^{ch}c^h$  con codominanza 
	\item Himalaiano: chiazze su parte del muso, porzione di orecchie e zampe: $c^hc^h$ o $cc^h$.
	\item Albino: nessun tipo di colorazione: omozigote $cc$.
\end{itemize}
Si notano pertanto quattro alleli diverse del gene che stanno in una relazione di dominanza tra di loro: 
\[c^+>c^{ch}>c^h>c\]
\subparagraph{Eterozigosi composita}
L'eterozigosi composita \`e un esempio di allelismo multiplo: due alleli mutanti si trovano in un individuo e la presenza di due mutazioni causano la comparsa in eterozigosi del fenotipo recessivo. 
La gravit\`a del fenotipo dipende dall'ordine di ipomorfismo oltre che da aspetti ambientali.
\subsubsection{Il livello di fenotipo osservato pu\`o influenzare la dominanza}
In base al punto di vista con cui si considera il fenotipo si possono osservare diversi tipi di dominanza.
\paragraph{Fibrosi cistica}
\subparagraph{Effetti}
La fibrosi cistica \`e un'anomalia genetica considerata recessiva: le persone affette producono una grande quantit\`a di muco denso e viscoso che ostruisce le vie aeree dei poloni e intasa i dotti che
collegano pancreas e intestino, causando infezioni respiratorie e problemi digestivi. 
\subparagraph{Analisi molecolare}
Il gene responsabile della fibrosi cistica \`e localizzato sul braccio lungo del cromosoma $7$ e codifica per il regolatore di conduttanza transmembrana della fibrosi cistica \emph{CFTR}, che agisce come
una barriera che regola il passaggio degli ioni cloruro attraverso la membrana. I soggetti affetti ne possiedono una forma mutata e disfunzionale che causa un accumulo degli ioni nella cellula. 
In condizione di eterozigosi vengono prodotte entrambe le proteine, pertanto a livello molecolare gli alleli sono codominanti, ma essendo che un allele funzionante produce \emph{CFTR} in quantit\`a 
sufficiente per il trasporto degli ioni non presenta effetti dannosi e si rivela recessiva a livello fisiologico. 
\subsubsection{Caratteristiche della dominanza}
La dominanza \`e il risultato di interazioni fra geni in un medesimo locus, ovvero un'interazione allelica. La dominanza non altera il modo in cui gli alleli sono ereditati, ma solo come vengono
espressi nel fenotipo. \`E pertanto un'interazione fra i prodotti dei geni e dipende al livello al quale il fenotipo viene osservato. 
\subsection{Penetranza ed espressivit\`a}
\subsection{Penetranza}
Una parete della popolazione pu\`o possedere un certo genotipo ma non mostrare il fenotipo atteso, ovvero il gene ha penetranza incompleta. Si definisce pertanto penetranza come la percentuale id 
individui di un certo genotipo che esprimono il fenotipo atteso.
\subsubsection{Esempi}
\paragraph{Polidattilia}
La polidattilia \`e un carattere dominante, ma casualmente alcuni individuo con l'allele possono avere un numero normale di dita nelle mani e nei piedi: il gene della polidattilia non \`e completamente
dominante. Il fenotipo non si manifesta nel $25$-$30\%$ degli individui portatori dell'allele.
\paragraph{Hungtington e SLA}
La penetranza di corea di Huntington e la SLA presentano una penetranza variabile nel tempo: inizia a $0$ durante la nascita e sale con l'et\`a. Mettendo sull'asse $X$ l'et\`a e sull'asse $Y$ la
penetranza si osserva un grafo a sigmoide.
\subsection{Espressivit\`a}
Il grado o intensit\`a con cui un particolare genotipo \`e espresso in un fenotipo.
\subsubsection{Esempi}
\paragraph{Polidattilia}
La polidattilia mostra un'espressivit\`a variabile: il numero, la forma e la posizione delle dita in eccesso variano tra gli individui.
\paragraph{Gradazioni  di pezzatura del mantello del Bracco}
Colorazione del mantello \`e dovuta a $7$ geni, ma con lo stesso genotipo si osservano colorazioni diverse. Le differenze sono dovute a modifiche dell'espressivit\`a di tali geni.
\paragraph{Mascella Asburgo}
Il prognatismo mandibolare \`e un carattere dominante a penetranza incompleta ed espressivit\`a variabile.
\paragraph{Sindrome di Waardenburg}
La sindrome di Waardenburg causa quattro caratteristiche:
\begin{multicols}{2}
	\begin{itemize}
		\item Capelli precocemente grigi.
		\item Perdita di udito.
		\item Ciocca di capelli bianca nella zona della fronte.
		\item Eterocromia dell'iride.
	\end{itemize}
\end{multicols}
Gli individui affetti possono presentare caratteristiche in combinazioni diverse.
\subsection{Alleli letali}
Si definiscono geni essenziali quei geni che se malfunzionanti portano alla morte anticipata dell'organismo. Quando un allele per tale gene \`e presente in forma mutata e porta
alla comparsa del fenotipo drammatico si dice allele letale. Per un allele letale dominante sia in omo che in eterozigosi si presenta il fenotipo e pertanto tende ad autolimitarsi nella
sua diffusione. Se invece \`e recessivo il fenotipo non si presenta in eterozigosi e pertanto possono rimanare nella popolazione protetti dall'allele selvatico. 
\subsubsection{Esempi}
\paragraph{Gatto di Man}
Il gatto dell'isola di Man in eterozigosi presenta una deformazione della spina dorsale e assenza di coda. In eterozigosi non \`e letale ma riduce l'aspettativa di vita in quanto
ha effetto su altri fenotipi, mentre in omozigosi si ha un aborto spontaneo. 
\paragraph{Topo giallo}
Gli esperimenti sul topo giallo vengono svolti nei primi del $900$ da Morgan. Svolge un incrocio tra topi Agouti e selvatici e altri che presentano una colorazione del pelo gialla per
studiare i principi di Mendel. Prende una popolazione Agouti $AA$ e una gialla $AA_y$. Il risultato \`e quello aspettato con $\frac{1}{2}$ della popolazione Agouti e l'altra gialla. 
Incrociando successivamente due topi gialli di questa generazione nota come $\frac{2}{3}$ siano gialli e $\frac{1}{3}$ siano Agouti, che sembrerebbe andare in contraddizione con le
regole mendeliane. La spiegazione \`e che una categoria sia invisibile, si sia formata a livello di gametogenesi e fecondazione ma avviene un aborto spontaneo in quanto l'allele in 
omozigosi \`e letale.
\subparagraph{Letalit\`a di un gene responsabile della colorazione del pelo}
Nel $1993$ si studia la ragione per cui una mutazione di un gene responsabile della colorazione del pelo porta alla morte dell'embrione se in omozigosi. Il gene si trova nel cromosoma
$2$ e ha come vicino il gene Raly, che ha tra i trascritti uno per la produzione di una proteina responsabile della corretta espressione genica e legante RNA\@. La mutazione del topo 
giallo non \`e puntiforme ma \`e una delezione del gene Raly, dello spazio intergenico e del promotore per il gene Agouti. Il colore giallo \`e dovuto a una sovraespressione del gene,
ora controllato dal promotore di Raly e la letalit\`a in omozigosi \`e dovuta alla mancanza della proteina che porta a un deragliamento dello sviluppo embrionale. 
\subsubsection{Alleli letali umani}
Esempi di alleli letali per l'uomo sono la sindrome di Tay-Sachs, la fibrosi cistica e la fenilchetonuria (\emph{PKU}), non tutti portano a un aborto ma causano un effetto visibile in
et\`a infantile riducendo l'aspettativa di vita. La qualit\`a della vita delle persone affette \`e migliorabile grazie alle interazioni dei geni con l'ambiente in modo da modulare
con esse la letalit\`a. 
\paragraph{Sindrome di Tay-Sachs}
La sindrome di Tay-Sachs \emph{TDS} \`e causata da mutazioni in omozigosi o eterozigosi composta della subunit\`a $\alpha$ del gene esosaminidasi $A$ \emph{HEXA} sul cromosoma $15q23$
\`E un disordine autosomale recessivo e progressivo neurodegenerativo che nella forma infantile \`e fatale nei primi $3$ anni di vita. \`E caratterizzato da un insieme di ritardi
nello sviluppo infantile seguito da paralisi, demenza e cecit\`a fino alla morte. Si riconosce grazie a un'area grigio biancastra intorno alla fovea centralis dell'occhio a causa di 
cellule gangliali ricche di lipidi. Una verifica patologica \`e fornita da neuroni a forma di pallone nel sistema nervoso centrale. \`E utile per un riconoscimento una precoce e 
persistente estensione della risposta ai suoni. 
\subparagraph{Genetica molecolare}
La lesione pi\`u frequente nella sindrome \`e un'inserzione di $4$ coppie di basi nell'esone $11$ del gene \emph{HEXA}. Il gene responsabile per la forma giovanile \`e allelico a quello
responsabile per la forma infantile e questi pazienti con la deficienza parziale sopravvivono fino ai $15$ anni. Il gene \emph{HEXA} codifica per la subunit\`a $\alpha$ dell'enzima
lisosomale $\beta$-esosaminidasi che insime al cofattore \emph{GM2} catalizza la degradazione del ganglioside \emph{GM2} e altre molecole contenenti terminale $N$-acetil esosaminasi. 
Mutazioni nella subunit\`a $\alpha$ o $\beta$ portano ad un accumulo del ganglioside \emph{GM2} nei neuroni e disordini neurodegenerativi detti \emph{GM2 gangliosidosi}. La malattia 
\`e causata da $78$ diverse mutazioni semplici, inserzioni o delezioni.
\paragraph{Interazioni geni-ambiente nella \emph{PKU}}
Le cellule viventi e gli organismi complessi hanno la necessit\`a di interagire con l'ambiente per acquisire materia causando una pressione evolutiva per fenomeni metabolici che
prendono le macromolecole dall'esterno per produrre energia o altre macromolecole. Si nota come possano esistere errori congeniti nel metabolismo, ovvero nei complessi pathway 
metabolici ci possono essere mutazioni di geni codificanti enzimi coinvolti in esso che portano ad un blocco del cammino. Un esempio di questo \`e la fenilchetonuria \emph{PKU}, 
che porta a una scorretta elaborazione della fenilanalina introdotta dalla dieta. Il fenotipo della malattia viene influenzato da vari fattori come la quantit\`a di fenilanalina
introdotta con la dieta (necessaria in quanto amminoacido essenziale), le mutazioni del gene per la fenilanalina idrossilasi (che la trasforma in tirosina), la produzione del cofattore 
necessario al suo funzionamento tetraidrobiopterina \emph{TBH}, i livelli di produzione di acido fenilpiruvico alla fine del pathway che viene trasportato fuori dal fegato nel sangue, 
la sua interazione con la barriera ematoencefalica e infine la gestione del suo accumulo nelle cellule nervose. Si nota come la complessit\`a delle interazioni con l'ambiente pu\`o 
rendere molto difficile il riconoscimento dei principi mendeliani attraverso lo studio degli effetti fenotipici, il determinismo genetico \`e reso pi\`u complesso. 
\subsubsection{Letalit\`a sintetica}
Due geni o due proteine vengono definiti letali sintetici quando la deficienza nell'espressione di uno dei due non compromette la vitalit\`a, mentre la contemporanea alterazione in 
entrambi \`e incompatibile con la sopravvivenza cellulare. Considerando due oggetti $A$ e $B$ che interagiscono tra di loro e con il DNA localizzandone una porzione creando 
un'interazione con esso. Si indica con $A^+$ e $B^+$ la loro forma funzionale e con $A^-$ e $B^-$ la loro forma non funzionale. Nel caso della formazione del 
complesso $A^+B^-$ o $A^-B^+$ il complesso, nonostante la forma modificata \`e ancora in grado di legare il DNA e svolgere la sua funzione con costante di affinit\`a minore ma comunque
abbastanza per impedire la nascita del fenotipo drammatico. Nel caso in cui invece entrambi siano mutati e si formi il fenotipo $A^-B^-$ il complesso perde la sua affinit\`a con il 
DNA causando la comparsa del fenotipo drammatico dovuto alla sua perdita di funzionalit\`a. 
\paragraph{Terapie antitumorali}
Questo concetto viene utilizzato per terapie antitumorali: i tumori possono presentare una
mutazione somatica con un difetto in uno degli elementi del complesso e per far perdere ad esso la funzione essenziale si utilizza un farmaco in grado di colpire il partner generando
la situazione letale nelle cellule tumorali. La tossicit\`a per le cellule non tumorali viene mantenuta bassa in quanto in esse non \`e presente la mutazione del partner. 
\section{Interazioni geniche e rapporti mendeliani modificati}
Si intende per interazione genica un'interazione in cui geni posti su loci diversi non sono indipendenti nella loro espressione fenotipica. I prodotti dei geni si combinano per produrre nuovi fenotipi
non prevedibili osservando un singolo locus.
\subsection{Interazione genica che produce nuovi fenotipi}
I geni su due loci interagiscono per determinare una singola caratteristica.
\subsubsection{Esempi}
\paragraph{Polli}
Si prende in considerazione la forma della cresta dei polli. Si consideri una generazione parentale in cui un individuo possiede una cresta frastagliata o Wyandotte e l'altro a fagiolo
o Brahma. Tentando di studiare la base genotipica per la differenza si incrociano gli individui. Si nota come nella $F_1$ tutti gli individui possiedano un fenotipo con cresta a noce. 
Incrociando ancora si ottiene una $F_2$ in cui si osservano quattro fenotipi:
\begin{itemize}
	\item Wyandotte $\frac{3}{16}$.
	\item Brahma $\frac{3}{16}$.
	\item Noce $\frac{9}{16}$.
	\item A cresta singola o Longhorn $\frac{1}{16}$.
\end{itemize}
I rapporti tra i fenotipi sembrano indicare un rapporto mendeliano e come la morfogenesi della cresta sia influenzata da due morfogeni. Chiamando i due geni $Rr$ e $Pp$ si determina il
fenotipo:
\begin{itemize}
	\item Wyandotte $RRpp$.
	\item Brahma $rrPP$.
	\item Noce $R-P-$.
	\item Longhorn $rrpp$.
\end{itemize}
L'assenza dei geni porta a un programma di base, mentre la presenza di uno dei due porta a situazioni intermedie tra entrambi presenti e entrambi assenti. 
\paragraph{Peperoni}
Si consideri una popolazione parentale con un peperone rosso e verde provenienti dalla stessa pianta (si possono incrociare e ottenere una progenie fertile). La generazione $P$ \`e 
rosso con verde e la $F_1$ presenta peperoni tutti rossi. Autofecondando la $F_1$ si ottiene una $F_2$ in cui compaiono due nuovi fenotipi:
\begin{itemize}
	\item Marrone $\frac{3}{16}$.
	\item Giallo $\frac{3}{16}$.
	\item Rosso $\frac{9}{16}$.
	\item Verde $\frac{1}{16}$.
\end{itemize}
La numerologia esclude l'allelismo multiplo. I rapporti tra i fenotipi sembrano indicare un rapporto mendeliano e come il colore del peperone sia influenzato da due geni. Chiamando i due
geni $Cc$ e $Rr$ si determina il
fenotipo:
\begin{itemize}
	\item Marrone $R-cc$.
	\item Giallo $rrC-$.
	\item Rosso $R-C-$.
	\item Verde $rrcc$.
\end{itemize}
La $F_1$ \`e pertanto un'eterozigote da ambo le parti e il fenotipo \`e l'interazione di quattro alleli in due geni con dominanza completa. Quando entrambi recessivi non si produce
pigmento e rimane il colore dato dalla clorofilla.
\paragraph{Lenticchie}
Si consideri un altro incrocio che conferma lo stesso principio. Si consideri una popolazione di lenticchie composta da individui con colore marrone-rosso $AAbb$ e con colore grigio 
$aaBB$. \`E un modello a due geni e la popolazione parentale \`e omozigote trans per entrambi. Nella $F_1$ tutti gli individui sono eterozigoti con entrambi associato con il fenotipo 
marrone. Incrociando $F_1$ si ottiene $F_2$ e si studia il fenotipo:
\begin{itemize}
	\item Marrone $\frac{9}{16}$, $A-B-$.
	\item Marrone-rosso $\frac{3}{16}$, $A-bb$.
	\item Grigio $\frac{3}{16}$, $aaB-$.
	\item Verde $\frac{1}{16}$, $aabb$.
\end{itemize}
\paragraph{Drosophila melanogaster}
Si consideri la pigmentazione dell'occhio composito della Drosophila: questo si presenta o rosso o bianco. Incrociando i due individui si nota come in $F_1$ tutta la progenie abbia
occhi rossi. Incrociando un individuo di $F_1$ $Bw^+BwSt^+St$ e si incrocia con un individuo con gli occhi bianchi $BwBwStSt$. Si ottiene una $F_2$ in cui si notano nuovi fenotipi. 
\begin{itemize}
	\item Occhio rosso $\frac{1}{4}$, $Bw^+St^+$.
	\item Occhio scarlatto $\frac{1}{4}$, $Bw^+St$.
	\item Occhio marrone $\frac{1}{4}$, $BwSt^+$.
	\item Occhio bianco $\frac{1}{4}$, $BwSt$.
\end{itemize}
Si nota come il rapporto \`e di $\frac{1}{4}$ si lavora con due geni e quattro alleli e il fenotipo \`e guidato da un omozigosi di entrambi e nel fenotipo \`e guidato dal fenotipo 
dell'individuo doppio eterozigote che quando incrociato con un doppio recessivo. 
\subsection{Fenocopia}
Si intende per fenocopia geni diversi che concorrono allo stesso fenotipo.
\subsubsection{Esempi}
\paragraph{Colorazione dell'occhio di Drosophila}
Considerando l'esempio precedente sulla colorazione dell'occhio di Drosophila melanogaster si nota come oltre ai geni precedenti che producono il pigmento, detti $b$ e $v$, ne esiste
un altro che controlla il suo trasporto all'occhio. Questo gene esiste in due forme alleliche: quella dominante o funzionale $w^+$ che mostra il fenotipo del colore prodotto dai primi
due geni e quella recessiva o non funzionale $w$ che mostra un fenotipo di colore bianco. Si nota pertanto come i geni concorrono allo stesso fenotipo e si dicono pertanto fenocopie. 
\subsection{Interazione genica con epistasi}
Si intende per epistasi un tipo di interazione genica in cui un gene maschera e nasconde l'effetto fenotipico di un altro su un locus diverso. IL gene che opera il mascheramento viene detto 
epistatico, mentre quello il cui effetto viene inibito \`e detto gene ipostatico. I geni epistatici possono essere dominanti o recessivi.
\subsubsection{Epistasi recessiva}
Nell'epistasi recessiva l'omozigote recessivo di un gene maschera gli effetti di entrambi gli alleli di un altro. Si ottiene un rapporto tra fenotipi di $9:3:4$.
\paragraph{Esempi}
\subparagraph{Labrador}
Si consideri il colore del pelo del labrador: 
\begin{itemize}
	\item $B\_\ E\_$ nero.
	\item $bbE\ \_$ marrone.
	\item $\_\_\ ee$ biondo.
\end{itemize}
Un locus genetico determina il tipo di pigmento prodotto dalle cellule cutanee: $B$ codifica per il pigmento nero, mentre $b$ per il marrone. Gli alleli del secondo locus influenzano la sua deposizione nel
fusto del pelo: l'allele dominante $E$ permette la deposizione del pigmento scuro, mentre quello recessivo $e$ ne impedisce la deposizione, causando una colorazione bionda del pelo. Si nota perci\`o
come la presenza del genotipo $ee$ maschera l'espressione degli alleli nero e marrone sul primo locus. \\
Si incroci ora una generazione parentale nero $BB\ EE$ con biondo $bb\ ee$. Si nota come si forma una $F_1$ tutta nera con $BbEe$. Con un incrocio stile mendeliano tra $F_1$ eterozigoti
si nota come compare un nuovo colore: i fenotipi diventano nero, marrone e biondo e sono in rapporto $9:3:4$. Si nota come si ha un'interazione tra gli alleli del gene $E$ e $B$. Si nota come il biondo 
compare con doppia omozigosi recessiva $ee$ e con tutte le combinazioni possibili dello stato allelico di $B$: $ee$ causa biondo. \\
Questo vuol dire che omozigosi recessiva di $e$ nasconde la genetica di $B$. Questa \`e un epistasi recessiva in quanto il mascheramento \`e attuato da un omozigosi recessiva. $e$ \`e epistatico e $b$ \`e 
ipostatico. In assenza di omozigosi recessiva per $E$ l'omozigosi $bb$ porta a una colorazione intermedia marrone. Questa \`e un'epistasi recessiva semplice in quanto si distingue il mascheramento da un 
secondo gruppo fenotipico dove il mascheramento non ci sarebbe ma l'omozigosi recessiva del gene ipostatico fa si che il fenotipo si differenzi da quello selvatico legato alla presenza di un allele 
dominante per entrambi i geni.
\subparagraph{Gruppo sanguigno \emph{AB0} - fenotipo Bombay}
Gli alleli sul locusi $AB0$ codificano per antigeni dei globuli rossi, breve catene di carboidrati racchiusi nelle membrane dei globuli rossi. La differenza tra $A$ e $B$ dipende da differenze chimiche
nello zucchero terminale della catena. $I^A$ e $I^B$ codificano per enzimi diversi che aggiungono zuccheri $A$ e $B$ al termine di catene dei carboidrati. Il substrato comune su cui agiscono questi
enzimi \`e una molecola $H$. $i$ non aggiunge zucchero a $H$ n\`e codifica un enzima funzionale. Gli individui con fenotipo Bombay sono omozigoti per una mutazione recessiva $h$ che codifica per un 
enzima difettoso che non \`e un grado di produrre $H$ e pertanto non vengono sintetizzati antigeni $AB0$. \\
Gli alleli sul locusi $AB0$ sono ipostatici rispetto all'allele recessivo $h$ epistatico. Come molti meccanismi di epistasi un gene che esercita il suo effetto su una fase precoce del processo biochimico
risulta epistatico sui geni che influenzano le fasi successive in quanto i loro effetti dipendono dal prodotto delle reazioni precedenti. 
\subsubsection{Epistasi dominante I}
Nell'epistasi dominante I l'allele dominante di un gene nasconde gli effetti di entrambi gli alleli di un altro. Si ottiene un rapporto tra fenotipi di $12:3:1$.
\paragraph{Esempi}
\subparagraph{Zucca}
Due loci determinano il colore dei frutti in una variet\`a di zucca estiva con tre colori:
\begin{itemize}
	\item $12$: $9$ $W\_\_\_$ bianchi.
	\item $3$: $wwY\_$ giallo.
	\item $1$: $wwyy$ verde.
\end{itemize}
Quando una pianta omozigote a colore bianco viene incrociata con una pianta omozigote a colori verdi e si riincrocia $F_1$ si ottengono i risultati indicati precedentemente. Si nota come in $F_2$ le
zucche bianche e colorate stanno in rapporto $3:1$. Questo suggerisce che un allele dominante su un locus impedisce la produzione di pigmento generando una progenie bianca. Pertanto il genotipo $W\_$
causa la scomparsa del pigmento, mentre negli altri compaiono zucche colorate. Tra $ww$ di $F_2$ le zucche gialle e verdi sono in rapporto $3:1$, a causa di un secondo locus che determina il pigmento, 
rispettivamente $Y\_$ e $yy$. \\
Il pigmento giallo \`e un prodotto in un processo a due fasi: una sostanza incolore $A$ \`e trasformata dall'enzima $I$ nella sostanza verde $B$ che viene opi trasformata dall'enzima $II$ nella sostanza
$C$, pigmento giallo del flutto. Le piante $ww$ producono l'enzima $I$ e possono essere verdi o gialle. $W$ sul primo locus inibisce la conversione di $A$ in $B$ causando i frutti bianchi. 
Pertanto l'allele $W$ \`e epistatico e maschera l'espressione dei geni che producono i pigmento. \`E epistatico dominante in quanto \`e sufficiente la presenza in singola copia affinch\`e la produzione
del pigmento sia inibita. \\
\subparagraph{Colore dei fiori foxglove}
Questi fiori possono avere colorazione punteggiata bianca, rossa o rosso leggero. Si ottiene una $F_1$ con incrocio di linee pure doppi eterozigoti per due geni coinvolti per il 
fenotipo di cui si valuta la genetica: $DdWw$. Si autoincrocia $F_1$ e si ha una relazione tra genotipo e fenotipo. Si ha un precursore non colorato convertito dai prodotti del gene 
$D$ che produce un pigmento colorato che deve raggiungere il posto corretto: il gene $W$ si occupa di tale trasporto. Quando dominante la pigmentazione viene confinata nei throat 
spots, di fatto fiore bianco con macchie rosse. In omozigosi di $dd$ l'assenza di funzione dovuta ad essa porta ad una colorazione light molto pi\`u leggera ma $W$ condiziona il 
trasporto. I conti suggeriscono che i rapporti sono ancora in $12:3:1$. Quando l'epistasi in omozigosi recessiva di $w$ si ha l'effetto del gene $D$: in dominanza il pigmento scuro
diffuso con fiore rosso, con combinazione di omozigosi recessiva di $dd$ si ha una diffusione di un colore pi\`u sbiadito. 
\subsubsection{Epistasi dominante II}
Nell'epistasi dominante II l'allele dominante di un gene nasconde gli effetti dell'allele dominante di un altro gene. Si ottiene un rapporto tra fenotipi di $13:3$.
\paragraph{Esempi}
\subparagraph{Galline livornesi}
Si valuti la pigmentazione delle galline con $P$ livornese bianca $AABB$ e americana colorata $aabb$. Si ottiene una $F_1$ $AaBb$ e il suo auto incrocio porta alla formazione di 
prodotti fenotipici $13:3$: il colore si ha solo per $A-bb$. Il mascheramento \`e operato dalla presenza di un allele dominante $B$, con epistasi dominante e recessiva in quanto in questo caso la rottura 
dell'epistasi nella contemporaneit\`a dell'omozigosi porta anch'essa all'assenza di colore. L'assenza del gene ipostatico copia il fenomeno di epistasi. 
\subsubsection{Epistassi recessiva doppia}
Nell'epistassi recessiva doppia l'omozigosi per l'allele recessivo a uno dei due loci \`e in grado di sopprimere un fenotipo. Si ottiene un rapporto tra fenotipi di $9:7$.
\paragraph{Esempi}
\subparagraph{Fiori}
Si considerino due fiori con fenotipo recessivo bianco e nell'ipotesi pi\`u semplice mendeliana, una generazione parentale bianca indica un'omozigosi recessiva per l'allele responsabile
del fenotipo. L'incrocio dovrebbe creare solo figli bianchi. Si nota come invece un'incrocio tra fiori bianchi in $F_1$ si ottiene una popolazione con colorazione selvatica viola. 
Questo \`e un fenomeno di complementazione: due genomi incrociati con fenotipo recessivo porta alla comparsa del fenotipo dominante. Si pu\`o interpretare proponendo che i due recessivi
siano fenocopie, con genetica complementare: un fiore \`e $C/C$ e $p/p$, mentre il partner \`e $c/c$ e $P/P$. Accettando questa ipotesi si nota come in $F_1$ tutta la popolazione
sia eterozigota doppia: la presenza di un allele dominante assicura la presenza di un fenotipo selvatico di un fiore colorato. Incrociando ora $F_1\times F_1$ i risultati seguono la
regola dell'incrocio di-ibrido con assortimento indipendente. Scomponendo l'incrocio di-ibrido si attende in $F_2$:
\begin{itemize}
	\item $\frac{3}{4}$ $C/-$:
		\begin{itemize}
			\item $\frac{3}{4}$ $P/-$: $\frac{9}{16}$ $C/-\ P/-$ selvatico
			\item $\frac{1}{4}$ $p/p$: $\frac{3}{16}$ $C/-\ p/p$ bianco.
		\end{itemize}
	\item $\frac{1}{4}$ $c/c$:
		\begin{itemize}
			\item $\frac{3}{4}$ $P/-$: $\frac{3}{16}$ $c/c\ P/-$ bianco.
			\item $\frac{1}{4}$ $p/p$: $\frac{1}{16}$ $c/c\ p/p$ bianco.
		\end{itemize}
\end{itemize}
Osservando le classi fenotipiche si nota come queste stanno in rapporto $9:7$. Questo a causa dell'interazione funzionale di geni diversi. Interpretando questo fenomeno come un 
mascheramento in cui un gene maschera un fenotipo dell'altro si pu\`o determinare qual \`e il gene che maschera l'altro: non \`e facile capire qual \`e che viene mascherato in quanto 
ci si trova in un caso di epistasi recessiva doppia. Senza una comprensione molecolare del percorso biosintetico non si \`e in grado di decidere quale gene determina sull'altro. \\
Si nota come nella via biosintetico ci sia un precursore su cui agisce $C$ che forma un intermedio incolore e un prodotto finale che da il colore sintetizzato da $P$. Con uno dei
due geni non presente il pathway \`e bloccato e si trova assenza di antocianina. In questo caso in $c/c$ la via biosintetica si interrompe con accumulo di precursore. In questa 
interpretazione l'allele $c$ maschera in omozigosi recessiva. Questa si osserva nella condizione di omozigosi di un gene $C$ che essendo a monte della via biosintetica rende il 
genotipo di $P$ irrilevante in quanto l'assenza dell'intermedio porta alla sua incapacit\`a di esprimere la sua funzione. \`E anche doppia in quanto l'assenza di $P$ ha la stessa
conseguenza. L'epistasi recessiva doppia: il gene epistatico \`e $C$ e quello mascherato si dice ipostatico $P$. \\
\subparagraph{Lumache d'acqua}
Il guscio della conchiglia  pu\`o essere marrone o non colorato ed \`e legato alla genetica di una via biosintetica multifase. L'albinismo dipende da due alleli recessivi su uno di due loci diversi. 
Incrociando due lumache albine tutta la progenie $F_2$ risulta albina. Reincrociando $F_1$ invece la $F_2$ risulta per $\frac{9}{16}$ composta da lumache pigmentate e per $\frac{7}{16}$ da lumache 
albine. Il rapporto $9:7$ si presenta quando gli alleli dominanti su due loci $A\_B\_$ producono lumache pigmentate e qualsiasi altro lumache albine.\\
Dal punto di vista biochimico \`e determinato da un processo di produzione del pigmento a due fasi: questo viene prodotto solo dopo che la sostanza $A$ viene trasformata dall'enzima $I$ in una sostanza
$B$ che viene trasformata nel pigmento da un enzima $II$. Pertanto alleli funzionanti di uno dei due loci portano all'albinismo. 

\subsection{Ridondanza genica}
Nella ridondanza genica i rapporti mendeliani sono modificati $15:1$: un fenotipo compare unicamente in doppia omozigosi recessiva. La ridondanza genetica pu\`o dipendere dalla
duplicazione genica. Il gene $A$ e $B$ potrebbero essere geni paraloghi, forse con antenato comune con duplicazione eptopica (su cromosomi diversi) in modo che segreghino in modo
indipendente. In questo caso la distanza dalla nascita dei paraloghi non \`e abbastanza appia per aver diversificato la loro funzione. Sono ridondanti per la funzione biochimica che
per essere persa si devono perdere $4$ alleli. 
\subsubsection{Esempi}
\paragraph{Capsula della pianta borsa del pastore}
Si consideri che questo oggetto ha una forma ovale e una triangolare. Si possono ottenere linee pure e triangolare \`e $TTVV$, mentre l'altro $ttvv$. Si ottiene una $F_1$ doppia 
eterozigote che si incrocia con s\`e stessa: il fenotipo ovale compare solo in doppia omozigosi recessiva. Questo vuol dire che il gene $T$ e quello $V$ interagiscono sullo stesso
punto del pathway, pertanto per perdere il passaggio morfogenetico si rende necessario disattivare entrambe le coppie.
\subsection{Interazione genetica dominante}
L'interazione genica dominante \`e caratterizzata da un tasso di-ibrido $9:6:1$. 
\subsubsection{Esempi}
\paragraph{Zucche} 
Si consideri la forma a disco o allungata o a sfera e si nota come la deviazione come l'omozigosi recessiva per un gene con dominante per l'altro e il complementare possiedono lo 
stesso fenotipo. 
\subsection{Sovradominanza o vantaggio dell'eterozigote}
Si consideri un globulo rosso con forma normale e antigeni se gene $H$ funzionante e stato allelico \emph{AB0}. Nell'altro si trova una forma a falce legata alla presenza di una
mutazione specifica nel gene della beta-globina. A valle di una meiosi si nota come la frequenza dei portatori e dei casi e la distribuzione della malaria: nei residenti in regioni
affette dalla malaria gli eterozigoti sembrano avere un vantaggio sugli omozigoti selvatici. Questo \`e un esempio di sovradominanza: un allele di per s\`e negativo in omozigosi 
diventa vantaggioso nello stato di eterozigosi o vantaggio dell'eterozigote. Il vantaggio \`e legato a un'interazione con l'ambiente in quanto resistenza alla riproduzine del 
plasmodio falciparum. 
\subsection{Un soppressore extragenico pu\`o cancellare gli effetti fenotipici di una mutazione}
In Drosophila esistono due proteine che interagiscono tra di loro: hairless $S$ e un suo soppressore \emph{SoH}. Nel wild type il soppressore sopprime il gene hairless e permette la formazione delle 
setole in Drosophila. Nel caso di mutazioni al soppressore viene modificata la concentrazione relativa di soppressore al gene hairless e l'alterazione dei rapporti causa la scomparsa delle setole.
Si nota come un doppio mutante per $H$ e \emph{SoH} ripristinando il rapporto stechiometrico permette la ricomparsa del fenotipo selvatico.
\subsection{La complementazione - determinare se le mutazioni sono sullo stesso locus o su loci diversi}
Per capire se mutazioni diverse che influenzano una certa caratteristica sono portate dallo stesso locus o da loci diversi si esegue un test di complementazione. Per eseguire il test sulle mutazioni 
recessive vengono incrociati genitori omozigoti per mutazioni differenti che producono una progenie eterogenea. Se le mutazioni sono alleliche allora la prole eterozigote possiede solo alleli mutanti
e mostrer\`a un fenotipo mutante. Se invece le mutazioni sono su loci diversi ciascun genitore \`e omozigote per l'allele recessivo ad un locus ma possiede i geni selvatici sull'altro, pertanto 
la progenie eterozigote eredita un allele mutante e uno selvatico. In questo caso la presenza di un allele selvatico complementa la mutazione a ciascuno dei due loci: la progenie \`e doppio eterozigote
e mostra il fenotipo selvatico. La complementazione si verifica se un individuo che porta due mutazioni recessive mostra un fenotipo wild type, indicando che le mutazioni riguardano geni non allelici. 
\subsection{Drosophila}
In Drosophila sono stati fatti molti esperimenti di complementazione. Si osserva il colore del corpo di colore nero, mentre quello selvatico sul marroncino. Incrociando due 
Drosophila con colorazione atipica si arriva al fenotipo selvatico. Si \`e in complementazione in quanto i fenotipi recessivi sono fenocopie: hanno genetica diversa e complementare:
$ee$ $BB$ $EE$ $bb$. Si ha il corpo nero per omozigosi recessiva di $e$ o $b$, ma si ha un rapporto di dominanza completa all'interno della coppia allelica di $B$ o $E$ e il doppio 
eterozigote ripristina la colorazione selvatica. Si nota come la complementazione \`e intergenica. 
\subsection{Analisi di complementazione}
Nell'analisi di complementazione sono svolti incroci multipli lungo mutanti recessivi per determinare quanti geni contribuiscono al fenotipo. Le mutazioni che falliscono nel 
complementarsi sono dette gruppo di complementazione che, in questo contesto, si riferisce a un gene. 
\section{Influenza del sesso su trasmissione ed espressione dei geni}
Ci si occupa di come l'espressione dei geni sugli autosomi pu\`o essere influenzata dal sesso del genitore da cui i geni sono trasmessi.
\subsection{Caratteri influenzati dal sesso}
I caratteri influenzati dal sesso sono determinati da geni autosomici ereditati secondo le leggi di Mendel ma espressi in modi diversi nei maschi e nelle femmine. Il carattere ha penetranza maggiore in
uno dei sessi, ovvero il numero di cromosomi $X$ influenza la loro espressione. Considerando il rapporto tra $maschi:femmine$ si nota come:
\begin{multicols}{2}
\begin{itemize}
	\item Labbro leporino $2:1$.
	\item Gotta $8:1$.
	\item Artrite reumatoide $1:3$.
	\item Osteoporosi $1:3$.
	\item Lupus eritematoso sistemico $1:9$.
\end{itemize}
\end{multicols}
\subsubsection{Esempi}
\paragraph{Barba nelle capre}
La presenza della barba in alcune capre \`e determinata dal gene autosomico $B^b$ dominante nei maschi e recessivo nelle femmine: nei primi basta un singolo allele $B^b\_$ per la sua espressione, mentre
le femmine necessitano di genotipo $B^bB^b$. 
\subparagraph{Incroci}
Incrociando un maschio senza barba $B^+B^+$ e una femmina con la barba $B^bB^b$ tutti gli individui in $F_1$ sono eterozigoti $B^bB^+$. Essendo il carattere dominante nei maschi e recessivo nelle
femmine tutti i maschi della generazione possiedono la barba, mentre le femmine non la presentano. Reincrociando gli individui di $F_1$:
\begin{multicols}{2}
	\begin{itemize}
		\item $\frac{1}{4}$ \`e $B^bB^b$.
		\item $\frac{1}{2}$ \`e $B^bB^+$.
		\item $\frac{1}{4}$ \`e $B^+B^+$. 
	\end{itemize}
	\begin{itemize}
		\item $\frac{3}{4}$ dei maschi hanno la barba.
		\item $\frac{1}{4}$ delle femmine hanno la barba.
	\end{itemize}
\end{multicols}
\paragraph{Calvizie}
La calvizie dell'uomo \`e un carattere influenzato dal sesso e funziona in modo del tutto analogo alla barba nelle capre.
\subsection{Caratteri limitati dal sesso}
I caratteri limitati dal sesso sono codificati da geni autosomici che si esprimono solo in un sesso: nell'altro il carattere mostra penetranza zero.
\subsubsection{Esempi}
\paragraph{Tipologia di piumaggio nel gallo}
Nei polli domestici alcuni maschi mostrano una tipologia di piumaggio detta piumaggio da gallo, gli altri maschi e tutte le femmine mostrano piumaggio da gallina. Questo \`e un carattere autosomico
recessivo limitato dal sesso dei maschi. Essendo il carattere autosomico i genotipi di maschio e femmina sono gli stessi, ma i fenotipi dono diversi:
\begin{center}
	\begin{tabular}{|c|c|c|}
		\hline
		Genotipo & Fenotipo maschile & Fenotipo femminile \\
		\hline
		$HH$ & Piumaggio da gallina & Piumaggio da gallina \\
		\hline
		$Hh$ & Piumaggio da gallina & Piumaggio da gallina \\
		\hline
		$hh$ & Piumaggio da gallo & Piumaggio da gallina \\
		\hline
	\end{tabular}
\end{center}
\paragraph{Pubert\`a precoce}
Nell'uomo la pubert\`a precoce \`e un carattere limitato dal sesso. Ci son diverse forme e quella limitata al sesso maschile \`e causata da un allele $P$ autosomico dominante che si esprime solo nei 
maschi. Quelli che possiedono tale carattere vanno incontro a una pubert\`a precoce intorno a quattro anni. Non vi \`e alcun danno alla funzionalit\`a sessuale, ma rimangono bassi di statura. I maschi
sono di solito eterozigoti $Pp$ e un maschio con pubert\`a precoce che si accoppia con una donna che non ha storia familiare di tale condizione trasmette l'allele a $\frac{1}{2}$ dei figli, ma l'allele
viene espresso solo nei maschi. 
\subsection{Eredit\`a extranucleare - citoplasmatica}
Non tutto il materiale genetico \`e contenuto nel nucleo: alcune caratteristiche sono codificati da geni nel citoplasma. Tali caratteristiche vengono trasmesse attraverso eredit\`a citoplasmatica. I
mitocondri e cloroplasti contengono DNA e codificano per caratteristiche importanti del fenotipo. Uno zigote riceve gli organelli citoplasmatici da un solo genitore, per lo pi\`u dall'uovo. In tal
modo i caratteri citoplasmatici sono presenti sia nei maschi che nelle femmine ma vengono trasmessi alla progenie dalla madre. Essendo che i mitocondri segregano casualmente nelle cellule della
progenie figli diversi della stessa madre o cellule diverse in un individuo della progenie possono risultare variabili nei fenotipi. Pertanto in certi individui si trova una condizione di 
omoplasmia mentre in altri di eteroplasmia. 
\subsubsection{Esempi}
\paragraph{Eredit\`a citoplasmatica nella pianta bella di notte}
Foglie e germogli di una variet\`a di bella di notte erano screziate e mostravano una mescolanza di macchie verdi e bianche. Inoltre alcuni rami avevano foglie tutte verdi, mentre altre tutte bianche. 
Incrociando fiori provenienti da rami screziati:
\begin{itemize}
	\item Semi di rami verdi producono sempre progenie verde.
	\item Semi di rami bianchi producono una progenie bianca
	\item Semi di rami screziatiproducono progenie verdi, bianche o screziate in rapporti variabili. 
\end{itemize}
Si dimostra pertanto che la screziatura \`e un caso di eredit\`a citoplasmatica: i fenotipi ella progenie erano dipendenti interamente dal genitore materno. In queste piante infatti la 
screziatura \`e causata da un gene difettoso nel cpDNA (DNA del cloroplasto) che inibisce la produzione del pigmento verde della clorofilla. 
\paragraph{Dimostrazione dell'origine materna dell'eredit\`a citoplasmatica}
In questo esperimento si utilizzano mutanti poki di Neurospora crassa. Il sistema sperimentale presenta un individuo con un gene mutante nucleare che viene coniugato con un gene che cambia la dimensione
delle spore aploidi prodotte dalla meiosi: un mutante produce una macrospora. Incrociando una Neurospora poki che produce spore grandi e una con spore di dimensione normale si ottiene un diploide dove
il contributo citoplasmatico \`e esclusivo della macrospora. Si nota come il biomarker, una mutazione del fenotipo citoplasmatico legato a una variazione del citocromo \`e dettato dalla natura della
macrospora utilizzata nella fecondazione. 
\paragraph{Malattie mitocondriali}
Sono state identificate malattie nell'uomo che mostrano un ereditariet\`a mitocondriale. Sono dovute a mutazioni del mtDNA: ha origine nei geni che codificano per i componenti della catena di trasporto
degli elettroni che produce la maggior parte del \emph{ATP}.
\subparagraph{Neuropatia ottica ereditaria di Leber}
\emph{LHON} causa ai pazienti che ne soffrono perdita della vista in entrambi gli occhi a causa della morte delle cellule del nervo ottico. La perdita della vista si verifica tra i $20$ e i $24$ anni, 
ma pu\`o aver luogo dopo l'adolescenza. Si nota una notevole variabilit\`a nella gravit\`a della malattia, anche all'interno della stessa famiglia.
\subsection{Effetto genetico materno}
L'effetto genetico materno avviene quando il fenotipo della progenie viene determinato dal genotipo della madre: i geni vengono trasmessi da entrambi i genitori ma il fenotipo non \`e determinato dal
genotipo della progenie. Si nota quando sostanze presenti nel citoplasma di una cellula uovo hanno un ruolo decisivo nelle fasi precoci dello sviluppo.
\subsubsection{Esempi}
\paragraph{Avvolgimento della conchiglia di Lymnaea peregra}
Le conchiglie possono avere un avvolgimento determinato da una coppia di alleli: destrorso $s^+$ dominante su quello sinistrorso $s$. Gli avvolgimenti sono determinati dal genotipo della madre: l'andamento
dipende dalla modalit\`a con cui il citoplasma si divide dopo la fecondazione direzionato da una sostanza prodotta dalla madre e trasmessa alla prole nel citoplasma dell'uovo.
\subparagraph{Incroci}
Se un maschio omozigote per gli alleli destrorsi $s^+s^+$ viene incrociato con una femmina omozigote per gli alleli sinistrorsi $ss$ tutti gli individui $F_1$ sono eterozigoti ma hanno la conchiglia
sinistrorsa secondo il genotipo della madre. Auto-fecondando $F_1$ $F_2$ ha rapporto fenotipico $1 s^+s^+:2s^+s:1ss$, mentre il fenotipo risulta verso destra in quanto il genotipo della madre eterozigote
codifica per qesto tipo di avvolgimento. 
\subsection{Imprinting genomico}
Si definisce imprinting genomico il fenomeno in cui avviene espressione differenziata del materiale genetico a seconda che sia stato ereditato dal genitore maschio o femmina. Molti geni oggetto di 
imprinting sono associati alla crescita fetale. Il meccanismo non \`e hciaro si la metilazione del DNA \`e essenziale. 
\subsubsection{Esempi}
\paragraph{Fattore di crescita insulino-simile nell'uomo e nei topi}
Il gene \emph{Igf2} mostra imprinting genomico: la progenie riceve un allele dal madre e uno dal padre: la copia paterna \`e espressa nel feto e nella placenta, mentre quella materna rimane silente.
Nella progenie sia i lmaschio che la femmina possiedono i geni, se il gene viene espresso dipende dal sesso del genitore che lo trasmette. Il gene si manifesta solo quando trasmesso dal
genitore di sesso maschile. L'allele paterno promuove la crescita della placenta e del feto. Quando la copia parentale viene eliminata si ottiene una placenta di piccole dimensioni e una progenie di 
basso peso alla nascita. 
\paragraph{Sindrome di Prader-Willi e di Angelman}
\subparagraph{Sindrome di Prader-Willi}
I bambini affetti dalla sindrome di Prader-Willi hanno mani e piedi piccoli, basa statura, basso sviluppo sessuale e ritardo intellettuale. Si nutrono con difficolt\`a alla nascita, ma appena iniziano a 
camminare sviluppano un appetito insaziabile. In molti soggetti si verifica la perdita di una regione sul braccio lungo del cromosoma $15$. 
\subparagraph{Sindrome di Angelman}
La delezione della regione deve per\`o essere ereditata dal
padre. La delezione della medesima regione pu\`o essere ereditata dalla madre, ma determina un insieme di sindromi diversi e causa la sindrome di Agelman: i sintomi sono accessi di riso, movimenti 
muscolari incontrollati, bocca di grandi dimensioni e convulsioni sporadiche.
\subparagraph{Conclusioni}
Affinch\`e lo sviluppo sia normale sembra che sia necessario ricevere la regione del cromosoma $15$ da entrambi i genitori.
\section{Anticipazione}
Un fenomeno genetico non spiegato dalle leggi di Mendel \`e l'anticipazione: un carattere genetico si manifesta in modo pi\`u intenso o in et\`a pi\`u precoce mano a mano che viene trasmesso da una 
generazione all'altra. Spesso \`e dovuta a regioni instabili del DNA che possono allungarsi durante la trasmissione del gene da una generazione all'altra: la gravit\`a e precocit\`a della comparsa
sono dovute alla lunghezza della regione instabile. 
\section{Fattori ambientali}
Essendo il fenotipo il risultato di un genotipo che si sviluppa in un ambiente: ogni genotipo pu\`o produrre diversi fenotipi a seconda delle condizioni ambientali in cui si trova. Si notano alleli
sensibili alla temperatura o termosensibili che si attivano a determinate temperature. 
\subsection{Effetti ambientali sul fenotipo}
\subsubsection{Esempi}
\paragraph{Colorazione della pelliccia dei conigli}
L'allele himalaiano nei conigli produce una pelliccia scura nelle parti terminali del corpo. L'allele si sviluppa per\`o solo quando il coniglio \`e allevato a una temperatura inferiore ai $25\si{\degree}$
altrimenti non sviluppa le chiazze.
\paragraph{Occhio di drosophila e ali vestigiali}
Il numero delle unit\`a ottiche o omatidi di Drosophila cambia in base alla temperatura di sviluppo della larva: a $15\si{\degree}$ anche pi\`u di mille, ma a $30\si{\degree}$ $250$ in meno. Un
meccanismo analogo avviene per le ali: a $18\si{\degree}$ si presentano molto piccole, mentre a temperature elevate normali. La temperatura ha anche un ruolo minore nella femmina. Inoltre un 
mutante shibire \`e vitale tra i $18\si{\degree}$ e i $28\si{\degree}$, ma a temperature pi\`u elevate si ha paralisi reversibile fino alla morte. Il mutante \`e un gene che codifica una proteina
essenziale per la trasmissione nervosa. 
\paragraph{Fenilchetonuria}
La malattia \`e dovuta a una mancanza di un enzima per il metabolismo della fenilanalina che quando si accumula pu\`o portare a danni celebrali. Adottando una dieta a basso contenuto di fenilanalina
si previene il ritardo mentale. 
\subsection{Eredit\`a delle caratteristiche continue}
Si intende per caratteristiche continue caratteristiche che presentano una distribuzione continua dei fenotipi. In quanto devono essere descritte in termini quantitativi si dicono anche caratteristiche
quantitative. Hanno spesso origine in quanto i fenotipi sono il risultato dell'interazione di molti geni o caratteristiche poligeniche.
\subsection{Pleiotropia}
Si dice pleiotropia il fenomeno per cui un singolo gene determina un certo numero di caratteri apparentemente non correlati.
\subsubsection{Esempi}
\paragraph{Singed - mutante di Drosophila}
L'allele singed \`e un omologo della fascina coinvolto nella regolazione della formazione di alcuni elementi del citoscheletro. 
\subparagraph{Effetti fenotipici}
\`E considerato un esempio di 
pleiotropia in quanto sue mutazioni hanno un effetto visibile sulla formazione di una caratteristica somatica esterna, la formazione e il numero e aspetto delle setole, ma i mutanti 
singed oltre all'alterazione nella formazione delle setole sono poco fertili e le uova non sembrano essere in grado di portare alla formazione di zigoti con sviluppo embrionale 
selvatico. 
\subparagraph{Genetica molecolare}
Incroci e studi di complementazione puntano al fatto che si tratta dello stesso gene e dello stesso allele. Per ricondurre questi due
fenotipi alla funzione di un solo allele difettivo. La spiegazione la si trova nel ruolo funzionale del gene singed omologo della fascina che porta alla formazione di fasci di actina
che controllano struttura del citoscheletro: i bundle di actina sono importanti per organizzazione e formazione delle setole a livello somatico ma questa stessa funzione \`e importante
anche nelle camere importanti per la maturazione dell'uovo: la stessa funzione genica si esplica in contesti diversi grazie a un unico meccanismo molecolare in cui le cellule 
organizzano strutture utili per il movimento in cellule specializzate. Questo \`e un fenomeno di pleiotropia: una mutazione e due fenotipi apparentemente non facilmente collegabili. 
\paragraph{Anemia falciforme}
Si nota come la mutazione nel gene della beta globina in posizione codificante modificando un acido glutammico in valina in posizione $6$ causa
l'anemia falciforme. L'ossigeno viene trasportato in modo aberrante con difetto di apporto di ossigeno, legando un evento genetico con difetto proteico con conseguenza legata a tale
difetto. Si nota come il difetto porta a diversi fenotipi: variano da difetti di crescita, a cognitivi, dolore, deformazione nelle ossa, ittero, problemi renali. Questi fenotipi 
apparentemente non collegati sono collegabili e andando a ritroso si nota come a partire dal difetto genetico e alterazione del trasporto dell'ossigeno si arriva ad avere difetti a 
livello epatico, cistifellea, cuore, ossa.
\subsubsection{Pleiotropia antagonistica}
Notando l'et\`a sull'asse $x$ e la forza di selezione sull'asse $y$ si ha una riduzione della forza di selezione che parte molto alta e scende gradatamente fino a diventare zero. 
Questo grafico indica la pleiotropia antagonistica. La forza di selezione cambia con l'et\`a: tra i $15$ e i $35$ anni. Questo vuole provare a convincere che l'effetto di pressione 
selettiva di alleli \`e legato a quando e come gli alleli hanno un impatto sulla trasmissione successiva. Se l'allele lo dimostra precocemente con effetto fenotipico importante 
l'allele ha un effetto forte sulla selezione impedendo che la possibilit\`a abbia una progenie azzerando la trasmissione. Se l'effetto si manifesta successivamente con l'et\`a diminuisce
tale forza di selezione. Alleli con effetti pleiotropici e conseguenza fenotipica antagonistica in funzione con l'et\`a: alleli con successo produttivo in giovinezza ma che 
sono associati a un pi\`u rapido deterioramento in tarda et\`a potrebbero essere selezionati e trasmessi nella popolazione. 
\subsection{Probabilit\`a}
Eventi genetici casuali possono distanziare genotipo e fenotipo con una serie di eventi di modulazione.
\subsubsection{Esempi}
\paragraph{Sindrome di predisposizione al cancro a penetranza incompleta}
Il passaggio da una cellula normale a una tumorale aggressiva richiede diversi eventi che possono partire da una presenza di eventi germinali ereditati che per\`o non sono sufficienti per la conversione
fenotipica: si devono acquisire eventi somatici, mutazioni che avvengono nella linea somatica in maniera stocastica. Pertanto eventi casuali influenzano l'evoluzione fenotipica.
\subparagraph{Processo di trasformazione}
Le cellule del tessuto epiteliale sono organizzate in maniera ordinata, polarizzata e ancorata a formare una barriera. Pu\`o avvenire casualmente una lesione pre-cancerosa a causa dello stress di 
replicazione che porta a danno al DNA e attiva il pathway di riparazione attraverso \emph{ATM}, \emph{ATR} e \emph{CHK} e di \emph{p19ARF}. Questo processo pu\`o portare a mutazioni di geni gatekeeper,
oncogeni o repressori dei tumori chiave come \emph{RAS} e \emph{MYC}. Ora il processo si ramifica: la risposta al danno al DNA attiva \emph{p53} e altri soppressori dei tumori che possono portare 
internamente all'arresto del ciclo cellulare, senescenza e apoptosi o esternamente al legame dcon i recettori del sistema immunitario e loro eliminazione. In altro caso mutazioni e perdita di eterozigosi
nei soppressori chiave dei tumori che regolano il ciclo cellulare causano mutazioni che portano a ottimizzare il fitness di cellule cancerogene maligne. 

