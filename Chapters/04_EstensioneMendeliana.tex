\chapter{Estensione dell'analisi genetica mendeliana}
Se il genetista classico si occupava di analizzare il fenotipo degli organismi che studiava quello moderno si occupa di studiare i meccanismi allelici come la dominanza, la 
recessivit\`a e le loro interazioni come il mascheramento. La cellula verr\`a considerata come una rete globale di interazione genica. Le interazioni tra i geni vengono studiati 
attraverso somiglianze fenotipiche tra i mutanti che permettono di aggregare i geni secondo la loro funzione. Per fare questo uno studio analizza il Saccharomyces cerevisiae costruendo
$23$ milioni di doppi mutanti e identificando \numprint{550000} interazioni positive e \numprint{350000} negative. I geni vengono poi raggruppati per funzione in un metagenoma che
evidenzia come i geni essenziali sono densamente connessi. 
\section{Pathway genetici}
Si nota pertanto come i geni non lavorano in modo isolato: nascono delle interazioni complicate dalle diversit\`a tra gli alleli. Le collaborazioni tra i geni che portano alla
creazione dei pathway possono essere di tipi diversi:
\begin{itemize}
	\item Pathway biosintetici: geni che producono pathway di enzimi che portano alla sintesi di un composto molecolare.
	\item Pathway di trasduzione del segnale: geni che apportano piccole modifiche a molecole permettendo il loro trasporto e localizzazione creando vie di segnalazione.
	\item Pathway di sviluppo: geni che influenzano aspetti di crescita e differenziamento di parti del corpo o strutture intercellulari.
\end{itemize}
\section{Interazioni tra alleli}
L'estensione naturale dell'analisi genetica mendeliana \`e considerare che dato un certo locus non \`e detto che vi si trova un solo gene. Inoltre \`e naturale come i geni non 
lavorano in modo isolato e per capire in un fenotipo si devono considerare pi\`u geni. I geni inoltre si confrontano con l'ambiente cellulare, extracellulare ed esterno. Questi tre
fattori devono essere tutti considerati quando si studiano i geni. Gli esperimenti di Mendel usavano alleli che mostravano dominanza completa, ma esistono altre interazioni.
\subsection{Dominanza incompleta}
In eterozigosi si trova un fenotipo intermedio rispetto a quello dell'omozigote dominante e recessivo: si ha la possibilit\`a di vedere fenotipicamente l'eterozigote.  Avviene in quanto
si ha per alleli diversi una diversa quantit\`a di prodotto. Ogni allele conta nel contribuire a un tratto fenotipico, importante nella genetica multifattoriale per mappare il contributo
quantitativo di alleli sparsi nel genoma  Nella dominanza incompleta ci si aspettano gradazioni intermedie ai due estremi di omozigosi dominante e omozigosi dell'allele meno funzionale. 
\subsubsection{Esempi}
\begin{itemize}
	\item In alcuni fiori il dominante puro ha fenotipo rosso il recessivo puro bianco, mentre l'eterozigote \`e rosa.
	\item Dimensione del pomodoro.
	\item Colore del piumaggio di polli e galline.
\end{itemize}
Si noti come le differenze nei sistemi di modello indicano come probabilmente si tratti di un principio universale. 
\subsection{Codominanza}
La codominanza avviene quando due alleli mostrano contemporaneamente la loro presenza. A differenza della dominanza incompleta non si produce un fenotipo intermedio ma l'effetto di 
ogni gene \`e discreto e in codominanza sono presenti entrambe le caratteristiche intermedie.
\subsubsection{Esempi}
\begin{itemize}
	\item Lenticchie con chiazze e puntini, in eterozigosi si mostrano sia chiazze che puntini. 
	\item Gruppi sanguigni nel modello $AB0$: 
		\begin{itemize}
			\item Omozigosi recessiva $ii$: non si hanno antigeni e si producono anticorpi $A$ e $B$.
			\item Eterozigosi $I^Ai$ ed omozigosi $I^AI^A$: si possiede l'antigene $A$ e si producono anticorpi $B$.
			\item Eterozigosi $I^Bi$ ed omozigosi $I^BI^B$: si possiede l'antigene $B$ e si producono anticorpi $A$.
			\item Eterozigosi $I^BI^A$: si possiedono gli antigeni $A$ e $B$ e non si producono anticorpi.
		\end{itemize}
	\item Sempre nel modello $MN$ per il sangue succeda la stessa cosa. 
\end{itemize}
\subsection{Allelismo multiplo}
Nell'allelismo multiplo i geni esistono in pi\`u di due forme alleliche e ogni individuo pu\`o possedere solo due alleli diversi di uno stesso gene. In questo caso si distinguono
tre categorie tra le quali esiste un ordine di dominanza:
\begin{itemize}
	\item Alleli amorfici o recessivi non funzionanti o nulli.
	\item Alleli parzialmente funzionanti o ipomorfici recessivi rispetto a quelli pi\`u funzionali.
	\item Alleli funzionanti o dominanti rispetto agli altri.
\end{itemize}
\subsubsection{Esempi}
\paragraph{Conigli}
Considerando i conigli come organismi modello e incrociandoli si nota la distribuzione di fenotipi e pertanto le caratteristiche degli alleli. In un certo cromosoma si trova un locus
con un'informazione genetica. Di questo gene ne esistono diverse forme che si caratterizzano come selvatica (associata con una colorazione marrone del pelo). Nella popolazione ci sono 
quattro forme di questo gene o alleli $C$ diversi con mutazioni diverse in punti specifici. Un coniglio albino \`e omozigote $cc$ di un allele non funzionante con assenza di 
pigmentazione. Un coniglio himalaiano \`e albino tranne che nella parte del muso, una porzione delle orecchie e nelle zampe $c^hc^h$, omozigote ma variante allelica di Agouti 
(selvatico). Il cincill\`a presenta chiazze grigie, con alleli e fenotipi diversi. Per capire che sono alleli multipli con interazioni l'uno con l'altro si fanno incroci. Con gli incroci
si nota che l'albino \`e un allele recessivo: deve essere omozigote $cc$ ma l'himalaiano \`e dominante sull'allele $c$ albino: esiste un himalaiano $c^hc$. Questa cosa \`e estensibile: 
il cincill\`a pu\`o essere eterozigote di $c^{ch}c^h$ e presentare codominanza. Il cincill\`a chiaro pu\`o essere eterozigote con albino $c^{ch}c$. Il selvatico oltre a essere omozigote 
pu\`o essere eterozigote con gli altri e presenta dominanza completa. Nella serie allelica ci sono relazioni complesse di dominanza e $C$ domina tutti gli altri e si trova un'ordine di 
dominanza: $c^+>c^{ch}\ge c^h>c$. 
\paragraph{Eterozigosi composita}
L'eterozigosi composita \`e un esempio di allelismo multiplo: due alleli mutanti si trovano in un individuo e la gravit\`a del fenotipo dipende dall'ordine di ipomorfismo. Inoltre 
possono intervenire anche aspetti ambientali. 
\subsection{Alleli letali}
Si definiscono geni essenziali quei geni che se malfunzionanti portano alla morte anticipata dell'organismo. Quando un allele per tale gene \`e presente in forma mutata e porta
alla comparsa del fenotipo drammatico si dice allele letale. Per un allele letale dominante sia in omo che in eterozigosi si presenta il fenotipo e pertanto tende ad autolimitarsi nella
sua diffusione. Se invece \`e recessivo il fenotipo non si presenta in eterozigosi e pertanto possono rimanare nella popolazione protetti dall'allele selvatico. 
\subsubsection{Esempi}
\paragraph{Gatto di Man}
Il gatto dell'isola di Man in eterozigosi presenta una deformazione della spina dorsale e assenza di coda. In eterozigosi non \`e letale ma riduce l'aspettativa di vita in quanto
ha effetto su altri fenotipi, mentre in omozigosi si ha un aborto spontaneo. 
\paragraph{Topo giallo}
Gli esperimenti sul topo giallo vengono svolti nei primi del $900$ da Morgan. Svolge un incrocio tra topi Agouti e selvatici e altri che presentano una colorazione del pelo gialla per
studiare i principi di Mendel. Prende una popolazione Agout $AA$ e una gialla $AA_y$. Il risultato \`e quello aspettato con $\frac{1}{2}$ della popolazione Agouti e l'altra gialla. 
Incrociando successivamente due topi gialli di questa generazione nota come $\frac{2}{3}$ siano gialli e $\frac{1}{3}$ siano Agouti, che sembrerebbe andare in contraddizione con le
regole mendeliane. La spiegazione \`e che una categoria sia invisibile, si sia formata a livello di gametogenesi e fecondazione ma avviene un aborto spontaneo in quanto l'allele in 
omozigosi \`e letale.
\subparagraph{Letalit\`a di un gene responsabile della colorazione del pelo}
Nel $1993$ si studia la ragione per cui una mutazione di un gene responsabile della colorazione del pelo porta alla morte dell'embrione se in omozigosi. Il gene si trova nel cromosoma
$2$ e ha come vicino il gene Raly, che ha tra i trascritti uno per la produzione di una proteina responsabile della corretta espressione genica e legante RNA. La mutazione del topo 
giallo non \`e puntiforme ma \`e una delezione del gene Raly, dello spazio intergenico e del promotore per il gene Agouti. Il colore giallo \`e dovuto a una sovraespressione del gene,
ora controllato dal promotore di Raly e la letalit\`a in omozigosi \`e dovuta alla mancanza della proteina che porta a un deragliamento dello sviluppo embrionale. 
\subsubsection{Alleli letali umani}
Esempi di alleli letali per l'uomo sono la sindrome di Tay-Sachs, la fibrosi cistica e la fenilchetonuria (\emph{PKU}), non tutti portano a un aborto ma causano un effetto visibile in
et\`a infantile riducendo l'aspettativa di vita. La qualit\`a della vita delle persone affette \`e migliorabile grazie alle interazioni dei geni con l'ambiente in modo da modulare
con esse la letalit\`a. 
\paragraph{Sindrome di Tay-Sachs}
La sindrome di Tay-Sachs \emph{TDS} \`e causata da mutazioni in omozigosi o eterozigosi composta della subunit\`a $\alpha$ del gene esosaminidasi $A$ \emph{HEXA} sul cromosoma $15q23$
\`E un disordine autosomale recessivo e progressivo neurodegenerativo che nella forma infantile \`e fatale nei primi $3$ anni di vita. \`E caratterizzato da un insieme di ritardi
nello sviluppo infantile seguito da paralisi, demenza e cecit\`a fino alla morte. Si riconosce grazie a un'area grigio biancastra intorno alla fovea centralis dell'occhio a causa di 
cellule gangliali ricche di lipidi. Una verifica patologica \`e fornita da neuroni a forma di pallone nel sistema nervoso centrale. \`E utile per un riconoscimento una precoce e 
persistente estensione della risposta ai suoni. 
\subparagraph{Genetica molecolare}
La lesione pi\`u frequente nella sindrome \`e un'inserzione di $4$ coppie di basi nell'esone $11$ del gene \emph{HEXA}. Il gene responsabile per la forma giovanile \`e allelico a quello
responsabile per la forma infantile e questi pazienti con la deficienza parziale sopravvivono fino ai $15$ anni. Il gene \emph{HEXA} codifica per la subunit\`a $\alpha$ dell'enzima
lisosomale $\beta$-esosaminidasi che insime al cofattore \emph{GM2} catalizza la degradazione del ganglioside \emph{GM2} e altre molecole contenenti terminale $N$-acetil esosaminasi. 
Mutazioni nella subunit\`a $\alpha$ o $\beta$ portano ad un accumulo del ganglioside \emph{GM2} nei neuroni e disordini neurodegenerativi detti \emph{GM2 gangliosidosi}. La malattia 
\`e causata da $78$ diverse mutazioni semplici, inserzioni o delezioni.
\paragraph{Interazioni geni-ambiente nella \emph{PKU}}
Le cellule viventi e gli organismi complessi hanno la necessit\`a di interagire con l'ambiente per acquisire materia causando una pressione evolutiva per fenomeni metabolici che
prendono le macromolecole dall'esterno per produrre energia o altre macromolecole. Si nota come possano esistere errori congeniti nel metabolismo, ovvero nei complessi pathway 
metabolici ci possono essere mutazioni di geni codificanti enzimi coinvolti in esso che portano ad un blocco del cammino. Un esempio di questo \`e la fenilchetonuria \emph{PKU}, 
che porta a una scorretta elaborazione della fenilanalina introdotta dalla dieta. Il fenotipo della malattia viene influenzato da vari fattori come la quantit\`a di fenilanalina
introdotta con la dieta (necessaria in quanto amminoacido essenziale), le mutazioni del gene per la fenilanalina idrossilasi (che la trasforma in tirosina), la produzione del cofattore 
necessario al suo funzionamento tetraidrobiopterina \emph{TBH}, i livelli di produzione di acido fenilpiruvico alla fine del pathway che viene trasportato fuori dal fegato nel sangue, 
la sua interazione con la barriera ematoencefalica e infine la gestione del suo accumulo nelle cellule nervose. Si nota come la complessit\`a delle interazioni con l'ambiente pu\`o 
rendere molto difficile il riconoscimento dei principi mendeliani attraverso lo studio degli effetti fenotipici, il determinismo genetico \`e reso pi\`u complesso. 
\subsection{Letalit\`a sintetica}
Due geni o due proteine vengono definiti letali sintetici quando la deficienza nell'espressione di uno dei due non compromette la vitalit\`a, mentre la contemporanea alterazione in 
entrambi \`e incompatibile con la sopravvivenza cellulare. Considerando due oggetti $A$ e $B$ che interagiscono tra di loro e con il DNA localizzandone una porzione creando 
un'interazione con una certa costante di DNA. Si indica con $A^+$ e $B^+$ la loro forma funzionale e con $A^-$ e $B^-$ la loro forma non funzionale. Nel caso della formazione del 
complesso $A^+B^-$ o $A^-B^+$ il complesso, nonostante la forma modificata \`e ancora in grado di legare il DNA e svolgere la sua funzione con costante di affinit\`a minore ma comunque
abbastanza per impedire la nascita del fenotipo drammatico. Nel caso in cui invece entrambi siano mutati e si formi il fenotipo $A^-B^-$ il complesso perde la sua affinit\`a con il 
DNA causando la comparsa del fenotipo drammatico dovuto alla sua perdita di funzionalit\`a. Questo concetto viene utilizzato per terapie antitumorali: i tumori possono presentare una
mutazione somatica con un difetto in uno degli elementi del complesso e per far perdere ad esso la funzione essenziale si utilizza un farmaco in grado di colpire il partner generando
la situazione letale nelle cellule tumorali. La tossicit\`a per le cellule non tumorali viene mantenuta bassa in quanto in esse non \`e presente la mutazione del partner. 
\section{Interazioni geniche e rapporti mendeliani modificati}
Si considera oltre a diversi alleli per un singolo gene come pi\`u geni possono influenzare lo stesso fenotipo. Ci saranno pertanto pi\`u coppie diverse di alleli su geni diversi che
convergono sulle stesse caratteristiche. 
\subsection{Esempi}
\subsubsection{Polli}
Si prende in considerazione la forma della cresta dei polli. Si consideri una generazione parentale in cui un individuo possiede una cresta frastagliata o Wyandotte e l'altro a fagiolo
o Brahma. Tentando di studiare la base genotipica per la differenza si incrociano gli individui. Si nota come nella $F_1$ tutti gli individui possiedano un fenotipo con cresta a noce. 
Incrociando ancora si ottiene una $F_2$ in cui si osservano quattro fenotipi:
\begin{itemize}
	\item Wyandotte $frac{3}{16}$.
	\item Brahma $frac{3}{16}$.
	\item Noce $frac{9}{16}$.
	\item A cresta singola o Longhorn $frac{1}{16}$.
\end{itemize}
I rapporti tra i fenotipi sembrano indicare un rapporto mendeliano e come la morfogenesi della cresta sia influenzata da due morfogeni. Chiamando i due geni $Rr$ e $Pp$ si determina il
fenotipo:
\begin{itemize}
	\item Wyandotte $RRpp$.
	\item Brahma $rrPP$.
	\item Noce $R-P-$.
	\item Longhorn $rrpp$.
\end{itemize}
L'assenza dei geni porta a un programma di base, mentre la presenza di uno dei due porta a situazioni intermedie tra entrambi presenti e entrambi assenti. 
\subsubsection{Peperoni}
Si consideri una popolazione parentale con un peperone rosso e verde provenienti dalla stessa pianta (si possono incrociare e ottenere una progenie fertile). La generazione $P$ \`e 
rosso con verde e la $F_1$ presenta peperoni tutti rossi. Autofecondando la $F_1$ si ottiene una $F_2$ in cui compaiono due nuovi fenotipi:
\begin{itemize}
	\item Marrone $frac{3}{16}$.
	\item Giallo $frac{3}{16}$.
	\item Rosso $frac{9}{16}$.
	\item Verde $frac{1}{16}$.
\end{itemize}
La numerologia esclude l'allelismo multiplo. I rapporti tra i fenotipi sembrano indicare un rapporto mendeliano e come il colore del peperone sia influenzato da due geni. Chiamando i due
geni $Cc$ e $Rr$ si determina il
fenotipo:
\begin{itemize}
	\item Marrone $R-cc$.
	\item Giallo $rrC-$.
	\item Rosso $R-C-$.
	\item Verde $rrcc$.
\end{itemize}
La $F_1$ \`e pertanto un'eterozigote da ambo le parti e il fenotipo \`e l'interazione di quattro alleli in due geni con dominanza completa. Quando entrambi recessivi non si produce
pigmento e rimane il colore dato dalla clorofilla.
\subsubsection{Lenticchie}
Si consideri un altro incrocio che conferma lo stesso principio. Si consideri una popolazione di lenticchie composta da individui con colore marrone-rosso $AAbb$ e con colore grigio 
$aaBB$. \`E un modello a due geni e la popolazione parentale \`e omozigote trans per entrambi. Nella $F_1$ tutti gli individui sono eterozigoti con entrambi associato con il fenotipo 
marrone. Incrociando $F_1$ si ottiene $F_2$ e si studia il fenotipo:
\begin{itemize}
	\item Marrone $frac{9}{16}$, $A-B-$.
	\item Marrone-rosso $frac{3}{16}$, $A-bb$.
	\item Grigio $frac{3}{16}$, $aaB-$.
	\item Verde $frac{1}{16}$, $aabb$.
\end{itemize}
\subsubsection{Drosophila melanogaster}
Si consideri la pigmentazione dell'occhio composito della Drosophila: questo si presenta o rosso o bianco. Incrociando i due individui si nota come in $F_1$ tutta la progenie abbia
occhi rossi. Incrociando un individuo di $F_1$ $Bw^+BwSt^+St$ e si incrocia con un individuo con gli occhi bianchi $BwBwStSt$. Si ottiene una $F_2$ in cui si notano nuovi fenotipi. 
\begin{itemize}
	\item Occhio rosso $frac{1}{4}$, $Bw^+St^+$.
	\item Occhio scarlatto $frac{1}{4}$, $Bw^+St$.
	\item Occhio marrone $frac{1}{4}$, $BwSt^+$.
	\item Occhio bianco $frac{1}{4}$, $BwSt$.
\end{itemize}
Si nota come il rapporto \`e di $\frac{1}{4}$ si lavora con due geni e quattro alleli e il fenotipo \`e guidato da un omozigosi di entrambi e nel fenotipo \`e guidato dal fenotipo 
dell'individuo doppio eterozigote che quando incrociato con un doppio recessivo. 
\subsection{Fenocopia}
Si intende per fenocopia geni diversi che concorrono allo stesso fenotipo.
\subsubsection{Esempi}
\paragraph{Colorazione dell'occhio di Drosophila}
Considerando l'esempio precedente sulla colorazione dell'occhio di Drosophila melanogaster si nota come oltre ai geni precedenti che producono il pigmento, detti $b$ e $v$, ne esiste
un altro che controlla il suo trasporto all'occhio. Questo gene esiste in due forme alleliche: quella dominante o funzionale $w^+$ che mostra il fenotipo del colore prodotto dai primi
due geni e quella recessiva o non funzionale $w$ che mostra un fenotipo di colore bianco. Si nota pertanto come i geni concorrono allo stesso fenotipo e si dicono pertanto fenocopie. 
\section{Epistasi}
Si intende per epistasi un esempio di interazione tra geni in base al quale un gene interferisce con l'espressione fenotipica di un altro gene non allelico per cui il fenotipo 
viene determinato dal primo gene e non dal secondo pur essendo rappresentati entrambi in un genotipo. 



\subsection{Epistassi recessiva doppia}
Nell'epistassi recessiva doppia 
\subsubsection{Esempi}
\paragraph{Fiori}
Si considerino due fiori con fenotipo recessivo bianco e nell'ipotesi pi\`u semplice mendeliana, una generazione parentale bianca indica un'omozigosi recessiva per l'allele responsabile
del fenotipo. L'incrocio dovrebbe creare solo figli bianchi. Si nota come invece un'incrocio tra fiori bianchi in $F_1$ si ottiene una popolazione con colorazione selvatica viola. 
Questo \`e un fenomeno di complementazione: due genomi incrociati con fenotipo recessivo porta alla comparsa del fenotipo dominante. Si pu\`o interpretare proponendo che i due recessivi
siano fenocopie, con genetica complementare: un fiore \`e $C/C$ e $p/p$, mentre il partner \`e $c/c$ e $P/P$. Accettando questa ipotesi si nota come in $F_1$ tutta la popolazione
sia eterozigota doppia: la presenza di un allele dominante assicura la presenza di un fenotipo selvatico di un fiore colorato. Incrociando ora $F_1\times F_1$ i risultati seguono la
regola dell'incrocio di-ibrido con assortimento indipendente. Scomponendo l'incrocio di-ibrido si attende in $F_2$:
\begin{itemize}
	\item $\frac{3}{4}$ $C/-$:
		\begin{itemize}
			\item $\frac{3}{4}$ $P/-$: $\frac{9}{16}$ $C/-\ P/-$ selvatico
			\item $\frac{1}{4}$ $p/p$: $\frac{3}{16}$ $C/-\ p/p$ bianco.
		\end{itemize}
	\item $\frac{1}{4}$ $c/c$:
		\begin{itemize}
			\item $\frac{3}{4}$ $P/-$: $\frac{3}{16}$ $c/c\ P/-$ bianco.
			\item $\frac{1}{4}$ $p/p$: $\frac{1}{16}$ $c/c\ p/p$ bianco.
		\end{itemize}
\end{itemize}
Osservando le classi fenotipiche si nota come queste stanno in rapporto $9:7$. Questo a causa dell'interazione funzionale di geni diversi. Interpretando questo fenomeno come un 
mascheramento in cui un gene maschera un fenotipo dell'altro si pu\`o determinare qual \`e il gene che maschera l'altro: non \`e facie capire qual \`e che viene mascherato in quanto 
ci si trova in un caso di epistasi recessiva doppia. Senza una comprensione molecolare del percorso biosintetico non si \`e in grado di decidere quale gene determina sull'altro. 
\subparagraph{Pathway biosintetico}
Si nota come nella via biosintetico ci sia un precursore su cui agisce $C$ che forma un intermedio incolore e un prodotto finale che da il colore sintetizzato da $P$. Con uno dei
due geni non presente il pathway \`e bloccato e si trova assenza di antocianina. In questo caso in $c/c$ la via biosintetica si interrompe con accumulo di precursore. In questa 
interpretazione l'allele $c$ maschera in omozigosi recessiva. Questa si osserva nella condizione di omozigosi di un gene $C$ che essendo a monte della via biosintetica rende il 
genotipo di $P$ irrilevante in quanto l'assenza dell'intermedio porta alla sua incapacit\`a di esprimere la sua funzione. \`E anche doppia in quanto l'assenza di $P$ ha la stessa
conseguenza. L'epistasi recessiva doppia: il gene epistatico \`e $C$ e quello mascherato si dice ipostatico $P$. 
\paragraph{Lumache d'acqua}
Il guscio della conchiglia che pu\`o essere marrone o non colorato \`e legato alla genetica di una via biosintetica di un composto colorato dove vengono identificati due enzimi 
codificati da due geni in cui un precursore viene convertito da un enzima in un intermedio che viene ancora elaborato da un altro che porta alla formazione del pigmento. L'assenza 
del primo gene maschera la funzione del secondo. La combinazione $aabb$ blocca tutto e in questo contesto, nel caso $A$ epistatico sia funzionante e il $b$ in omozigosi 
recessiva si avrebbe un guscio bianco. In $F_2$ con incrocio di-ibrido si avrebbe un rapporto atteso $9:7$.


\subsection{Epistasi recessiva} 
Nell'epistasi recessiva l'omozigote recessivo di un gene maschera gli effetti di entrambi gli alleli di un altro.
\subsubsection{Esempi}
\paragraph{Labrador}
Si consideri il colore del pelo del labrador: 
\begin{itemize}
	\item $B-E$ nero.
	\item $bbE-$ marrone.
	\item $--ee$ biondo.
\end{itemize}
Si incroci ora una generazione parentale nero $BBEE$ con biondo $bbee$. Si nota come si forma una $F_1$ \`e tutta nera con $BbEe$. Con un incrocio stile mendeliano tra $F_1$ eterozigoti
si nota come compare un nuovo colore: nero, marrone e giallo in rapporto $9:3:4$. Si nota come si ha un'interazione tra gli alleli del gene $E$ e $B$. I biondi sono $4$ e uno dei 
geni maschera l'altro. Si nota come il biondo compare con doppia omozigosi recessiva e con tutte le combinazioni possibili dello stato allelico di $B$: $ee$ causa biondo. Questo vuol
dire che omozigosi recessiva di $e$ nasconde la genetica di $B$. Questa \`e un'epistasi recessiva in quanto il mascheramento \`e attuato da un'omozigosi recessiva. $e$ \`e epistatico 
e quello ipostatico \`e $b$. Con epistasi rotta: non omozigosi recessiva per $E$ l'omozigosi $bb$ porta a una colorazione intermedia marrone. I nove casi \`e quella dove con un 
allele dominante per $B$ ed $E$ in assenza di mascheramento e la funzionalit\`a di $B$ raggiunga l'intensit\`a maggiore. Questa \`e un'epistasi recessiva semplice in quanto si 
distingue il mascheramento da un secondo gruppo fenotipico dove il mascheramento non ci sarebbe ma l'omozigosi recessiva del gene ipostatico fa si che il fenotipo si differenzi da quello
selvatico legato alla presenza di un allele dominante per entrambi i geni.
\paragraph{Gruppo sanguigno \emph{AB0}}
Si nota come sulla membrana del globulo pu\`o essere presente uno dei lipidi con una coda che protrude e degli zuccheri. Una struttura zuccherina ancorata alla membrana e presenta
segnali a due colori. Questi corrispondono ad antigeni del sistema AB, serie allelica con tre alleli, uno recessivo $i$, uno $I_A$ e uno $I_B$ dominanti su $i$ e codominanti tra di 
loro. Si considerino due genitori di gruppo sanguigno $0$. Per quello che si sa ha una genetica determinata $ii$. Pertanto si prevede il gruppo sanguigno dei due figli: possono
avere solo $ii$ a meno di mutazioni de novo. Ma in questo caso pu\`o nascere un figlio di gruppo $A$ o $B$. Forse l'errore \`e attribuire ad un fenotipo un genotipo certo. Alla luce
del discorso sull'epistasi un'altra ipotesi \`e che il fenotipo $0$ contiene un genotipo con allele dominante mascherato. Per mascherare $I_A$ o $I_B$ si nota come l'allele $I$ \`e
responsabile della decorazione dei glicolipide di un ultimo zucchero variante di galattosio, differenziandosi tra di loro per piccola variazione: galattosio e galattosammina. Si 
ipotizza che la possibilit\`a di esprimere la funzione dominante \`e legata alla presenza del substrato della presenza della prima decorazione glicolipidica interessata all'attacco
dell'ultimo esamero di zucchero. In caso di genetica che preclude la formazione della prima ancora glicolipidica la funzionalit\`a di $I$ viene resa invisibile. Questa \`e recessiva
in quanto si deve perdere l'espressione dell'ancora. Il gene responsabile della sintesi della coda \`e il gene $H$ e $hh$ pu\`o mascherare la presenza di un allele dominante $I$. 
Questo \`e il fenotipo Bombay ed \`e legato a un gene la cui omozigosi recessiva maschera la funzionalit\`a del gruppo genico \emph{AB0}: $hh$ \`e la situazione epistatica. 

\subsection{Epistasi dominante I}
Nell'epistasi dominante I l'allele dominante di un gene nasconde gli effetti di entrambi gli alleli di un altro.
\subsubsection{Esempio}
\paragraph{Zucchina}
Si nota il colore verde in omozigosi $aabb$ con assenza di vie che producono pigmenti che lo degradano facendo sparire la colorazione e bianco $AABB$. Incrociando questi due individui
si ottengono una $F_1$ $AaBb$ doppia eterozigote. Fenotipicamente si nota l'assenza di colore in $F_1$, pertanto il bianco \`e legato alla presenza di un allele per il gene $A$ e $B$ 
con dominanza completa nelle rispettive coppie alleliche. Incrociando ora $F_1$ con s\`e stessa. Si nota"
\begin{itemize}
	\item $12$: $9$ $A-B-$ bianchi.
	\item $3$: $A-bb$ giallo.
	\item $1$: $aabb$ verde.
\end{itemize}
Si nota come i numeri si assestano a un rapporto $12:3:1$. Il giallo \`e omozigosi di $bb$ e un allele dominante per $A$ mentre tutti gli altri casi sono bianchi. Questo \`e una
deviazione dei rapporti mendeliani che sembra risultare dalla combinazione di due classi fenotipiche e genotipiche previste. Guardando complessivamente le condizioni la presenza
di almeno un allele dominante di $B$ controlla il fenotipo che ha colorazione bianca. $b$ in omozigosi recessiva il fenotipo varia con allele $A$ dominante si ha colore giallo e con
entrambi inattivi si ha verde. Questa pu\`o essere considerata un esempio di epistasi, ma il mascheramento \`e reso possibile dalla presenza di un allele dominante: epistasi dominante
che pu\`o essere rivelata da un rapporto $12:3:1$. Quella pi\`u abbondante riflette il mascheramento. Colorazione legata all'accumulo di un pigmento in quanto non viene prodotto uno
dei due geni che li converte. Per rompere l'epistasi dominante si rende necessario omozigosi recessiva. 
\paragraph{Colore dei fiori foxglove}
Questi fiori possono avere colorazione punteggiata bianca, rossa o rosso leggero. Si ottiene una $F_1$ con incrocio di linee pure doppi eterozigoti per due geni coinvolti per il 
fenotipo di cui si valuta la genetica: $DdWw$. Si autoincrocia $F_1$ e si ha una relazione tra genotipo e fenotipo. Si ha un precursore non colorato convertito dai prodotti del gene 
$D$ che produce un pigmento colorato che deve raggiungere il posto corretto: il gene $W$ si occupa di tale trasporto quando dominante la pigmentazione viene confinata nei throat 
spots, di fatto fiore bianco con macchie rosse. In omozigosi di $dd$ l'assenza di funzione dovuta ad essa porta ad una colorazione light molto pi\`u leggero ma $W$ condiziona il 
trasporto. I conti suggeriscono che i rapporti sono ancora in $12:3:1$. Quando l'epistasi in omozigosi recessiva di $w$ si ha l'effetto del gene $D$: in dominanza il pigmento scuro
diffuso con fiore rosso, con combinazione di omozigosi recessiva di $dd$ si ha una diffusione di un colore pi\`u sbiadito. 
\subsection{Epistasi dominante II}
Nell'epistasi dominante II l'allele dominante di un gene nasconde gli effetti dell'allele dominante di un altro gene.
\subsubsection{Esempi}
\paragraph{Galline livornesi}
Si valuti la pigmentazione delle galline con $P$ livornese bianca $AABB$ e americana colorata $aabb$. Si ottiene una $F_1$ $AaBb$ e il suo auto incrocio porta alla formazione di 
prodotti fenotipici $13:3$: il rapporto \`e complicato da interazioni tra geni. CI sono due fenotipi ma compaiono in proporzioni $13:3$ questo in quanto il colore si ha solo 
per $A-bb$. Il mascheramento \`e operato dalla presenza di un allele dominante $B$, con epistasi dominante e recessiva in quanto in questo caso la rottura dell'epistasi nella 
contemporaneit\`a dell'omozigosi porta anch'essa all'assenza di colore. L'assenza del gene ipostatico copia il fenomeno di epistasi. 
\subsection{Complementariet\`a}
Nella complementetariet\`a per la produzione del fenotipo \`e necessario un allele dominante per ciascuno dei due geni. 

\subsection{Ridondanza genica}
Nella ridondanza genica i rapporti mendeliani sono modificati $15:1$: un fenotipo compare unicamente in doppia omozigosi recessiva. La ridondanza genetica pu\`o dipendere dalla
duplicazione genica. Il gene $A$ e $B$ potrebbero essere geni paraloghi, forse con antenato comune con duplicazione eptopica (su cromosomi diversi) in modo che segreghino in modo
indipendente. In questo caso la distanza dalla nascita dei paraloghi non \`e abbastanza ambia per aver diversificato la loro funzione. Sono ridondanti per la funzione biochimica che
per essere persa si devono perdere $4$ alleli. 
\subsubsection{Esempi}
\paragraph{Capsula della pianta borsa del pastore}
Si consideri che questo oggetto ha una forma ovale e una triangolare. Si possono ottenere linee pure e triangolare \`e $TTVV$, mentre l'altro $ttvv$. Si ottiene una $F_1$ doppia 
eterozigote che si incrocia con s\`e stessa: il fenotipo ovale compare solo in doppia omozigosi recessiva. Questo vuol dire che il gene $T$ e quello $V$ interagiscono sullo stesso
punto del pathway, pertanto per perdere il passaggio morfogenetico si rende necessario disattivare entrambe le coppie.
\subsection{Interazione genetica dominante}
L'interazione genica dominante \`e caratterizzata da un tasso di-ibrido $9:6:1$. 
\subsubsection{Esempi}
\paragraph{Zucche} 
Si consideri la forma a disco o allungata o a sfera e si nota come la deviazione come l'omozigosi recessiva per un gene con dominante per l'altro e il complementare possiedono lo 
stesso fenotipo. 
\subsection{Sovradominanza o vantaggio dell'eterozigote}
Si consideri un globulo rosso con forma normale e antigeni se gene $H$ funzionante e stato allelico \emph{AB0}. Nell'altro si trova una forma a falce legata alla presenza di una
mutazione specifica nel gene della beta-globina. A valle di una meiosi si nota come la frequenza dei portatori e dei casi e la distribuzione della malaria: nei residenti in regioni
affette dalla malaria gli eterozigoti sembrano avere un vantaggio sugli omozigoti selvatici. Questo \`e un esempio di sovradominanza: un allele di per s\`e negativo in omozigosi 
diventa vantaggioso nello stato di eterozigosi o vantaggio dell'eterozigote. Il vantaggio \`e legato a un'interazione con l'ambiente in quanto resistenza alla riproduzine del 
plasmodio falciparum. 
\subsection{Un soppressore extragenico pu\`o cancellare gli effetti fenotipici di una mutazione}
Si osservano due proteine una hairless $S$ e una soppressore di hairless \emph{SoH} che possono interagire: quando l'interazione avviene il soppressore sopprime il gene e permette
che in Drosophila si formino setole. Nel caso in cui la genetica cambi nel caso: nella cellula ci saranno a causa dell'eterozigosi sia interazioni con il soppressore che uno 
squilibrio della stechiometria in quanto hairless \`e ridotta a causa di eterozigosi e questo fa si che ci sia un'alterazione dei rapporti che porta alla formazione delle setole che 
non si formano. Inoltre con un doppio mutante con eterozigosi per $H$ che per \emph{SoH} che ripristina il rapporto stechiometrico che non crea un eccesso che ripresenta il 
fenotipo selvatico. 
\section{Complementazione}
Si era gi\`a notata la complementazione con i fiori bianchi e viola. Considerando fenotipi regessivi con genetica complessa, un incrocio smaschera se un individuo \`e affetto dal 
fenotipo per una regione genetica equivalente o distinta: con complementazione geni diversi, altrimenti difetto sullo stesso gene.
\subsection{Drosophila}
In Drosophila sono stati fatti molti esperimenti di complementazione. Si osserva il colore del corpo di colore nero, mentre quello selvatico sul marroncino. Incrociando due 
Drosophila con colorazione atipico porta al fenotipo selvatico. Si \`e in complementazione in quanto i fenotipi recessivi sono fenocopie: hanno genetica diversa e complementare:
$ee$ $BB$ $EE$ $bb$. SI ha il corpo nero per omozigosi recessiva di $e$ o $b$, ma si ha un rapporto di dominanza completa all'interno della coppia allelica di $B$ o $E$ e il doppio 
eterozigote ripristina la colorazione selvatica. Si nota come la complementazione \`e intergenica. 
\subsection{Analisi di complementazione}
Nell'analisi di complementazione sono svolti incroci multipli lungo mutanti recessivi per detereminare quanti geni contribuiscono al fenotipo. Le mutazioni che falliscono nel 
complementarsi sono dette gruppo di complementazione che, in questo contesto, si riferisce a un gene. 
\section{Lievito gemmante S. cerevisiea}
Si nota come i lieviti possono esistere sia in forma aploide che diploide, nella seconda sono in grado di fare meiosi per produrre spore resistenti ad ambienti difficili. S. 
cereviisae la meiosi dei funghi \`e particolare in quanto i quattro prodotti delle cellule aploidi sono confinati in un asco. Micromanipolando gli aschi si pu\`o seguire la meiosi
in diretta, cosa non possibile nei mammiferi in quanto i gameti si trovano in un pool. Nello sporacrassa fa una meiosi equivalente ma molto ordinato spazialmente: le cellule si 
organizzano in un astuccio compatto e le cellule occupano posizioni specifiche si ha una mitosi dopo la seconda meiosi e le cellule si trovano in un ordine particolare. Appoggiando 
le spore in un terreno di crescita singolarmente e valutando lo stato allelico delle cellule in quanto questi funghi possono vivere come aploidi. Si nota come l'ottade ordinata
consente di dedurre eventi di crossing over dalla disposizione delle spore. SI ha alternanza di alleli dominanti e recessivi in modo ordinato. Con Neurospora crassa un esperimento
fatto da Beadle e Tatum ha a che fare con la possibilit\`a di mappare la genetica di un percorso biosintetico legato alla produzione e sintesi di un amminoacido. In una protevva
con Neurospora si utilizzano raggi X per mutagenizzare e nella zona dei corpi fruttiferi dove avvengono le meiosi. Dopo questo fanno degli incroci, meiosi e quando si arriva 
all'ottade si cerca di fare microdissezione, le si mette in terreno di coltura e osservare se sono vive e che tipo di caratteristiche possono avere. Si interessano alla via che
produce arginina: se sono in grado sno in grado di crescere in assenza di arginina. Si isola un mutante \emph{arg}: molte provette di terreno completo dove si fanno germinare le
spore aploidi a valle del passo precedente. Siccome il terreno \`e completo crescono le spore che non hanno subito mutazioni in geni essenziali. Da questo pool di potenziali mutanti
si trasferiscono gli apolidi in un terreno minimo per notare qualcuno di questi che non riesce pi\`u a crescere. Crescono tutti tranne uno, un mutante nel gene essenziale, hanno un
difetto genetico in quanto non crescono sul terreno minimo. Partendo dalla riserva si vuole capire dove \`e il difetto: si prendono quattro terreni: uno di controllo e tre uno \`e
il terreno minimo  e uno \`e terreno minimo con tutti amminoacidi (crescita) uno con vitamine ma non amminoacidi (non cresce). Mutazione di un gene non essenziale coinvolto nella
produzione di amminoacidi (auxotrofia). Ora si tenta di determinare quale amminoacido non si riesce a sintetizzare. Il mutante per arginina \`e detto \emph{arg1} e ci sono diversi 
mutanti per lo stesso fenotipo. Viene scelto Neurospora crassa in quanto il modello \`e in grado di crescere in laboratorio in tempi relativamente veloci facilitando l'enorme
variabilit\`a dei terreni. Vengono scelte le spore aploidi (non si possono maneggiare i geni essenziali) ma le mutazioni recessive mascherate nell'interazione allelica con la copia
selvatica in un diploide si manifestano in cellule aploidi. Si usano le spore aploidi per rivelare la mutazione recessiva ma il passaggio a diploide aiuta a capire la natura 
molecolare della mutazione.
\subsection{Studio dei mutanti}
Dopo aver trovato i vari mutanti si cerca di ricostruire la loro storia molecolare. Si trovano tre mutanti \emph{arg1}, \emph{arg2} e \emph{arg3} cloni diversi di un processo di 
mutagenesi e selezione. Si fa un test di complementazione per prototrofia: si prendono due aploidi e li si fa accoppiare per ottenere un diploide che unisca i due genomi e si 
cerca di verificare se complementano. Se ci fosse complementazione si riacquista la prototrofia per l'effetto metabolico e il diploide potrebbe crescere su un terreno privo di 
arginina. Pertanto \emph{arg1} e \emph{arg2} sono due mutanti per due geni diversi recessivi che fanno parte di un pathway biosintetico. Trovati i tre mutanti si fanno degli incroci
per valutare la complementazione. Questi complementano $1$ con $2$, $2$ con $3$ e $1$ con $3$: ci sono tre geni essenziali per la sintesi dell'arginina. Successivamente i mutanti 
singoli vengono testati su terreni privi di arginina ma contenente sostanze a lei simili: arginina, ornitina e citrullina che sono abbastanza plausibili come passaggi unici verso 
l'arginina. Nel pathway ornitina -> citrullina -> arginina. Si usano terreni che contenevano o no uno di questi elementi. Si testano i mutanti aploidi originali per vedere come
rispondevano in questi terreni. Un mutante \emph{arg7} \`e in grado di crescere su un terreno contenente ornitina, citrullina e arginina. \emph{arg3} che complementa con \emph{arg3}
non \`e in grado di crescere con terreno di ornitina ma con gli altri. \emph{arg1} non \`e in grado di crescere in terreno senza ornitina o citrullina. Questo suggerisce che
\emph{arg7}, \emph{arg3} e \emph{arg1} sono alleli mutati di geni diversi posizionabili in punto diverso del pathway. Da qui nasce l'ipotesi un gene, un enzima. L'ordine nel 
pathway \`e $7$->$3$->$1$.
\subsubsection{Pathway dell'arginina in S. cerevisiae}
Si parte dal glutammato usando cofattori ed energia chimica e un enzima per ogni passaggio. \emph{arg2} definisce la funzione selvatica per passare dal glutammato 
all'acetil-glutammato. \emph{arg5, 6} gene con doppia funzione enzimatica. Il gene \emph{arg1} converte citrullina in arginino-succinato senza cui l'ultimo enzima non \`e in grado 
di sintetizzare l'arginina. A monte si trova il \emph{arg3} e ancora a monte \emph{arg7}. 
