\chapter{Estensione dell'analisi genetica mendeliana}
Se il genetista classico si occupava di analizzare il fenotipo degli organismi che studiava quello moderno si occupa di studiare i meccanismi allelici come la dominanza, la 
recessivit\`a e le loro interazioni come il mascheramento. La cellula verr\`a considerata come una rete globale di interazione genica. Le interazioni tra i geni vengono studiati 
attraverso somiglianze fenotipiche tra i mutanti che permettono di aggregare i geni secondo la loro funzione. Per fare questo uno studio analizza il Saccharomyces cerevisiae costruendo
$23$ milioni di doppi mutanti e identificando \numprint{550000} interazioni positive e \numprint{350000} negative. I geni vengono poi raggruppati per funzione in un metagenoma che
evidenzia come i geni essenziali sono densamente connessi. 
\section{Interazioni tra alleli}
L'estensione naturale dell'analisi genetica mendeliana \`e considerare che dato un certo locus non \`e detto che vi si trova un solo gene. Inoltre \`e naturale come i geni non 
lavorano in modo isolato e per capire in un fenotipo si devono considerare pi\`u geni. I geni inoltre si confrontano con l'ambiente cellulare, extracellulare ed esterno. Questi tre
fattori devono essere tutti considerati quando si studiano i geni. Gli esperimenti di Mendel usavano alleli che mostravano dominanza completa, ma esistono altre interazioni.
\subsection{Dominanza incompleta}
In eterozigosi si trova un fenotipo intermedio rispetto a quello dell'omozigote dominante e recessivo: si ha la possibilit\`a di vedere fenotipicamente l'eterozigote.  Avviene in quanto
si ha per alleli diversi una diversa quantit\`a di prodotto. Ogni allele conta nel contribuire a un tratto fenotipico, importante nella genetica multifattoriale per mappare il contributo
quantitativo di alleli sparsi nel genoma  Nella dominanza incompleta ci si aspettano gradazioni intermedie ai due estremi di omozigosi dominante e omozigosi dell'allele meno funzionale. 
\subsubsection{Esempi}
\begin{itemize}
	\item In alcuni fiori il dominante puro ha fenotipo rosso il recessivo puro bianco, mentre l'eterozigote \`e rosa.
	\item Dimensione del pomodoro.
	\item Colore del piumaggio di polli e galline.
\end{itemize}
Si noti come le differenze nei sistemi di modello indicano come probabilmente si tratti di un principio universale. 
\subsection{Codominanza}
La codominanza avviene quando due alleli mostrano contemporaneamente la loro presenza. A differenza della dominanza incompleta non si produce un fenotipo intermedio ma l'effetto di 
ogni gene \`e discreto e in codominanza sono presenti entrambe le caratteristiche intermedie.
\subsubsection{Esempi}
\begin{itemize}
	\item Lenticchie con chiazze e puntini, in eterozigosi si mostrano sia chiazze che puntini. 
	\item Gruppi sanguigni nel modello $AB0$: 
		\begin{itemize}
			\item Omozigosi recessiva $ii$: non si hanno antigeni e si producono anticorpi $A$ e $B$.
			\item Eterozigosi $I^Ai$ ed omozigosi $I^AI^A$: si possiede l'antigene $A$ e si producono anticorpi $B$.
			\item Eterozigosi $I^Bi$ ed omozigosi $I^BI^B$: si possiede l'antigene $B$ e si producono anticorpi $A$.
			\item Eterozigosi $I^BI^A$: si possiedono gli antigeni $A$ e $B$ e non si producono anticorpi.
		\end{itemize}
	\item Sempre nel modello $MN$ per il sangue succeda la stessa cosa. 
\end{itemize}
\subsection{Allelismo multiplo}
Nell'allelismo multiplo i geni esistono in pi\`u di due forme alleliche e ogni individuo pu\`o possedere solo due alleli diversi di uno stesso gene. In questo caso si distinguono
tre categorie tra le quali esiste un ordine di dominanza:
\begin{itemize}
	\item Alleli amorfici o recessivi non funzionanti o nulli.
	\item Alleli parzialmente funzionanti o ipomorfici recessivi rispetto a quelli pi\`u funzionali.
	\item Alleli funzionanti o dominanti rispetto agli altri.
\end{itemize}
\subsubsection{Esempi}
\paragraph{Conigli}
Considerando i conigli come organismi modello e incrociandoli si nota la distribuzione di fenotipi e pertanto le caratteristiche degli alleli. In un certo cromosoma si trova un locus
con un'informazione genetica. Di questo gene ne esistono diverse forme che si caratterizzano come selvatica (associata con una colorazione marrone del pelo). Nella popolazione ci sono 
quattro forme di questo gene o alleli $C$ diversi con mutazioni diverse in punti specifici. Un coniglio albino \`e omozigote $cc$ di un allele non funzionante con assenza di 
pigmentazione. Un coniglio himalaiano \`e albino tranne che nella parte del muso, una porzione delle orecchie e nelle zampe $c^hc^h$, omozigote ma variante allelica di Agouti 
(selvatico). Il cincill\`a presenta chiazze grigie, con alleli e fenotipi diversi. Per capire che sono alleli multipli con interazioni l'uno con l'altro si fanno incroci. Con gli incroci
si nota che l'albino \`e un allele recessivo: deve essere omozigote $cc$ ma l'himalaiano \`e dominante sull'allele $c$ albino: esiste un himalaiano $c^hc$. Questa cosa \`e estensibile: 
il cincill\`a pu\`o essere eterozigote di $c^{ch}c^h$ e presentare codominanza. Il cincill\`a chiaro pu\`o essere eterozigote con albino $c^{ch}c$. Il selvatico oltre a essere omozigote 
pu\`o essere eterozigote con gli altri e presenta dominanza completa. Nella serie allelica ci sono relazioni complesse di dominanza e $C$ domina tutti gli altri e si trova un'ordine di 
dominanza: $c^+>c^{ch}\ge c^h>c$. 
\paragraph{Eterozigosi composita}
L'eterozigosi composita \`e un esempio di allelismo multiplo: due alleli mutanti si trovano in un individuo e la gravit\`a del fenotipo dipende dall'ordine di ipomorfismo. Inoltre 
possono intervenire anche aspetti ambientali. 
\subsection{Alleli letali}
Si definiscono geni essenziali quei geni che se malfunzionanti portano alla morte anticipata dell'organismo. Quando un allele per tale gene \`e presente in forma mutata e porta
alla comparsa del fenotipo drammatico si dice allele letale. Per un allele letale dominante sia in omo che in eterozigosi si presenta il fenotipo e pertanto tende ad autolimitarsi nella
sua diffusione. Se invece \`e recessivo il fenotipo non si presenta in eterozigosi e pertanto possono rimanare nella popolazione protetti dall'allele selvatico. 
\subsubsection{Esempi}
\paragraph{Gatto di Man}
Il gatto dell'isola di Man in eterozigosi presenta una deformazione della spina dorsale e assenza di coda. In eterozigosi non \`e letale ma riduce l'aspettativa di vita in quanto
ha effetto su altri fenotipi, mentre in omozigosi si ha un aborto spontaneo. 
\paragraph{Topo giallo}
Gli esperimenti sul topo giallo vengono svolti nei primi del $900$ da Morgan. Svolge un incrocio tra topi Agouti e selvatici e altri che presentano una colorazione del pelo gialla per
studiare i principi di Mendel. Prende una popolazione Agout $AA$ e una gialla $AA_y$. Il risultato \`e quello aspettato con $\frac{1}{2}$ della popolazione Agouti e l'altra gialla. 
Incrociando successivamente due topi gialli di questa generazione nota come $\frac{2}{3}$ siano gialli e $\frac{1}{3}$ siano Agouti, che sembrerebbe andare in contraddizione con le
regole mendeliane. La spiegazione \`e che una categoria sia invisibile, si sia formata a livello di gametogenesi e fecondazione ma avviene un aborto spontaneo in quanto l'allele in 
omozigosi \`e letale.
\subparagraph{Letalit\`a di un gene responsabile della colorazione del pelo}
Nel $1993$ si studia la ragione per cui una mutazione di un gene responsabile della colorazione del pelo porta alla morte dell'embrione se in omozigosi. Il gene si trova nel cromosoma
$2$ e ha come vicino il gene Raly, che ha tra i trascritti uno per la produzione di una proteina responsabile della corretta espressione genica e legante RNA. La mutazione del topo 
giallo non \`e puntiforme ma \`e una delezione del gene Raly, dello spazio intergenico e del promotore per il gene Agouti. Il colore giallo \`e dovuto a una sovraespressione del gene,
ora controllato dal promotore di Raly e la letalit\`a in omozigosi \`e dovuta alla mancanza della proteina che porta a un deragliamento dello sviluppo embrionale. 
\subsubsection{Alleli letali umani}
Esempi di alleli letali per l'uomo sono la sindrome di Tay-Sachs, la fibrosi cistica e la fenilchetonuria (\emph{PKU}), non tutti portano a un aborto ma causano un effetto visibile in
et\`a infantile riducendo l'aspettativa di vita. La qualit\`a della vita delle persone affette \`e migliorabile grazie alle interazioni dei geni con l'ambiente in modo da modulare
con esse la letalit\`a. 
\paragraph{Sindrome di Tay-Sachs}
La sindrome di Tay-Sachs \emph{TDS} \`e causata da mutazioni in omozigosi o eterozigosi composta della subunit\`a $\alpha$ del gene esosaminidasi $A$ \emph{HEXA} sul cromosoma $15q23$
\`E un disordine autosomale recessivo e progressivo neurodegenerativo che nella forma infantile \`e fatale nei primi $3$ anni di vita. \`E caratterizzato da un insieme di ritardi
nello sviluppo infantile seguito da paralisi, demenza e cecit\`a fino alla morte. Si riconosce grazie a un'area grigio biancastra intorno alla fovea centralis dell'occhio a causa di 
cellule gangliali ricche di lipidi. Una verifica patologica \`e fornita da neuroni a forma di pallone nel sistema nervoso centrale. \`E utile per un riconoscimento una precoce e 
persistente estensione della risposta ai suoni. 
\subparagraph{Genetica molecolare}
La lesione pi\`u frequente nella sindrome \`e un'inserzione di $4$ coppie di basi nell'esone $11$ del gene \emph{HEXA}. Il gene responsabile per la forma giovanile \`e allelico a quello
responsabile per la forma infantile e questi pazienti con la deficienza parziale sopravvivono fino ai $15$ anni. Il gene \emph{HEXA} codifica per la subunit\`a $\alpha$ dell'enzima
lisosomale $\beta$-esosaminidasi che insime al cofattore \emph{GM2} catalizza la degradazione del ganglioside \emph{GM2} e altre molecole contenenti terminale $N$-acetil esosaminasi. 
Mutazioni nella subunit\`a $\alpha$ o $\beta$ portano ad un accumulo del ganglioside \emph{GM2} nei neuroni e disordini neurodegenerativi detti \emph{GM2 gangliosidosi}. La malattia 
\`e causata da $78$ diverse mutazioni semplici, inserzioni o delezioni.
\paragraph{Interazioni geni-ambiente nella \emph{PKU}}
Le cellule viventi e gli organismi complessi hanno la necessit\`a di interagire con l'ambiente per acquisire materia causando una pressione evolutiva per fenomeni metabolici che
prendono le macromolecole dall'esterno per produrre energia o altre macromolecole. Si nota come possano esistere errori congeniti nel metabolismo, ovvero nei complessi pathway 
metabolici ci possono essere mutazioni di geni codificanti enzimi coinvolti in esso che portano ad un blocco del cammino. Un esempio di questo \`e la fenilchetonuria \emph{PKU}, 
che porta a una scorretta elaborazione della fenilanalina introdotta dalla dieta. Il fenotipo della malattia viene influenzato da vari fattori come la quantit\`a di fenilanalina
introdotta con la dieta (necessaria in quanto amminoacido essenziale), le mutazioni del gene per la fenilanalina idrossilasi (che la trasforma in tirosina), la produzione del cofattore 
necessario al suo funzionamento tetraidrobiopterina \emph{TBH}, i livelli di produzione di acido fenilpiruvico alla fine del pathway che viene trasportato fuori dal fegato nel sangue, 
la sua interazione con la barriera ematoencefalica e infine la gestione del suo accumulo nelle cellule nervose. Si nota come la complessit\`a delle interazioni con l'ambiente pu\`o 
rendere molto difficile il riconoscimento dei principi mendeliani attraverso lo studio degli effetti fenotipici, il determinismo genetico \`e reso pi\`u complesso. 
\subsection{Letalit\`a sintetica}
Due geni o due proteine vengono definiti letali sintetici quando la deficienza nell'espressione di uno dei due non compromette la vitalit\`a, mentre la contemporanea alterazione in 
entrambi \`e incompatibile con la sopravvivenza cellulare. Considerando due oggetti $A$ e $B$ che interagiscono tra di loro e con il DNA localizzandone una porzione creando 
un'interazione con una certa costante di DNA. Si indica con $A^+$ e $B^+$ la loro forma funzionale e con $A^-$ e $B^-$ la loro forma non funzionale. Nel caso della formazione del 
complesso $A^+B^-$ o $A^-B^+$ il complesso, nonostante la forma modificata \`e ancora in grado di legare il DNA e svolgere la sua funzione con costante di affinit\`a minore ma comunque
abbastanza per impedire la nascita del fenotipo drammatico. Nel caso in cui invece entrambi siano mutati e si formi il fenotipo $A^-B^-$ il complesso perde la sua affinit\`a con il 
DNA causando la comparsa del fenotipo drammatico dovuto alla sua perdita di funzionalit\`a. Questo concetto viene utilizzato per terapie antitumorali: i tumori possono presentare una
mutazione somatica con un difetto in uno degli elementi del complesso e per far perdere ad esso la funzione essenziale si utilizza un farmaco in grado di colpire il partner generando
la situazione letale nelle cellule tumorali. La tossicit\`a per le cellule non tumorali viene mantenuta bassa in quanto in esse non \`e presente la mutazione del partner. 
\section{Interazioni geniche e rapporti mendeliani modificati}
Si considera oltre a diversi alleli per un singolo gene come pi\`u geni possono influenzare lo stesso fenotipo. Ci saranno pertanto pi\`u coppie diverse di alleli su geni diversi che
convergono sulle stesse caratteristiche. 
\subsection{Esempi}
\subsubsection{Polli}
Si prende in considerazione la forma della cresta dei polli. Si consideri una generazione parentale in cui un individuo possiede una cresta frastagliata o Wyandotte e l'altro a fagiolo
o Brahma. Tentando di studiare la base genotipica per la differenza si incrociano gli individui. Si nota come nella $F_1$ tutti gli individui possiedano un fenotipo con cresta a noce. 
Incrociando ancora si ottiene una $F_2$ in cui si osservano quattro fenotipi:
\begin{itemize}
	\item Wyandotte $frac{3}{16}$.
	\item Brahma $frac{3}{16}$.
	\item Noce $frac{9}{16}$.
	\item A cresta singola o Longhorn $frac{1}{16}$.
\end{itemize}
I rapporti tra i fenotipi sembrano indicare un rapporto mendeliano e come la morfogenesi della cresta sia influenzata da due morfogeni. Chiamando i due geni $Rr$ e $Pp$ si determina il
fenotipo:
\begin{itemize}
	\item Wyandotte $RRpp$.
	\item Brahma $rrPP$.
	\item Noce $R-P-$.
	\item Longhorn $rrpp$.
\end{itemize}
L'assenza dei geni porta a un programma di base, mentre la presenza di uno dei due porta a situazioni intermedie tra entrambi presenti e entrambi assenti. 
\subsubsection{Peperoni}
Si consideri una popolazione parentale con un peperone rosso e verde provenienti dalla stessa pianta (si possono incrociare e ottenere una progenie fertile). La generazione $P$ \`e 
rosso con verde e la $F_1$ presenta peperoni tutti rossi. Autofecondando la $F_1$ si ottiene una $F_2$ in cui compaiono due nuovi fenotipi:
\begin{itemize}
	\item Marrone $frac{3}{16}$.
	\item Giallo $frac{3}{16}$.
	\item Rosso $frac{9}{16}$.
	\item Verde $frac{1}{16}$.
\end{itemize}
La numerologia esclude l'allelismo multiplo. I rapporti tra i fenotipi sembrano indicare un rapporto mendeliano e come il colore del peperone sia influenzato da due geni. Chiamando i due
geni $Cc$ e $Rr$ si determina il
fenotipo:
\begin{itemize}
	\item Marrone $R-cc$.
	\item Giallo $rrC-$.
	\item Rosso $R-C-$.
	\item Verde $rrcc$.
\end{itemize}
La $F_1$ \`e pertanto un'eterozigote da ambo le parti e il fenotipo \`e l'interazione di quattro alleli in due geni con dominanza completa. Quando entrambi recessivi non si produce
pigmento e rimane il colore dato dalla clorofilla.
\subsubsection{Lenticchie}
Si consideri un altro incrocio che conferma lo stesso principio. Si consideri una popolazione di lenticchie composta da individui con colore marrone-rosso $AAbb$ e con colore grigio 
$aaBB$. \`E un modello a due geni e la popolazione parentale \`e omozigote trans per entrambi. Nella $F_1$ tutti gli individui sono eterozigoti con entrambi associato con il fenotipo 
marrone. Incrociando $F_1$ si ottiene $F_2$ e si studia il fenotipo:
\begin{itemize}
	\item Marrone $frac{9}{16}$, $A-B-$.
	\item Marrone-rosso $frac{3}{16}$, $A-bb$.
	\item Grigio $frac{3}{16}$, $aaB-$.
	\item Verde $frac{1}{16}$, $aabb$.
\end{itemize}
\subsubsection{Drosophila melanogaster}
Si consideri la pigmentazione dell'occhio composito della Drosophila: questo si presnta o rosso o bianco. Incrociando i due individui si nota come in $F_1$ tutta la progenie abbia
occhi rossi. Incrociando un individuo di $F_1$ $Bw^+BwSt^+St$ e si incrocia con un individuo con gli occhi bianchi $BwBwStSt$. Si ottiene una $F_2$ in cui si notano nuovi fenotipi. 
\begin{itemize}
	\item Occhio rosso $frac{1}{4}$, $Bw^+St^+$.
	\item Occhio scarlatto $frac{1}{4}$, $Bw^+St$.
	\item Occhio marrone $frac{1}{4}$, $BwSt^+$.
	\item Occhio bianco $frac{1}{4}$, $BwSt$.
\end{itemize}
Si nota come il rapporto \`e di $\frac{1}{4}$ si lavora con due geni e quattro alleli e il fenotipo \`e guidato da un omozigosi di entrambi e nel fenotipo \`e guidato dal fenotipo 
dell'individuo doppio eterozigote che quando incrociato con un doppio recessivo. 
