\chapter{Mitosi e meiosi}
\section{Cromosomi e ciclo cellulare}
\subsection{Cromosomi}
Negli eucarioti il genoma \`e organizzato in cromosomi, molecole di DNA che in determinati momenti del ciclo cellulare si presentano altamente conservati e ben visibili. In alcuni 
momenti il cromosoma \`e costituito da un singolo cromatide mentre in altri \`e formato da due cromatidi fratelli. Le estremit\`a stabili dei cromosomi si dicono telomeri e questi 
presentano una regione contratta detta centromero, luogo di formazione del cinetocoro a cui si attaccano i microtubule del fuso. Il numero di cromosomi \`e tipico per ogni specie, negli
umani ne sono presenti $46$. Nella specie umana si trovano $23$ coppie di cromosomi compresi quelli sessuali $X$ e $Y$ ($XX$ per le femmine e $XY$ per i maschi). Gli esseri umani sono
pertanto diploidi: si trovano due serie di cromosomi organizzate in coppie omologhe che presentano una coppia di alleli (versioni di uno stesso gene) che codificano per una 
caratteristica. Le coppie di cromosomi si trovano nelle cellule somatiche e i loro membri sono detti omologhi. In ogni coppia uno dei cromosomi \`e di origine paterna e uno di origine
materna. Se sono presenti $2$ serie di cromosomi ha un corredo cromosomico diploide, se ne \`e presente solo una si dice aploide. 
\subsection{Ciclo cellulare}
Il ciclo vitale di una cellula si divide in due grandi parti: l'interfase in cui la cellula cresce e la fase $M$ in cui avviene la divisione nucleare e cellulare. 
\subsubsection{Interfase}
L'interfase viene a sua volta divisa in varie fasi: 
\begin{itemize}
	\item Fase $G_1$: la cellula si accresce e pu\`o decidere se entrare in $G_0$ o fase di quiescenza o raggiungere il checkpoint $G_1/S$. Una volta superato il checkpoint la
		cellula \`e programmata per dividersi.
	\item Fase $S$: viene duplicato il DNA.
	\item Fase $G_2$: la cellula si prepara per la mitosi. Questa continua fino a che si raggiunge il checkpoint $G_2/M$, dopo il quale la cellula pu\`o dividersi. 

\end{itemize}
\subsubsection{Fase $\mathbf{M}$}
Nella fase $M$ avvengono la mitosi e la citocinesi, ovvero la divisione cellulare che dar\`a origine a due cellule figlie che rientrano nella fase $G_1$.
\section{Mitosi}
La mitosi \`e il processo di divisione cellulare che garantisce la conservazione e la distribuzione dello stesso numero di cromosomi da una cellula madre alle due cellule figlie. Il
materiale cromosomico raddoppia una volta e la cellula si divide una volta. 
\subsection{Fasi della mitosi}
\subsubsection{Interfase}
Durante l'interfase \`e presente la membrana nucleare e i cromosomi sono in forma rilassata, entrano nel nucleo della cellula i centrosomi. 
\subsubsection{Profase}
La profase inizia quando i lunghi filamenti di cromatina cominciano a condensarsi attraverso processi di spiralizzazione in cui i cromosomi diventano pi\`u corti e pi\`u spessi. Ogni
cromosoma replicato durante la fase $S$ precedente consiste di una coppia di cromatidi fratelli. Ogni cromatide contiene un centromero. Si forma inoltre il fuso mitotico. 
\subsubsection{Prometafase}
Nella prometafase la membrana nucleare si disgrega e i microtubuli del fuso entrano in contatto con i cromosomi. 
\subsubsection{Metafase}
Nella metafase i cromosomi si allineano sulla piastra metafasica, il piano equatoriale della cellula. Per la loro corretta separazione si forma una connessione tra i microtubuli del 
cinetocoro e i cromosomi replicati. Il cinetocoro \`e un complesso proteico che aderisce al centromero. 
\subsubsection{Anafase}
Durante l'anafase i cromatidi fratelli si separano muovendosi verso i poli opposti.
\subsubsection{Telofase}
Durante la telofase i cromosomi giungono ai poli del fuso, si ricostituisce la membrana nucleare e i cromosomi subiscono un rilassamento. 
\subsection{Attivazione della fase $\mathbf{M}$}
I responsabili dell'inizio della fase $M$ in una cellula sono il \emph{MPF} (fattore di promotore della fase $M$) e la \emph{ciclina B}. 
\subsubsection{Fase $\mathbf{G_1}$}
All'inizio della fase $G_1$ i livelli di $MPF$ e di \emph{ciclina B} sono praticamente nulli. La cellula comincia a sintetizzare \emph{ciclina B}.
\subsubsection{Fase $\mathbf{S}$}
Durante la fase $S$ i livelli aumentati di \emph{ciclina B} si combinano con \emph{CDK} (chinasi ciclina-dipendente), producendo un aumento di \emph{MPF} inattivo. 
\subsubsection{Fase $\mathbf{G_2}$}
Durante la fase $G_2$ dell'interfase si accumula \emph{ciclina B}. Verso la fine della $G_2$ l'\emph{MPF} (fattore promotore della fase $M$) viene attivato attraverso fosforilazione 
da fattori di attivazione determinando la frammentazione dell'involucro nucleare, la condensazione dei cromosomi, l'assemblaggio del fuso e tutti gli altri fenomeni associati alla fase
$M$. \`E pertanto il livello critico di \emph{MPF} attivo a causare la progressione della cellula attraverso il punto di controllo $G_2/M$ e l'ingresso in mitosi. 
\subsubsection{Metafase}
Verso la fine della metafase la degradazione della \emph{ciclina B} riduce la quantit\`a di \emph{MPF} attivo provocando l'anafase, la telofase, la cinetochinesi e l'interfase. 
\subsection{Eventi drammatici nella mitosi}
\subsubsection{Rotture del DNA e polverizzazione dei cromosomi da errori nella mitosi}
In questo studio si tenta di identificare un meccanismo in cui errori nella segregazione dei cromosomi durante la mitosi genera rotture del DNA attraverso la formazione dei micronuclei. 
I micronuclei si formano quando errori mitotici producono cromosomi lagging. Studiandoli si nota come subiscono una replicazione del DNA asincrona e difettiva risultando in danno al
DNA e spesso frammentazione del cromosoma nel micronucleo. Il destino dei micronuclei \`e vario: possono persistere per molte generazioni o essere ridistribuiti in nuclei figli, pertanto
la segregazione errata pu\`o portare a mutazioni e riarrangiamenti del cromosoma che possono integrarsi nel genoma. La polverizzazione dei cromosomi nei micronuclei pu\`o essere anche
la causa del fenomeno di cromotripsi, dove cromosomi o loro braccia subiscono massive rotture del DNA e riarrangiamenti. Due modelli animali dove l'errore di segregazione risulta in 
sviluppo tumorale mostrano eventi di cromotripsi. I micronuclei si formano dai cromosomi in ritardo nell'anafase o da frammenti di cromosomi acentrici. Non si conosce precisamente la
composizione e propriet\`a funzionali dei micronuclei ma mostrano molte somiglianze con il nucleo. Diversi studi danno risposte diverse al fatto che i micronuclei siano attivi 
trascrizionalmente, replichino il DNA o abbiano una normale risposta al danno. Il fato ultimo del cromosoma intrappolato nei micronuclei rimane poco chiaro. 
\paragraph{Esperimento}
Per determinare se i micronuclei appena formati sviluppino danni al DNA si generano micronuclei in cellule sincronizzate e li si traccia attraverso il ciclo cellulare. 
\subparagraph{Sincronizzazione} 
Come primo approccio di sincronizzazione i micronuclei sono stati generati da cellule $U2OS$ trasformate dal rilascio di depolimerizzazione dei microtubuli indotta dal nocodazolo. 
Inoltre in quanto l'aneuploidia pu\`o causare un arresto del ciclo cellulare causato da \emph{p53} questa \`e stata silenziata da interferenza a RNA (RANi) in modo da permettere di 
monitorare il destino delle cellule a fasi successive del ciclo cellulare. Un altro metodo indipendente per generare i micronuclei avviene attraverso una linea cellulare umana $HT1080$ 
che porta un cromosoma umano artificiale $HAC$ con un cinetocoro che pu\`o essere condizionalmente inattivato. In questo sistema l'assemblaggio del cinetocoro sull'HAC \`e bloccata dal
lavaggio di dossiciclina dal medium in modo che HAC sia inabile di attaccarsi al fuso mitotico ed \`e lasciata indietro durante l'anafase riformandosi come micronucleo. 
\subparagraph{Osservazione dei micronuclei}
Presi insieme i micronuclei non presentano significativo danno al DNA durante $G_1$ ma una grande frazione lo acquisisce durante la fase $S$, danno che periste in $G_2$. Per determinare
se l'acquisizione del danno richiede la replicazione del DNA le cellule micronucleate sincronizzate sono state rilasciate in un medio contenente timidina per bloccare la replicazione del
DNA. Si nota come il blocco della replicazione abolisce l'acquisizione del danno al DNA dimostrando che le rotture nei micronuclei avvengono in una maniera dipendente dalla replicazione.
Per l'osservazione del danno si rilasciano le cellule sincronizzate in un medium con e in uno senza con $2mM$ timidina. Le cellule sono state colorate per \emph{TUNEL} (verde) e 
\emph{ciclina B1} (rosso). In un'altra osservazione le cellule vengono marcate con \emph{bromodeossiuridina} ($BrdU$), riconoscibile con un anticorpo e mostra la sintesi del DNA. Si
nota come il micronucleo si colora di rosso in un momento successivo, confermando l'ipotesi che il danno al DNA venga acquisito a causa di un ritardo nella sintesi del DNA al suo interno
rispetto ai cromosomi nei nuclei. 
\subparagraph{Rotture cromosomiche}

\subparagraph{Il destino dei cromosomi nei micronuclei}

\subsubsection{Cromotripsi causata dal danno del DNA nei micronuclei}

\subsubsection{Endoreplicazione, la poliploidia con uno scopo}

\section{Meiosi}
Gli organismi superiori si riproducono mediante l'unione di due cellule sessuali specializzate i gameti (aploidi) che si uniscono a formare un'unica cellula: lo zigote (diploide). I 
gameti sono prodotti nelle gonadi (testicolo e ovaio) a partire dalle cellule germinali. Se i gameti avessero lo stesso numero di cromosomi delle cellule del genitore che lo produce 
allora lo zigote avrebbe un numero doppio di cromosomi, raddoppiamento che si verificherebbe ad ogni generazione. Il mantenimento del numero costante di cromosomi \`e assicurato da
un processo di divisione cellulare ``riduzionale" detto meiosi. Durante la meiosi una cellula diploide va incontro a $2$ divisioni cellulari (prima e seconda divisione meiotica) 
producendo potenzialmente $4$ cellule aploidi.
\subsection{Meiosi $\mathbf{1}$}
Durante la prima meiosi i membri di ogni coppia di cromosomi omologhi prima si uniscono e poi si separano e vengono distribuiti in nuclei distinti (divisione riduzionale). 
\subsubsection{Profase $\mathbf{I}$}
La profase viene divisa a sua volta in due fasi.
\paragraph{Profase $\mathbf{I}$ intermedia}
I cromosomi iniziano a condensarsi e si forma il fuso. 
\paragraph{Profase $\mathbf{I}$ tardiva}
I cromosomi omologhi si appaiano, si verifica il crossing-over e la membrana nucleare si disgrega.
\subsubsection{Metafase $\mathbf{I}$}
Le coppie di cromosomi omologhi si allineano lungo la piastra metafasica.
\subsubsection{Anafase $\mathbf{I}$}
I cromosomi omologhi si separano muovendosi verso i poli opposti. 
\subsubsection{Telofase $\mathbf{I}$}
I cromosomi giungono ai poli del fuso e il citoplasma si divide.
\subsection{Meiosi $\mathbf{2}$}
Durante la seconda meiosi i cromatidi che costituiscono ciascun cromosoma omologo si separano e vengono distribuiti ai nuclei delle cellule figlie (divisione equazionale). Si producono
cos\`i alla fine quattro cellule aploidi. 
\subsubsection{Profase $\mathbf{II}$}
I cromosomi si condensano nuovamente. 
\subsubsection{Metafase $\mathbf{II}$}
I singoli cromosomi si allineano lungo la piastra equatoriale. 
\subsubsection{Anafase $\mathbf{II}$}
I cromatidi fratelli si separano spostandosi verso i poli opposti. 
\subsubsection{Telofase $\mathbf{II}$}
I cromosomi giungono ai poli del fuso e il citoplasma si divide. 
\subsection{Confronto con mitosi}
I processi di base della meiosi sono simili a quelli della mitosi a netto di:
\begin{itemize}
	\item La meiosi comporta $2$ successive divisioni nucleari e citoplasmatica con potenziale produzione di $4$ cellule. 
	\item Nonostante le due divisioni il DNA subisce una sola duplicazione durante l'interfase che precede la divisione meiotica. 
	\item Ognuna delle $4$ cellule prodotte contiene un numero aploide di cromosomi, un solo esemplare di ogni coppia di omologhi. 
	\item Durante la meiosi l'informazione genetica che proviene da entrambi i genitori viene mescolata in maniera casuale in modo che ogni cellula possieda una combinazione di geni
		potenzialmente unica. 
\end{itemize}
\subsection{Crossing-over}

\subsubsection{\emph{PRDM9} organizza gli hotspot dei nucleosomi e limita la migrazione delle giunzioni di Holliday}



