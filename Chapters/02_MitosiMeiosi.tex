\chapter{Mitosi e meiosi}
\section{Riproduzione cellulare}
Affinch\`e una cellula si riproduca con successo:
\begin{itemize}
	\item Le sue informazioni genetiche devono essere copiate.
	\item Le copie di queste informazioni devono essere separate le une dalle altre.
	\item La cellula deve dividersi.
\end{itemize}
\subsection{Riproduzione procariote}
Il processo di replicazione del cromosoma batterico circolare e di divisione cellulare viene detto scissione binaria. La replicazione inizia in un \emph{Ori} (origine di replicazione). Le origini dei 
due cromosomi si allontanano verso gli estremi opposti della cellula. In alcuni batteri proteine fissano i cromosomi alla membrana plasmatica alle estremit\`a opposte. Tra i cromosomi si forma una 
nuova parete cellulare che porta allo sviluppo di due cellule. In condizioni ottimali pu\`o avvenire anche ogni $20min$.
\subsection{Riproduzione eucariote}
Anche la riproduzione della cellula eucariote richiede che si verifichino processi di replicazione del DNA, separazione e divisione nel citoplasma delle copie. Il nucleo possiede una struttura interna 
detta matrice nucleare costituita da fibre proteiche che mantengono relazioni spaziali precise tra i componenti nucleari. 
\section{Cromosomi}
Negli eucarioti il genoma \`e organizzato in cromosomi, molecole di DNA che in determinati momenti del ciclo cellulare si presentano altamente conservati e ben visibili. Ogni specie contiene un numero
caratteristico di cromosomi (umani $46$). A causa della riproduzione sessuata sono presenti due serie di cromosomi: una ereditata dal genitore di sesso femminile e l'altra da quello maschile. Ogni 
cromosoma possiede pertanto un cromosoma corrispondente nell'altra con cui forma una coppia omologa. I due cromosomi di una coppia omologa hanno struttura simile e trasporta informazioni relative alla
stessa serie di tratti ereditari. Copie di geni su cromosomi omologhi sono dette alleli. Costituiscono un'eccezione i cromosomi sessuali. Si dice ploidia il numero di serie di informazioni genetiche
che una cellula possiede.
\begin{itemize}
	\item Si dicono diploidi le cellule con due serie di informazioni.
	\item Si dicono aploidi le cellule con un'unica serie di informazioni.
	\item Si dicono poliploidi le cellule che contengono pi\`u serie di informazioni.
\end{itemize}
Il cromosoma pu\`o essere costituito da un singolo cromatide (una molecola di DNA) o da due cromatidi fratelli (due molecole di DNA).
\subsection{Struttura}
Essendo i cromosomi molto lunghi si rende necessario compattarli intorno alle proteine istoniche fino a costruire un cromosoma a forma di bastoncello. Nel corso della vita della cellula i cromosomi sono 
troppo esili per essere osservati, ma dopo la divisione cellulare si condensano fino a formare strutture osservabili. 
\subsubsection{Centromero}
Il centromero \`e il punto di attacco dei microtubuli del fuso, i filamenti responsabile del movimento del cromosoma durante la divisione cellulare. Ha l'aspetto di una strozzatura e prima 
della divisione si forma in esso una struttura multi-proteica detta cinetocore a cui si saldano i microtubuli del fuso. La posizione del centromero permette di classificare i cromosomi in metacentrici, 
submetacentrici, acrocentrici e telocentrici. 
\subsubsection{Telomeri}
I telomeri costituiscono la regione terminale, proteggono e stabilizzano le terminazioni lineari dei cromosomi. Garantiscono la stabilit\`a del cromosoma e limitano la divisione cellulare. Hanno un 
ruolo importante nella senescenza e insorgenza del cancro.
\subsubsection{Origini di replicazione}
Le origini di replicazione sono i siti in cui ha inizio la sintesi del DNA. Quando ogni cromosoma si replica produce una copia di s\`e stesso, un cromatidio fratello che gli rimane unito a livello del
centromero. 
\section{Ciclo cellulare e mitosi}
\subsection{Ciclo cellulare}
Il ciclo vitale di una cellula si divide in due grandi parti: l'interfase in cui la cellula cresce e la fase $M$ in cui avviene la divisione nucleare e cellulare. 
\subsubsection{Interfase}
L'interfase  \`e il periodo di crescita e sviluppo compreso tra divisioni cellulari. Viene a sua volta divisa in varie fasi: 
\begin{itemize}
	\item Fase $G_1$: la cellula si accresce e pu\`o decidere se entrare in $G_0$ o fase di quiescenza o raggiungere il checkpoint $G_1/S$. Una volta superato il checkpoint la
		cellula \`e programmata per dividersi.
	\item Fase $S$: viene duplicato il DNA.
	\item Fase $G_2$: la cellula si prepara per la mitosi. Questa continua fino a che si raggiunge il checkpoint $G_2/M$, dopo il quale la cellula pu\`o dividersi. 

\end{itemize}
\subsubsection{Fase $\mathbf{M}$}
Nella fase $M$ avvengono la mitosi e la citocinesi, ovvero la divisione cellulare che dar\`a origine a due cellule figlie che rientrano nella fase $G_1$.
\subsection{Mitosi}
La mitosi \`e il processo di divisione cellulare che garantisce la conservazione e la distribuzione dello stesso numero di cromosomi da una cellula madre alle due cellule figlie. Il
materiale cromosomico raddoppia una volta e la cellula si divide una volta. 
\subsubsection{Fasi della mitosi}
\paragraph{Interfase}
Durante l'interfase \`e presente la membrana nucleare e i cromosomi sono in forma rilassata, entrano nel nucleo della cellula i centrosomi. 
\paragraph{Profase}
La profase inizia quando i lunghi filamenti di cromatina cominciano a condensarsi attraverso processi di spiralizzazione in cui i cromosomi diventano pi\`u corti e pi\`u spessi. Ogni
cromosoma replicato durante la fase $S$ precedente consiste di una coppia di cromatidi fratelli. Ogni cromatide contiene un centromero. Si forma inoltre il fuso mitotico, nelle cellule animali a partire
da una coppia di centrosomi che migrano alle estremit\`a della cellula. I centrosomi contengono il centriolo costituito da microtubuli.
\paragraph{Prometafase}
Nella prometafase la membrana nucleare si disgrega e i microtubuli del fuso entrano in contatto con i cromosomi: ogni cromosoma si fissa a microtubuli che provengono da poli opposti del fuso: un 
microtubulo proveniente dal centrosoma si fissa al cinetocore di uno dei cromatidi fratelli, mente il microtubulo proveniente dall'altro centrosoma si fissa a quello rimasto libero. Questa disposizione
viene detta  biorientamento cromosomico.
\paragraph{Metafase}
Nella metafase i cromosomi si allineano sulla piastra metafasica, il piano equatoriale della cellula, situata tra i due centrosomi. Ora un checkpoint di assemblaggio del fuso mitotico garantisce che
ogni cromosoma si allineato sulla piastra e assicurato alle fibre dei poli opposti del fuso. Il passaggio da questo checkpoint dipende dalla tensione generata sul cinetocore quando i due cromatidi vengono
tirati in direzioni opposti.
\paragraph{Anafase}
Durante l'anafase i cromatidi fratelli si separano muovendosi verso i poli opposti. Proteine dette motori molecolari rimuovono tubulina dal fuso in modo da generare forze che trascinano il cromosoma 
verso i poli. 
\paragraph{Telofase}
Durante la telofase i cromosomi giungono ai poli del fuso, si ricostituisce la membrana nucleare e i cromosomi subiscono un rilassamento. 
\subsubsection{Attivazione della fase $\mathbf{M}$}
I responsabili dell'inizio della fase $M$ in una cellula sono il \emph{MPF} (fattore di promotore della fase $M$) e la \emph{ciclina B}. 
\paragraph{Fase $\mathbf{G_1}$}
All'inizio della fase $G_1$ i livelli di $MPF$ e di \emph{ciclina B} sono praticamente nulli. La cellula comincia a sintetizzare \emph{ciclina B}.
\paragraph{Fase $\mathbf{S}$}
Durante la fase $S$ i livelli aumentati di \emph{ciclina B} si combinano con \emph{CDK} (chinasi ciclina-dipendente), producendo un aumento di \emph{MPF} inattivo. 
\paragraph{Fase $\mathbf{G_2}$}
Durante la fase $G_2$ dell'interfase si accumula \emph{ciclina B}. Verso la fine della $G_2$ l'\emph{MPF} (fattore promotore della fase $M$) viene attivato attraverso fosforilazione 
da fattori di attivazione determinando la frammentazione dell'involucro nucleare, la condensazione dei cromosomi, l'assemblaggio del fuso e tutti gli altri fenomeni associati alla fase
$M$. \`E pertanto il livello critico di \emph{MPF} attivo a causare la progressione della cellula attraverso il punto di controllo $G_2/M$ e l'ingresso in mitosi. 
\paragraph{Metafase}
Verso la fine della metafase la degradazione della \emph{ciclina B} riduce la quantit\`a di \emph{MPF} attivo provocando l'anafase, la telofase, la cinetochinesi e l'interfase. 
\section{Eventi drammatici nella mitosi}
\subsection{Rotture del DNA e polverizzazione dei cromosomi da errori nella mitosi}
In questo studio si tenta di identificare un meccanismo in cui errori nella segregazione dei cromosomi durante la mitosi genera rotture del DNA attraverso la formazione dei micronuclei. 
I micronuclei si formano quando errori mitotici producono cromosomi lagging. Studiandoli si nota come subiscono una replicazione del DNA asincrona e difettiva risultando in danno al
DNA e spesso frammentazione del cromosoma nel micronucleo. Il destino dei micronuclei \`e vario: possono persistere per molte generazioni o essere ridistribuiti in nuclei figli, pertanto
la segregazione errata pu\`o portare a mutazioni e riarrangiamenti del cromosoma che possono integrarsi nel genoma. La polverizzazione dei cromosomi nei micronuclei pu\`o essere anche
la causa del fenomeno di cromotripsi, dove cromosomi o loro braccia subiscono massive rotture del DNA e riarrangiamenti. Due modelli animali dove l'errore di segregazione risulta in 
sviluppo tumorale mostrano eventi di cromotripsi. I micronuclei si formano dai cromosomi in ritardo nell'anafase o da frammenti di cromosomi acentrici. Non si conosce precisamente la
composizione e propriet\`a funzionali dei micronuclei ma mostrano molte somiglianze con il nucleo. Diversi studi danno risposte diverse al fatto che i micronuclei siano attivi 
trascrizionalmente, replichino il DNA o abbiano una normale risposta al danno. Il fato ultimo del cromosoma intrappolato nei micronuclei rimane poco chiaro. 
\subsubsection{Esperimento}
Per determinare se i micronuclei appena formati sviluppino danni al DNA si generano micronuclei in cellule sincronizzate e li si traccia attraverso il ciclo cellulare. 
\paragraph{Sincronizzazione} 
Come primo approccio di sincronizzazione i micronuclei sono stati generati da cellule $U2OS$ trasformate dal rilascio di depolimerizzazione dei microtubuli indotta dal nocodazolo. 
Inoltre in quanto l'aneuploidia pu\`o causare un arresto del ciclo cellulare causato da \emph{p53} questa \`e stata silenziata da interferenza a RNA (RANi) in modo da permettere di 
monitorare il destino delle cellule a fasi successive del ciclo cellulare. Un altro metodo indipendente per generare i micronuclei avviene attraverso una linea cellulare umana $HT1080$ 
che porta un cromosoma umano artificiale $HAC$ con un cinetocore che pu\`o essere condizionalmente inattivato. In questo sistema l'assemblaggio del cinetocoro sull'HAC \`e bloccata dal
lavaggio di dossiciclina dal medium in modo che HAC sia inabile di attaccarsi al fuso mitotico ed \`e lasciata indietro durante l'anafase riformandosi come micronucleo. 
\paragraph{Osservazione dei micronuclei}
Presi insieme i micronuclei non presentano significativo danno al DNA durante $G_1$ ma una grande frazione lo acquisisce durante la fase $S$, danno che periste in $G_2$. Per determinare
se l'acquisizione del danno richiede la replicazione del DNA le cellule micronucleate sincronizzate sono state rilasciate in un medio contenente timidina per bloccare la replicazione del
DNA. Si nota come il blocco della replicazione abolisce l'acquisizione del danno al DNA dimostrando che le rotture nei micronuclei avvengono in una maniera dipendente dalla replicazione.
Per l'osservazione del danno si rilasciano le cellule sincronizzate in un medium con e in uno senza con $2mM$ timidina. Le cellule sono state colorate per \emph{TUNEL} (verde) e 
\emph{ciclina B1} (rosso). In un'altra osservazione le cellule vengono marcate con \emph{bromodeossiuridina} ($BrdU$), riconoscibile con un anticorpo e mostra la sintesi del DNA. Si
nota come il micronucleo si colora di rosso in un momento successivo, confermando l'ipotesi che il danno al DNA venga acquisito a causa di un ritardo nella sintesi del DNA al suo interno
rispetto ai cromosomi nei nuclei. 
\paragraph{Rotture cromosomiche}
Successivamente si procede per testare la predizione che replicazione anormale del DNA nei micronuclei pu\`o generare rotture cromosomiche. Si preparano dalle cellule non trasformate
del primo ciclo cellulare dopo il rilascio del nocodazolo o dai controlli trattati con $DMSO$. Si nota come il $7.6\%$ dei cromosomi esibisce cromosomi che appaiono frammentati 
colorati attraverso $DAPI$. Il meccanismo di polverizzazione coinvolge compattamento di cromosomi parzialmente replicati indotto dall'attivit\`a di \emph{CDK} e viene detto compattazione
cromosomica prematura. 
\paragraph{Il destino dei cromosomi nei micronuclei}
Le aberrazioni cromosomi acquisite nei micronuclei possono essere reincorporate nel genoma. La maggior parte dei micronuclei sono stabilmente mantenuti durante l'interfase e nonostante
alcuni micronuclei possono essere estrusi non ne sono stati individuati dall'esperimento. I micronuclei non erano degradati, non co-localizzano con i lisosomi e non si fondono con 
il nucleo primario, ma dopo la rottura della membrana nucleare alcuni micronuclei possono unirsi ad altri cromosomi mitotici ed essere distribuiti alle cellule figlie. Si mostra
pertanto come i micronuclei persistono in diverse generazioni e che il cromosoma contenuto in esso pu\`o essere segregato nei nuclei delle cellule figlie. Pertanto riarrangiamenti 
del DNA e mutazioni nei micronuclei possono essere incorporati nel genoma di una cellula. Questo meccanismo potrebbe giustificare il fenomeno della cromotripsi. 
\paragraph{Cromotripsi}
La cromotripsi \`e stata scoperta sequenziando il genoma delle cellule tumorali ed \`e definita da cambi del numero di copie del DNA in piccola scala e riarrangiamenti intracromosomiali 
ristretti a un singolo cromosoma o a un suo braccio. Sono stati proposti due modelli non esclusivi per la cromotripsi:
\begin{itemize}
	\item La frammentazione di un cromosoma seguita da riunione attraverso unione di terminazioni non omologhe.
	\item La replicazione del DNA aberrante risultante in stalli della forcella e cambio di stampo o replicazione indotta da rotture e mediata da micro-omologie.
\end{itemize}
\subsection{Cromotripsi causata dal danno del DNA nei micronuclei}
La cromotripsi \`e caratterizzata da riarrangamenti genomici estensivi e un pattern oscillante di numero di copie di DNA ristretti a uno o pi\`u cromosomi. Il meccanismo non \`e 
conosciuto ma potrebbe essere causato dall'isolamento di un cromosoma nei micronuclei. Nell'esperimento si dimostra come il meccanismo della cromotripsi pu\`o coinvolgere la 
frammentazione e il riassemblaggio di un singolo cromatide da un micronucleo. Studi del genoma del cancro mostrano come esistano eventi di mutazione che generano mutazioni tutte in 
una volta durante un singolo ciclo cellulare. Un esempio di questo \`e la cromostripsi, dove avviene un pattern unico di riarrangamenti raggruppati coinvolgendo uno o pochi cromosomi. 
Si dimostra attraverso imaging di singole cellule con analisi ``Look-Seq" come la formazione di micronuclei pu\`o generare uno spettro di riarrangiamenti cromosomiali complesso, fornendo
la prova per un meccanismo che porta alla cromotripsi. 
\subsubsection{Strategia look-seq}
Per determinare le conseguenze genomiche del danno al DNA nei micronuclei si prendono cellule non trasformate \emph{RPE-1} sincronizzate dal rilascio di nocodazolo e poste in piastre
a pozzetti. Si identificano i pozzetti contenenti una singola cellula micronucleata. Attraverso imaging delle cellule in vivo si identificano le cellule dove la membrana micronucleare
si \`e rotta dopo l'inizio della fase $S$. Questi esperimenti sono stati utilizzando dopo attraverso eliminazione di \emph{p53} attraverso \emph{siRNA}. Dopo una divisione della cellula
micronucleata si selezionano le figlie senza micronuclei indicando che il cromosoma micronucleare \`e stato reincorporato nel nucleo primario. Le cellule sono state selezionate in quanto
la rottura disattiva processi di replicazione e trascrizione del DNA. Le cellule figlie sono state successivamente separate, amplificate (multistrand displacement amplification, 
\emph{MDA}), sequenziate ed analizzate indipendentemente.
\subsubsection{Destino dei cromosomi}
\paragraph{Caso $\mathbf{1}$}
Il cromosoma in ritardo viene correttamente segregato ma partizionato in un micronucleo in una cellula figlia. Il cromosoma in esse viene sotto replicato e segregato risultando in un
rapporto tra le figlie di questa di $2:1$.
\paragraph{Caso $\mathbf{2}$}
Il cromosoma in ritardo \`e mal segregato in un micronucleo in una cellula figlia. Il crsomosoma \`e sotto replicato e segregato asimmetricamente, risultando in un rapporto tra le figlie
di questa di $3:2$.
\subsubsection{Conclusioni}
La segregazione mitotica errata pu\`o essere altamente mutagenica, con importanti implicazioni per come questi errori e le aneploidie potrebbero aver contribuito al cancro o altre 
malattie umane. La cromotripsi \`e presente in una piccola percentuale di cancri umani e altri disordini congenitali, ma il tasso di cromostripsi \`e probabilmente pi\`u alto in 
quanto la maggior parte di questi eventi compromettono il fitness cellulare e potrebbero essere individuati solo da un'analisi unicellulare. I micronuclei potrebbero pertanto 
essere un'importante fonte di variabilit\`a genetica. 
\subsection{Endoreplicazione, la poliploidia con uno scopo}
Si nota come un aspetto interessante della diversit\`a dei tipi cellulari \`e che molte cellule negli organismi dipolidi sono poliploidi. Questo evento di dice endoploidia ed \`e 
essenziale per il normale sviluppo e fisiologia di molti diversi organismi. Si studiano come sia piante ed animali usino varianti del ciclo cellulare o endoreplicazioni risultando in
cellule poliploidi che supportano specifici aspetti dello sviluppo. L'endoploidia pu\`o inoltre avvenire in risposta a certi stress fisiologici e come pu\`o portare allo sviuluppo di 
tumori. I fattori che contribuiscono all'endoreplicazione sono stress esterno, crescita e differenziazione. Pu\`o essere indotta inoltre per creare una catastrofe mitotica alle 
cellule tumorali che causa endomitosi e sopravvivenza della cellule e una de poliloidizzazione che porta a una proliferazione mitotica. 
\subsubsection{Meccanismi di endoreplicazione}
\paragraph{Endocicli}
Gli endocicli sono definiti come cicli cellulari consistenti di una fase $S$ e $G$ senza la divisione cellulare. Le cellule endociclanti non entrano in mitosi: non condensano i 
cromosomi e non rompono la membrana nucleare. I tricomi sorgono dalle cellule poliploidi che possono essere trovate sulla superficie di tessuti di piante. 
\paragraph{Rereplicazione}
La rereplicazione risulta da regolazione aberrante in cui la sintesi del DNA \`e iniziata multiple volte a origini di replicazione individuali durante una singola fase $S$. Questo 
risulta in una crescita del contenuto del DNA.
\paragraph{Endomitosi}
Durante l'endomitosi  le cellule entrano la mitosi e iniziano a condensare i cromosomi senza segregarli ma invece rientrando in uno stato simile a $G_1$ e dopo lafase $S$. I 
megacariociti usano endomitosi durante la maturazione portando a una struttura nucleare globulata da cui gemmano coaguli che promuovono trombociti. 
\subsubsection{Esempi di tessuti endociclanti}
\paragraph{Embrione vegetale}
Un embrione vegetale consiste di una capsula del seme che copre l'endosperma e circonda e fornisce nutrienti per i cotiledoni crescenti e per l'ipocotile dell'embrio. Le cellule 
sospensore sorgono dalla divisione asimmetrica dell'uovo fertilizzato e connettono l'embrio all'endosperma.
\paragraph{Ovarie della Drosophila}
Le ovarie della Drosophila consistono di $12-15$ ovarioli che contengono una serie di camere uovo in sviluppo. Il germarium porta le cellule staminale della linea germinale e somatiche 
che si differenziano in cellule infermiere e oociti e in cellule del follicolo rispettivamente. Le seconde fanno endocicli dirante l'oogenesi in risposta a segnalazione di Notch che 
sottoregola gli stimolatori della mitosi e attiva suoi inibitori. 
\paragraph{\emph{TGC} dei roditori}
Le \emph{TGC} dei roditori sono altamente poliploidi e facilitano l'impiantamento dell'embrione contribuendo all'invasine della parete uterina. 
\paragraph{Ipocotile}
L'ipocotile vegetale subisce endocicli per crescere rapidamente al di sopra del suolo. L'endoreplicazione si ferma una volta che la pianta raggiunge il sole. 
\subsubsection{Regolazione dell'endociclo della Drosophila}
Un complesso vettore di controlli assicura l'unicit\`a della replicazione durante la progressione endociclica. I fattori principali sono indicati nell'immagine in rosso quando inattivi 
e in verde quando attivi rispettivamente nella fase $S$ e $G$. Il controllo di \emph{CycE/Cdk2} forma il nucleo della regolazione endociclica: insieme a \emph{CycE}  hanno attivit\`a
bassa durante la fase $G$ quando $APC/C^{fzr/cdh1}$ reprime l'accumulo di \emph{Geminin} permettendo la fomrazione di \emph{pre-RC}. La stimolazione da parte di \emph{E2F} della 
trascrizione \emph{CycE} porta all'attivazione di \emph{CycE/Cdk2} e l'iniziazione della replicazione del DNA che causa la distruzione di \emph{E2F1}. \emph{CycE/Cdk2} reprime la 
formazione di \emph{pre-RC} e inattiva $APC/C^{fzr/cdh1}$ che permette un accumulo di \emph{Geminin} che inibisce la formazione di \emph{pre-RC}. 
\subsubsection{Varianti dell'endociclo}
Poliplodia somatica pu\`o accadere da abbreviazioni del ciclo cellulare, in cui diverse fasi del ciclo sono saltate o causano l'uscita dal ciclo cellulare. 
\subsubsection{Il modello della soglia a \emph{CDK}}
Questo modello \`e stato proposto per la fissione del lievito e poi esteso alle cellule animali si proponeva come l'iniziazione della fase $S$ ed $M$ sono causate da diverse soglie di
attivit\`a della chinasi dipendente da ciclina \emph{CDK}. Questa viene attivata duante la fase $S$ da cicline di tipo $E$ od $A$ e per la fase $M$ di tipo $A$ o $B$ complessate
rispettivamente con \emph{CDK2} e \emph{CDK1}. Nelle cellule endociclanti la soglia per la fase $S$ \`e periodicamente raggiunta, mentre durante la mitosi si trova un basso livello di
\emph{CDK}.
\section{La riproduzione sessuata - Meiosi}
Gli organismi superiori si riproducono mediante l'unione di due cellule sessuali specializzate i gameti (aploidi) che si uniscono a formare un'unica cellula: lo zigote (diploide). I 
gameti sono prodotti nelle gonadi (testicolo e ovaio) a partire dalle cellule germinali. Se i gameti avessero lo stesso numero di cromosomi delle cellule del genitore che lo produce 
allora lo zigote avrebbe un numero doppio di cromosomi, raddoppiamento che si verificherebbe ad ogni generazione. Il mantenimento del numero costante di cromosomi \`e assicurato da
un processo di divisione cellulare ``riduzionale" detto meiosi. Durante la meiosi una cellula diploide va incontro a $2$ divisioni cellulari (prima e seconda divisione meiotica) 
producendo potenzialmente $4$ cellule aploidi. Successivamente avviene il mescolamento delle informazioni geniche dai due genitori durante la fecondazione. Anche la meiosi \`e preceduta da un'interfase
che comprende le fasi $G_1$, $S$ e $G_2$. La prima delle due divisioni si dice riduzionale, mentre la seconda equazionale. La seconda divisione meiotica \`e analoga a quella della mitosi a parte per il
numero di cromosomi che \`e gi\`a stato dimezzato.
\subsection{Meiosi $\mathbf{1}$}
Durante la prima meiosi i membri di ogni coppia di cromosomi omologhi prima si uniscono e poi si separano e vengono distribuiti in nuclei distinti (divisione riduzionale). 
\subsubsection{Profase $\mathbf{I}$}
La profase viene divisa a sua volta in cinque fasi.
\paragraph{Leptotene}
I cromosomi si contraggono e diventano visibili.
\paragraph{Zigotene}
I cromosomi continuano a condensarsi, gli omologhi si appaiano e inizia la sinapsi, un legame stretto tra le coppie. Ogni coppia di cromosomi omologi sinaptici \`e costituita da quattro cromatidi e 
detta bivalente o tetrade.
\paragraph{Pachitene}
I cromosomi si accorciano e addensano. Tra i cromosomi omologhi si forma il complesso sinaptonemale tripartito. Inizia il crossing-over.
\paragraph{Diplotene}
I centromeri dei cromosomi appaiati si separano e i due omologhi rimangono uniti nel chiasma, risultato del crossing-over.
\paragraph{Diacinesi}
Si dissolve la membrana nucleare e si forma il fuso. 
\subsubsection{Metafase $\mathbf{I}$}
Le coppie di cromosomi omologhi si allineano lungo la piastra metafasica.
\subsubsection{Anafase $\mathbf{I}$}
I cromosomi omologhi si separano muovendosi verso i poli opposti. I cromatidi fratelli restano attaccati e si spostano insieme. 
\subsubsection{Telofase $\mathbf{I}$}
I cromosomi giungono ai poli del fuso e il citoplasma si divide.
\subsection{Intercinesi}
Il periodo tra meiosi I e II si dice intercinesi. In questa fase si riforma la membrana nucleare intorno ai cromosomi in ogni polo, il fuso si disgrega e i cromosomi si distendono.
\subsection{Meiosi $\mathbf{II}$}
Durante la seconda meiosi i cromatidi che costituiscono ciascun cromosoma omologo si separano e vengono distribuiti ai nuclei delle cellule figlie (divisione equazionale). Si producono
cos\`i alla fine quattro cellule aploidi. 
\subsubsection{Profase $\mathbf{II}$}
I cromosomi si condensano nuovamente, si forma il fuso e si disgrega la membrana nucleare.
\subsubsection{Metafase $\mathbf{II}$}
I singoli cromosomi si allineano lungo la piastra equatoriale. 
\subsubsection{Anafase $\mathbf{II}$}
I cromatidi fratelli si separano spostandosi verso i poli opposti. 
\subsubsection{Telofase $\mathbf{II}$}
I cromosomi giungono ai poli del fuso e il citoplasma si divide. 
\subsection{Confronto con mitosi}
I processi di base della meiosi sono simili a quelli della mitosi a netto di:
\begin{itemize}
	\item La meiosi comporta $2$ successive divisioni nucleari e citoplasmatica con potenziale produzione di $4$ cellule. 
	\item Nonostante le due divisioni il DNA subisce una sola duplicazione durante l'interfase che precede la divisione meiotica. 
	\item Ognuna delle $4$ cellule prodotte contiene un numero aploide di cromosomi, un solo esemplare di ogni coppia di omologhi. 
	\item Durante la meiosi l'informazione genetica che proviene da entrambi i genitori viene mescolata in maniera casuale in modo che ogni cellula possieda una combinazione di geni
		potenzialmente unica. 
	\item Nella metafase $I$ della mesiosi le coppie omologhe si allineano, mentre nella mitosi sono i singoli cromosomi a disporsi sulla piastra metafasica. Nella metafase $I$ della meiosi
		i cromosomi appaiati si separano e migrano possiede due cromatidi uniti al centromero, mentre nell'anafase della mitosi i cromatidi fratelli si dividono e i cromsomi che si muovono sono
		costituiti da un unico cromatidio.
\end{itemize}
\subsection{Fonti di variazione genetica nella meiosi}
\subsubsection{Crossing-over}
Il fenomeno di crossing-over \`e l'evento di ricombinazione meiotica. Durante la meiosi un'induzione programmata di rotture a doppio strand di DNA (\emph{DSB}) che porta allo scambio
di materiale tra cromosomi omologhi. Questi scambi portano ad un aumento della diversit\`a genomica e sono essenziali per la segregazione corretta alla prima divisione meiotica. Si trova
un controllo molecolare della distribuzione dei \emph{DSB} meiotici in mammiferi da un gene che si evolve rapidamente contenente un dominio contenente \emph{PR}: \emph{PRDM9}. I siti 
di rottura sono determinati e si trovano altre molecole che si occupano dei processi di riparazione che hanno permesso di delineare i cammini di ricombinazione che portano a crossover
e non-crossover con ruoli diversi nell'evoluzione genomica. 
\paragraph{Organizzazione dei cromosomi e citologia durante la profase meiotica $\mathbf{I}$}
La profase meiotica $I$ si divide in leptotene, zigotene, pachitene e diplotene. Si nota l'organizzazione dei cromosomi durante le varie fasi attraverso due cromatidi fratelli. La 
ricombinazione meiotica inizia con la formazione di \emph{DSB} durante il leptonema ed \`e completata prima della fine del pachynema. La synapsi \`e iniziata durante il zygonema durante
il quale entrambe le terminazioni dei cromosomi sono attaccate alla membrana nucleare. La transizione da leptonema a zygonema viene detto stage a bouquet in cui i telomeri si 
raggruppano lungo un polo nucleare. 
\paragraph{Meccanismo di crossing over}
Il crossing over inizia con una \emph{DSB} ad un sito specifico individuato da \emph{PRDM9} che media la rottura e recluta \emph{SPO11} che permette la ricombinazione meiotica. Alla fine
si trova un cromatide separato con rottura asimmetrica che compie una strand invasion. A seguito dell'invasione si possono trovare due intermedi: uno di crossover e uno di non crossover.
Il primo viene risolto andando a sostituire nei due cromatidi fratelli la sequenza successiva alla rottura, il secondo un gene del cromatide non rotto si trova su quello che \`e stato 
rotto. L'evento di crossover pu\`o essere individuato unicamente se differiscono in marcatori genetici. 
\paragraph{Modello del ruolo di \emph{PRDM9} nella localizzazione meiotica di \emph{DSB}}
La proteina PR domain binding $9$ si lega a un motivo di DNA specifico attraverso vettori di zinc finger \emph{C2H2}. In seguito il dominio \emph{PR/SET} promuove la trimetilazione 
della lisina $4$ sull'istone \emph{H3} sui nucleosomi adiacenti. La Kr\"uppel-associated box \emph{KRAB} potrebbe portare interazioni con altre proteine. Questi passi e altri permettono
il reclutamento del macchinario di \emph{DSB} insieme alla proteina di ricombinazione meiotica \emph{SPO11}.
\subparagraph{Struttura e funzione di \emph{PRDM9}}
La proteina \emph{PRDM9} \`e un iston-metiltrasferasi che consiste di tre regioni principali: una $N$ terminale che contiene un dominio Kr\"uppel-associated box \emph{KRAB} e un dominio
repressore \emph{SSX} \emph{SSXRD}, poi si trova un dominio \emph{PR/SET} circondato da una \emph{pre-SET} zinc knuckle e un \emph{post-SET} zinc finger. Si trova un lungo vettore di 
zinc finger \emph{C2H2} $C$ terminale. Si trovano diverse varianti della proteina che \`e in grado di evolvere rapidamente. Il dominio \emph{PR/SET} \`e circondato da zinc knuckle e 
finger in quanto potrebbero contribuire al legame con substrato e cofattore o essere coinvolti con l'interazione di altre proteine. 
\paragraph{\emph{PRDM9} organizza gli hotspot dei nucleosomi e limita la migrazione delle giunzioni di Holliday}
Nei mammiferi la ricombinazione genica durante la meiosi \`e limitata a un piccolo insieme di regioni di $1$-$2$ kilobasi dette hotspots. La loro locazione \`e determinata dai domini 
di zinc finger di \emph{PRDM9} che lega il DNA e trimetila l'istone \emph{H3}. Questo prepara ad una \emph{DSB} e scambio reciproco di DNA tra cromatidi formando giunzioni di Holliday. 
Si nota come il legame con \emph{PRDM9} riorganizza i nucleosomi in pattern simmetrici creando una regione estesa senza nucleosomi. Queste regioni sono centrate da un motivo legante a
\emph{PRDM9}. Si nota anche come \emph{DSB} si trovi al centro di queste regioni. Pertanto combinando questi risultati si trova che il crossing-over \`e ristretto a regioni marcate 
da \emph{H3K4me3}.
\subsubsection{La separazione casuale dei cromosomi omologhi}
Un altro meccanismo della meiosi che contribuisce alla variazione genetica \`e la distribuzione casuale dei cromosomi in anafase $I$: ogni coppia di omologhi si allinea e separa casualmente, pertanto
i cromosomi vengono divisi nelle cellule figlie indipendentemente dalla loro origine paterna o materna. Questo processo viene anche detto assortimento indipendente. 
\subsection{La meiosi nel ciclo vitale di animali e piante}
\subsubsection{Animali}
\paragraph{Maschio}
La produzione dei gameti nell'animale maschio si dice spermatogenesi e avviene nei testicoli. Le cellule germinali primordiali si dividono per via mitotica producendo spermatogoni, cellule diploidi che
possono compiere diversi cicli di mitosi o iniziare la meiosi. La cellula ora viene detta spermatocita primario che completa la meiosi $I$ producendo due spermatociti secondari aploidi che entrano in
meiosi $II$ producendo due spermatidi aploidi. 
\paragraph{Femmina}
La produzione dei gameti nell'animale femmina si dice oogenesi. Nelle ovaie le cellule germinali primordiali si dividono per produrre gli oogoni che possono subire cicli ripetuti di mitosi o entrare in
meiosi. Quando lo fanno vengono detti oociti primari che si dividono. Nell'oogenesi la citocinesi \`e diseguale: gran parte del citoplasma viene localizzato in una delle due cellule aplodi, l'oocita
secondario, mentre l'altro ora corpo polare pu\`o ancora dividersi o non farlo. L'oocita secondario completa la mesiosi II con citocinesi diseguale formando l'ovulo e il secondo corpo polare. Solo
l'ovulo pu\`o essere fecondato mentre i corpi polari si dissolvono. Nei mammiferi inoltre la formazione dei gameti pu\`o avvenire solo per pochi anni. L'oogenesi inizia prima della nascita e quando si
d\`a origine agli oociti primari la meiosi si interrompe. Dopo l'ovulazione i livelli ormonali stimolano uno o pi\`u oociti primari a riprendere la meiosi e la seconda divisione meiotica viene rinviata 
fino al contatto con lo spermatozoo. 
\subsubsection{Piante}
Molte piante possiedono un ciclo di vita che alterna tra due fasi distinte che corrispondono a generazioni diverse: una fase a sporofito diploide e una a gametofito aploide. Lo sporofito produce spore
aploidi attraverso meiosi mentre il gametofito gameti aploidi attraverso mitosi. Questo ciclo vitale viene anche detto alternanza di generazione i prodotti della meiosi sono detti 
spore che subiscono diverse divisioni mitotiche per produrrei i gameti. Nelle angiosperme lo sporofito \`e la parte visibile della pianta mentre il gametofito \`e formato da poche cellule aploidi nello
sporofito. Il fiore contiene le strutture riproduttive. Lo stame conteiene i microsporociti che va incontro a meiosi per produrre microspore aploidi. Le microspore subiscono mitosi per produrre un 
granulo pollinico immaturo e uno di questi fornisce le istruzioni per la crescita di un tubulo pollinico, mentre l'altro o nucleo generativo si divide per produrre due cellule speratiche. 
Insieme formano il gamete maschile. L'ovario invece contiene i megasporociti, cellule diploidi che durante la mitosi producono $4$ megaspore aploidi di cui solo $1$ sopravvive. Questa si divide 
mitoticamente per tre volte creando il gametofito femminile o sacco embrionale. Quando cellula uovo e le cellule spermatiche si uniscono costituiscono un endosperma triploide.
