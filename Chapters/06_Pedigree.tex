\chapter{Pedigree}
I principi di Mendel sulla segregazione dei geni sono validi per tutti gli eucarioti come l'uomo. Lo studio dell'ereditariet\`a dei caratteri genetici dell'uomo \`e complicata dal
fatto che non possono essere effettuati incroci programmatici. L'accertamento del tipo di ereditariet\`a avviene nell'ambito dei caratteri monogenici. Si rende necessario analizzare
i caratteri mediante lo studio degli alberi genealogici esaminando la comparsa del carattere in individui che lo manifestano chiaramente. 
\section{Simboli dei pedigree}
\begin{multicols}{2}
	\begin{itemize}
		\item Quadrato: sesso maschile.
		\item Cerchio: sesso femminile.
		\item Rombi: sesso non noto (figlio che deve nascere o non specificato).
		\item Colore pieno: individuo affetto.
		\item Colore vuoto: individuo sano.
		\item Punto nel simbolo: portatore obbligato.
		\item Barra verticale nel simbolo: portatore asintomatico.
		\item Numero nel simbolo: individui multipli.
		\item Barra orizzontale sul simbolo: individuo deceduto.
		\item Freccia con $P$: probando, primo membro affetto della famiglia ad essere preso in considerazione.
		\item ``?'' nel simbolo: storia familiare dell'individuo non nota. 
		\item Linee orizzontali tra simboli: accoppiamento.
		\item Linee verticali tra figli: rapporto genitore-figli.
		\item Linea che si biforca: gemelli, triangolo se gemelli.
		\item Doppia linea orizzontaleUnione tra individui imparentati, consanguineit\`a.
		\item Parentesi quadre e linea tratteggiata: adozione.
		\item Numero romano: generazione. 
	\end{itemize}
\end{multicols}

