\chapter{Interazioni gene ambiente}
Interessa studiare il genotipo, il meccanismo contenuto nell'informazione. Il percorso da genotipo a fenotipo \`e accidentato e si propongono modulatori dell'espressione legati a fattori
ambientali che possono essere esterni (temperatura conformazione della proteina con mutazione, coniglio himalaiano), l'ambiente interno in un organismo pluricellulare interazioni tra
cellule, cellule matrice, sostanze rilasciate che raggiungono altre cellule. Interazioni eterotipiche come batteri o funghi del microbioma che comunicano con le cellule umane. Un
altro effetto legato all'ambiente \`e il background genetico: un gene principale la cui funzione \`e influenzata da molti altri geni che ne modificano l'espressivit\`a. Pi\`u geni
controllano lo stesso carattere come la fenilchetomuria modulata dagli enzimi che producono il co-fattore che aiutano l'enzima a funzionare. Ci sono influenze nel percorso che sono
difficili da inquadrare nelle altre caratteristiche ma che possono essere fenomeni stocastici. Il caso pu\`o influenzare penetranza ed espressivit\`a. Si nota come il fenotipo 
\`e una combinazione di diverse influenze.
\section{Caratteri autosomici limitati/influenzati dal sesso}
Nei fenomeni di penetranza ed espressivit\`a alcuni alleli autosomici potrebbero essere influenzati dal contesto genomico: dalla presenza di un numero di cromosomi $X$ e questa 
influenza dei cromosomi sessuali e dei loro geni pu\`o essere leggera o forte: caratteri limitati dal sesso. Alcuni tratti $maschi:femmine$ sono labbro leporino $2:1$, gotta $8:1$, 
artite reumatoide $1:3$, osteoporosi $1:3$, lupus eritematoso sistemico $1:9$. 
\subsection{Esempi}
\subsubsection{Barba nelle capre}
Si nota come $BB$ e $bb$. Si tratta di alleli comune. Le linee pure e gli eterozigoti si nota come la barba \`e presente solo nei maschi e non nelle femmine: la presenza del tratto 
fenotipico \`e dominante nel maschio ma recessiva nella femmina. 
\subsubsection{Calvizie}
La calvizie  nell'uomo funziona come la barba nelle capre con omozigosi $BB$ nell'uomo e $bb$ nella donna. Gli individui eterozigoti sono calvi: dominante nel maschio, recessivo nel
sesso femminile. Non \`e limitato al sesso ma \`e da esso influenzato e si manifesta come tratto dominante nel maschio e recessivo nella donna.
\subsection{Limitato al sesso}
Il gene autosomico in numero di coppie uguale negli individuo ma limitato al sesso: geni localizzati su autosomi controllano un carattere che si manifesta in un solo sesso come la
produzione del latte, comparsa delle corna nella pecora e distribuzione dei peli facciali nell'uomo. Piumaggio da gallo nel pollo: tratto autosomico recessivo limitato ai maschi. Un
altro esempio \`e la pubert\`a precoce circostricca al maschio, una mutazione dominante del recettore \emph{LH}. 
\subsubsection{Occhio di Drosophila}
L'occhio di Drosophila \`e composto da molte unit\`a ottiche in prossimit\`a o omatidi e si nota come il numero cambia in base alla temperatura dello sviluppo della larva e si 
approssima una retta che mostra come la complessit\`a dell'occhio varia: a $15$ gradi anche pi\`u di mille ma a $30$ sono $250$ unit\`a ottiche in meno. Questa cosa \`e un esempio di 
una norma di reazione, la genetica \`e la stessa, l'effetto di una variabile esterna pu\`o avere su un programma genetico comune a tutti gli individui di una popolazione. Cosa
simile per le ali con temperatura e lunghezza delle ali: a $18$ gradi sono molto piccole mentre a temperature pi\`u alte sono ali normali. Si deve avere l'allele ali vestigiali 
autosomico. La temperatura influenza l'aspetto fenotipico elgato alla genetica. A $29$ gradi non si apprezza un difetto mentre a $18$ si vede una notevole disfunzione nella produzione di
ali. Si nota anche un dimorfismo con ruolo nella temperatura minore nella femmina. Un altro esempio dell'effetto della temperatura in Drosophila si ha un mutante shibire vitale 
tra i $18$ e i $28$ gradi, ma se la temperatura arriva a $29$ gradi con effetto soglia si ha paralisi reversibile e se non lo si porta a temperatura pi\`u confortevole l'insetto muore. 
Questo mutante \`e un gene che codifica una proteina essenziale per la trasmissione nervosa e l'allele mutato \`e sensibile alla temperatura. 
\subsubsection{Conchiglie}
Si nota come le conchiglie hanno un orientamento destrorso o sinistrorso. L'avvolgimento destrorso \`e il risultato di un allele autosomico dominante e si incrocia con uno per 
l'avvolgimento sinistrorso. Si nota come la $F_1$ eterozigote \`e sinistrorsa, questo \`e un tratto monogenico con allele dominante per il sinistrorso. Si analizza ora $F_2$ 
autofecondata da $F_1$ che \`e completamente $F_2$. La scomparsa del dominante in $F_2$ con due incroci diversi: $s^+s^+$ femminile destrorsa e il compagno $ss$ sinistrorso, il 
complementare di quello visto prima: sesso opposto al fenotipo. In $F_1$ sono tutti destrorsi. Si nota pertanto come il fenotipo dell'individuo non \`e associato al suo genotipo ma a
quello della madre. La genetica della madre predice il fenotipo dei figli con $S^+$ dominante e faccia andare a destra. Se fosse cos\`i le madri $F_2$ avrebbero tutti figli equivalenti
destrorsi. La madre $F_2$ $ss$ dovrebbe far cambiare il fenotipo dei figli. Questo \`e un effetto genetico materno, un'eccezione in cui il fenotipo della progenie \`e stabilito dal
genotipo della madre, dipendente dal citoplasma della cellula uovo: il genotipo della madre condiziona tutte le cellule somatiche e germinali dell'individuo. Quelle germinali femminile
sono cellula uovo e il gamete femminile \`e particolarmente grande con ruolo del citoplasma ruolo importante nelle prime divisioni dello zigote. Pertanto il fenotipo dei figli 
\`e legato alle dicisioni del citoplasma della cellula uovo nel guidare il posizionamento del fuso mitotico nelle prime fasi delle divisione dello zigote influenzando l'avvitamento 
dell'organismo. 
\subsubsection{Ereditariet\`a extranucleare}
Variazioni fenotipo-genotipo. \`E un esempio di ereditariet\`a uniparentale: mitocondri e cloroplasti vengono ereditati da un solo genitore, tipicamente la madre. Le cellule che si 
dividono, i mitocondri hanno genomi diversi. L'eterozigosi nel genoma mitocondriale viene detta eteroplasmia. Durante la divisione cellulare i mitocondri vengono divisi in parti 
uguali nel citoplasma. Per caso in una divisione cellulare i mitocondri sono separati per tipo e si passa da una condizione di eteroplasmia a una di omoplasmia segredando mitocondri 
con una caratteristica da una apre e l'altra dall'altra con effetto fenotipico. 
\paragraph{Eredit\`a citomplasmatica nella pianta bella di notte}
Fenomeno fenotipico legato alla colorazione di foglie specializzate apicali. L'esperimento porta a concludere una preferenzialit\`a nella trasmissione di un tratto fenotipico, la 
colorazione delle piante. Nella colonna verticale si osserva il seme o macrospora e dall'altra si vedono le caratteristiche della pianta del polline per fecondare. Lo schema guida
alla conclusione che sembra evidente ed applicabile a tutte le $F_1$ il fenotipo \`e guidato esclusivamente dal fenotipo della pianta da cui \`e stato preso la macrospora: foglie bianche
progenie bianca indipendentemente dall'origine del polline. Lo stesso per il verde. Lo screziato invece pu\`o presentare fenotipo bianco, verde o screziato. In questo caso 
l'ambiente \`e inteso come il contenuto citoplasmatico delle cellule germinali e si tratta della presenza nel citoplasma di organelli con informazione genetica: i cloroplasti. La
macrospora con porzione apicale bianca viene indicata come cellula aploide con la diversa produzione della clorofilla. Il polline non contribuisce elementi genetici citoplasmatici. 
Il caso variegato \`e quello pi\`u interessante in quanto partendo dalla pianta variegata ci pu\`o essere un omoplasmia del tratto bianco, verde e altre siano eteroplasmiche con la 
compresenza di due tipi di genomi del cloroplasto e a seconda della spora utilizzata si ha piante omogeneamente verdi o bianche oppure continueranno a manifestare la screziatura, una
riduzione all'omoplasmia nelle cellule somatiche durante le divisioni mitotiche del soma. 
\paragraph{Confermare l'ipotesi}
Gli esperimenti sono rimasti come la storia dei mutanti poki di Neurospora. Il sistema sperimentale coniuga la possibilit\`a di seguire l'ereditariet\`a di un gene nucleare simboleggiato
dal colore dei puntini in eterozigosi rossa nero di informazione nucleare e viene coniugato con il cambio della dimensione delle spore aploidi prodotte dalla meiosi di neurospora: un
mutante produce un'equivalente di una macrospora. Facendo un incrocio tra una neurospora poki che produce le spore grandi e una che fa spore di dimensione normale si ottiene un diploide
dove il contributo citoplasmatico \`e quasi esclusivo della macrospora. L'esperimento non \`e fisiologico ma fenocopia un'ereditariet\`a uniparentale del materiale citoplasmatico. 
Si nota nell'asco a valle della fecondazione con neurospore diploidi a cui si stimola il passaggio in meiosi con produzione di astuccio con $8$ cellule e confrontando in parallelo
i due setup in parallelo si ha l'atteso per la segregazione del biomarker di informazione nucleare ma il colore che indica il fenotipo citoplasmatico legato a una variazione nel 
citocromo (gene mitocondriale) questo \`e dettato nel $100\%$ dei casi dalla natura della macrospora utilizzata nella fecondazione. Il citoplasma condiziona il fenotipo legato alla
condizione del citocromo mentre il gene nucleare segue il pattern di segregazione atteso. Il citoplasma viene da solo un genitore. 
\section{Probabilit\`a}
Il ruolo della probabilit\`a, di eventi genetici causali che vengono ragguppati di come genotipo e fenotipo possono essere ditanziati da una serie di fenomeni di modulazione. L'influenza
di eventi genetici casuali. Un esempio \`e legato alla patologia dove il contributo dell'evoluzione del genoma a livello somatico \`e essenziale per la manifestazione genotipica: 
sindrome di predisposizione a una neoplasia con penetranza incompleta. Un fenomeno casuale pu\`o guidare la relazione genotipo fenotipo. Il tessuto epiteliale organizzato con cellule
messe in modo ordinato con cellule polarizzate ed ancorate a formare una barriera, in fondo si nota una morfologia alterata con eccessiva proliverazione ed espressione di nuovi 
marcatori. Tra questi due estremi si nota lesione $3$ cancerosa e tumore iniziale con l'acquisizione di caratteristiche. Si propone che il passaggio da una cellula normale o tumorale
aggressiva \`e un processo che richiede diversi eventi che pu\`o partire da una presenza di eventi germinali ereditati ma che non sono sufficienti per la conversione fenotipica. 
Questa richiede l'acquisizione di eventi somatici, classificata sotto mutazioni in geni. Se \`e vero che l'acquisizione di eventi somatici \`e legata a mutazioni si propone un fenomeno
stocastico che influenza l'evoluzione fenotipica. 
\section{Pleiotropia}
Si dice pleiotropia il fenomeno per cui un singolo gene determina un certo numero di caratteri apparentemente non correlati.
\subsection{Singed - mutante di Drosophila}
La funzione dell'allele singed che \`e un omologo della fascina coinvolto nella regolazoine della formazione di alcuni elementi del citoscheletro. \`E considerato un esempio di 
pleiotropia in quanto sue mutazioni hanno un effetto visibile sulla formazione di una caratteristica somatica esterna, la formazione e il numero e aspetto delle setole, ma i mutanti 
singed oltre all'alterazione nella formazione delle setole sono poco fertili e le uova non sembrano essere in grado di portare alla formazione di zigoti con sviluppo embrionale 
selvatico. Si tratta di due fenotipi lontani. Incroci e studi di complementazione puntano al fatto che si tratta dello stesso gene e dello stesso allele. Per ricondurre questi due
fenotipi alla funzione di un solo allele difettivo. La spiegazione la si trova nel ruolo funzionale del gne singed omologo della fascina che porta alla formazione di fasci di actina
che controllano struttura del citoscheletro: i bundle di actina sono importanti per organizzazione e formazione delle setole a livello somatico ma questa stessa funzione \`e importante
anche nelle camere importanti per la maturazione dell'uovo: la stessa funzione genica si esplica in contesti diversi grazie a un unico meccanismo molecolare in cui le cellule 
organizzano strutture utili per il movimento in cellule specializzate. Questo \`e un fenomeno di pleiotropia: una mutazione e due fenotipi apparentemente non facilmente collegabili. 
\subsection{Pleiotropia nell'anemia falciforme}
Si nota come la mutazione nel gene della beta globina in posizione codificante con amminoacido che cambia e si vede che un acido glutammico diventa valina in posizione $6$ causando 
l'anemia falciforme. L'ossigeno viene trasportato in modo aberrante con difetto di apporto di ossigeno, legando un evento genetico con difetto proteico con conseguenza legata a tale
difetto. SI nota come il difetto porta a diversi fenotipi: variano da difetti di crescita, a cognitivi, dolore, deformazione nelle ossa, ittero, problemi renali. QUesti fenotipi 
apparentemente non collgati sono collegabili e andando a ritroso si nota come a partire dal difetto genetico e alterazione del trasporot dell'ossigeno si arriva ad avere difetti a 
livello epatico, cistifellea, cuore, ossa e viene presentato come esempio eclatante della complessit\`a di fenomeni pleiotropici. 
\subsection{Pleiotropia antagonistica}
Notando l'et\`a sull'asse $x$ e la forza di selezione sull'asse $y$ si ha una riduzione della forza di selezione che parte molto alta e scende gradatamente fino a diventare zero. 
Questo grafico indica la pleiotropia antagonistica. La forza di selezione cambia con l'et\`a: tra i $15$ e i $35$ anni. Questo vuole provare a convincere che l'effetto di pressione 
selettiva di alleli \`e legato a quando e come gli alleli hanno un impatto sulla trasmissione successiva. Se l'allele lo dimostra precocemente con effetto fenotipico importante 
l'allele ha un effetto forte sulla selezione impedendo che la possibilit\`a abbia una progenie azzerando la trasmissione. Se l'effetto si manifesta successivamente con l'et\`a diminuisce
tale forza di selezione. Alleli con effetti pleiotropici e la conseguenza fenotipica \`e antagonistica in funzione con l'et\`a: alleli con successo produttivo in giovinezza ma che 
sono associati a un pi\`u rapido deterioramento in tarda et\`a potrebbero essere selezionati e trasmessi nella popolazione. 
