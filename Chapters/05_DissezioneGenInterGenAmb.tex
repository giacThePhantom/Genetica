\chapter{Interazioni gene ambiente}
\section{Probabilit\`a}
Il ruolo della probabilit\`a, di eventi genetici causali che vengono ragguppati di come genotipo e fenotipo possono essere ditanziati da una serie di fenomeni di modulazione. L'influenza
di eventi genetici casuali. Un esempio \`e legato alla patologia dove il contributo dell'evoluzione del genoma a livello somatico \`e essenziale per la manifestazione genotipica: 
sindrome di predisposizione a una neoplasia con penetranza incompleta. Un fenomeno casuale pu\`o guidare la relazione genotipo fenotipo. Il tessuto epiteliale organizzato con cellule
messe in modo ordinato con cellule polarizzate ed ancorate a formare una barriera, in fondo si nota una morfologia alterata con eccessiva proliverazione ed espressione di nuovi 
marcatori. Tra questi due estremi si nota lesione $3$ cancerosa e tumore iniziale con l'acquisizione di caratteristiche. Si propone che il passaggio da una cellula normale o tumorale
aggressiva \`e un processo che richiede diversi eventi che pu\`o partire da una presenza di eventi germinali ereditati ma che non sono sufficienti per la conversione fenotipica. 
Questa richiede l'acquisizione di eventi somatici, classificata sotto mutazioni in geni. Se \`e vero che l'acquisizione di eventi somatici \`e legata a mutazioni si propone un fenomeno
stocastico che influenza l'evoluzione fenotipica. 
