\chapter{Pedigree}
I principi di Mendel sulla segregazione dei geni sono validi per tutti gli eucarioti come l'uomo. Lo studio dell'ereditariet\`a dei caratteri genetici dell'uomo \`e complicata dal
fatto che non possono essere effettuati incroci programmatici. L'accertamento del tipo di ereditariet\`a avviene nell'ambito dei caratteri monogenici. Si rende necessario analizzare
i caratteri mediante lo studio degli alberi genealogici esaminando la comparsa del carattere in individui che lo manifestano chiaramente. 

\section{Simboli dei pedigree}

\begin{multicols}{2}
	\begin{itemize}
		\item Quadrato: sesso maschile.
		\item Cerchio: sesso femminile.
		\item Rombi: sesso non noto (figlio che deve nascere o non specificato).
		\item Colore pieno: individuo affetto.
		\item Colore vuoto: individuo sano.
		\item Punto nel simbolo: portatore obbligato.
		\item Barra verticale nel simbolo: portatore asintomatico.
		\item Numero nel simbolo: individui multipli.
		\item Barra orizzontale sul simbolo: individuo deceduto.
		\item Freccia con $P$: probando, primo membro affetto della famiglia ad essere preso in considerazione.
		\item ``?'' nel simbolo: storia familiare dell'individuo non nota. 
		\item Linee orizzontali tra simboli: accoppiamento.
		\item Linee verticali tra figli: rapporto genitore-figli.
		\item Linea che si biforca: gemelli, triangolo se gemelli.
		\item Doppia linea orizzontale: unione tra individui imparentati, consanguineit\`a.
		\item Parentesi quadre e linea tratteggiata: adozione.
		\item Numero romano: generazione. 
	\end{itemize}
\end{multicols}

\section{Analisi del pedigree}
Il numero limitato di figli di molte famiglie umane rende impossibile individuare in un singolo pedigree chiari rapporti di tipo mendeliano. Si deve pertanto interpretare i dati a disposizione in modo da
chiarire la natura del tratto preso in considerazione.

\subsection{I caratteri autosomici recessivi}
I caratteri autosomici recessivi compaiono con la stessa frequenza in entrambi i sessi (a meno di differenze nella penetranza) e si manifestano unicamente quando l'individuo eredita un allele per genitore.
Se il carattere \`e raro la maggior parte dei genitori \`e eterozigote e non affetta: il carattere sembra saltare generazioni. 
Pu\`o essere trasmesso per diverse generazioni senza che compaia nel pedigree.
Quando entrambi i genitori sono eterozigoti ci si aspetta che circa un quarto della progenie esprima il carattere.
Nel caso in cui entrambi i genitori sono affetti da un tratto autosomico recessivo questo \`e presente in tutta la progenie. 
Se il tratto \`e raro gli individui sono tipicamente omozigoti per l'allele normale. 
Quando una persona con il carattere si unisce a un partner esterno alla famiglia nessuno dei figli manifesta il tratto, ma ne saranno portatori eterozigoti. 

\subsection{Caratteri autosomici dominanti}
I caratteri autosomici dominanti si manifestano con la stessa frequenza in entrambi i sessi ed entrambi sono in grado di trasmetterli alla progenie: ogni individuo con un carattere dominante deve aver ereditato l'allele da almeno un genitore.
I caratteri dominanti pertanto non saltano le generazioni a meno che un individuo acquisisca il carattere come mutazione \emph{de novo} o esso ha una penetranza ridotta.
Se un allele dominante \`e raro la maggior parte degli individui che lo manifestano \`e eterozigote.
Con un genitore eterozigote e l'altro privo $\frac{1}{2}$ della progenie presenta il carattere.
Con entrambi i genitori eterozigoti $\frac{3}{4}$ della progenie presenta il carattere. 
Individui sani non trasmettono il carattere se ha penetranza completa.

\subsection{Caratteri recessivi legati al cromosoma $\mathbf{X}$}
I caratteri recessivi legati a $X$ appaiono pi\`u frequentemente nei maschi in quanto hanno bisogno di ereditare solo una copia dell'allele che esprime il carattere, mentre le femmine ne devono ereditare due.
I maschi che lo possiedono sono nati da madri non affette ma portatrici in quanto ereditano da lei l'allele.
In quanto la trasmissione avviene da femmina non affetta a maschio affetto a femmina non affetta tende a saltare le generazioni. 
Una donna eterozigote avr\`a met\`a dei figli affetta e met\`a delle figlie portatrice non affetta. 
I caratteri recessivi legati a $X$ non sono trasmessi da padre a figlio in quanto un figlio eredita dal padre il cromosoma $Y$ e non $X$.
Tutte le figlie di un uomo affetto saranno portatrici e una donna per presentare il carattere deve essere omozigote per esso e sar\`a presente in tutti i figli maschi. 

\subsection{Caratteri dominanti legati al cromosoma $\mathbf{X}$}
I caratteri dominanti legati a $X$ compaionon in maschi e femmine, ma con maggiore frequenza nelle seconde. 
Ogni individuo deve avere un genitore affetto e i caratteri non saltano le generazioni.
I maschi affetti li trasmettono a tutte le figlie ma non ai figli. 
Le donne affette se eterozigoti trasmettono il carattere a met\`a dei figli e met\`a delle figlie. 
I maschi ereditano il carattere dominante solo dalla madre, mentre le femmine uno per ogni genitore. 

\subsection{Caratteri legati al cromosoma $\mathbf{Y}$}
I caratteri legati a $Y$ presentano un modello di eredit\`a caratteristico: sono affetti solo i maschi e il carattere si trasmette da padre a figlio.
Un maschio affetto ha progenie maschile completamente affetta. 
Che nel cromosoma $Y$ umani si trovi poca quantit\`a genetica come la mascolinit\`a.
Inoltre essendo che ogni maschio possiede solo un cromosoma $Y$ i caratteri legati ad esso non sono n\`e dominanti n\`e recessivi.

