\chapter{Pedigree}
I principi di Mendel sulla segregazione dei geni sono validi per tutti gli eucarioti come l'uomo. Lo studio dell'ereditariet\`a dei caratteri genetici dell'uomo \`e complicata dal
fatto che non possono essere effettuati incroci programmatici. L'accertamento del tipo di ereditariet\`a avviene nell'ambito dei caratteri monogenici. Si rende necessario analizzare
i caratteri mediante lo studio degli alberi genealogici esaminando la comparsa del carattere in individui che lo manifestano chiaramente. 

\section{Simboli dei pedigree}
\begin{multicols}{2}
	\begin{itemize}
		\item Quadrato: sesso maschile.
		\item Cerchio: sesso femminile.
		\item Rombi: sesso non noto (figlio che deve nascere o non specificato).
		\item Colore pieno: individuo affetto.
		\item Colore vuoto: individuo sano.
		\item Punto nel simbolo: portatore obbligato.
		\item Barra verticale nel simbolo: portatore asintomatico.
		\item Numero nel simbolo: individui multipli.
		\item Barra orizzontale sul simbolo: individuo deceduto.
		\item Freccia con $P$: probando, primo membro affetto della famiglia ad essere preso in considerazione.
		\item ``?'' nel simbolo: storia familiare dell'individuo non nota. 
		\item Linee orizzontali tra simboli: accoppiamento.
		\item Linee verticali tra figli: rapporto genitore-figli.
		\item Linea che si biforca: gemelli, triangolo se gemelli.
		\item Doppia linea orizzontale: unione tra individui imparentati, consanguineit\`a.
		\item Parentesi quadre e linea tratteggiata: adozione.
		\item Numero romano: generazione. 
	\end{itemize}
\end{multicols}

\section{Analisi del pedigree}
Il numero limitato di figli di molte famiglie umane rende impossibile individuare in un singolo pedigree chiari rapporti di tipo mendeliano. Si deve pertanto interpretare i dati a disposizione in modo da
chiarire la natura del tratto preso in considerazione.

	\subsection{I caratteri autosomici recessivi}
	I caratteri autosomici recessivi compaiono con la stessa frequenza in entrambi i sessi (a meno di differenze nella penetranza) e si manifestano unicamente quando l'individuo eredita un allele per genitore.
	Se il carattere \`e raro la maggior parte dei genitori \`e eterozigote e non affetta: il carattere sembra saltare generazioni. 
	Pu\`o essere trasmesso per diverse generazioni senza che compaia nel pedigree.
	Quando entrambi i genitori sono eterozigoti ci si aspetta che circa un quarto della progenie esprima il carattere.
	Nel caso in cui entrambi i genitori sono affetti da un tratto autosomico recessivo questo \`e presente in tutta la progenie. 
	Se il tratto \`e raro gli individui sono tipicamente omozigoti per l'allele normale. 
	Quando una persona con il carattere si unisce a un partner esterno alla famiglia nessuno dei figli manifesta il tratto, ma ne saranno portatori eterozigoti. 

	\subsection{Caratteri autosomici dominanti}
	I caratteri autosomici dominanti si manifestano con la stessa frequenza in entrambi i sessi ed entrambi sono in grado di trasmetterli alla progenie: ogni individuo con un carattere dominante deve aver ereditato l'allele da almeno un genitore.
	I caratteri dominanti pertanto non saltano le generazioni a meno che un individuo acquisisca il carattere come mutazione \emph{de novo} o esso ha una penetranza ridotta.
	Se un allele dominante \`e raro la maggior parte degli individui che lo manifestano \`e eterozigote.
	Con un genitore eterozigote e l'altro privo $\frac{1}{2}$ della progenie presenta il carattere.
	Con entrambi i genitori eterozigoti $\frac{3}{4}$ della progenie presenta il carattere. 
	Individui sani non trasmettono il carattere se ha penetranza completa.

	\subsection{Caratteri recessivi legati al cromosoma $\mathbf{X}$}
	I caratteri recessivi legati a $X$ appaiono pi\`u frequentemente nei maschi in quanto hanno bisogno di ereditare solo una copia dell'allele che esprime il carattere, mentre le femmine ne devono ereditare due.
	I maschi che lo possiedono sono nati da madri non affette ma portatrici in quanto ereditano da lei l'allele.
	In quanto la trasmissione avviene da femmina non affetta a maschio affetto a femmina non affetta tende a saltare le generazioni. 
	Una donna eterozigote avr\`a met\`a dei figli affetta e met\`a delle figlie portatrice non affetta. 
	I caratteri recessivi legati a $X$ non sono trasmessi da padre a figlio in quanto un figlio eredita dal padre il cromosoma $Y$ e non $X$.
	Tutte le figlie di un uomo affetto saranno portatrici e una donna per presentare il carattere deve essere omozigote per esso e sar\`a presente in tutti i figli maschi. 

	\subsection{Caratteri dominanti legati al cromosoma $\mathbf{X}$}
	I caratteri dominanti legati a $X$ compaionon in maschi e femmine, ma con maggiore frequenza nelle seconde. 
	Ogni individuo deve avere un genitore affetto e i caratteri non saltano le generazioni.
	I maschi affetti li trasmettono a tutte le figlie ma non ai figli. 
	Le donne affette se eterozigoti trasmettono il carattere a met\`a dei figli e met\`a delle figlie. 
	I maschi ereditano il carattere dominante solo dalla madre, mentre le femmine uno per ogni genitore. 

	\subsection{Caratteri legati al cromosoma $\mathbf{Y}$}
	I caratteri legati a $Y$ presentano un modello di eredit\`a caratteristico: sono affetti solo i maschi e il carattere si trasmette da padre a figlio.
	Un maschio affetto ha progenie maschile completamente affetta. 
	Che nel cromosoma $Y$ umani si trovi poca quantit\`a genetica come la mascolinit\`a.
	Inoltre essendo che ogni maschio possiede solo un cromosoma $Y$ i caratteri legati ad esso non sono n\`e dominanti n\`e recessivi.

\section{Esempi}
	
	\subsection{Fibrosi cistica}
	La fibrosi cistica \`e un carattere autosomico recessivo frequente nella popolazione europea.
	Il gene responsabile quando alterato \`e nella regione $7q31.2-31.3$, nel braccio lungo e sottobanda $2$. 
	Mappato per clonaggio posizionale.

		\subsubsection{Funzione del gene}
		Il gene produce una proteina lunga, un canale transmembrana con due domini collegati da una regione citoplasmatica.
		La proteina scambia ioni, controllando il bilancio di cloro.
		La mutazione pi\`u frequente nella popolazione europea e nord-americana \`e una mutazione che si tratta di una delezione di $3$ nucleotidi e rimuove un amminoacido $508$ in una regione che
		non sembra fondamentale per la funzionalit\`a della proteina. 
		La mutazione sembra impedire la localizzazione della proteina.
		Negli alveoli polmonari avviene una diminuzione dei fluidi che devono aggiungersi alla secrezione delle ghiandole aumenta il rischio di infezione dovuto a pseudo-monas. 


		\paragraph{Altre mutazioni}
		In caso di sospetto si pu\`o valutare un pannello delle mutazioni pi\`u frequenti: pazienti eterozigoti compositi, due alleli con due mutazioni diverse.

	\subsection{Sindrome di Marfan}
	La sindrome di Marfan \`e un carattere autosomico dominante. 
	\`E un difetto del tessuto connettivo che colpisce il sistema scheletrico e il sistema cardiovascolare.

		\subsubsection{Prodotto genico - fibrillina}
		Il tessuto connettivo intorno alla base dell'aorta si indebolisce portando ad allargamento e rottura. 
		Pu\`o capitare che ci sia una rottura dell'aorta traumatica non recuperabile.
		La fibrillina \`e una delle componenti che rendono i tessuti connettivi elastici.
		
	\subsection{Fenilchetonuria \emph{PKU}}
	La \emph{PKU} viene ereditata come un carattere autosomico recessivo. 

		\subsubsection{Funzione del gene}
		Gli alleli normali nel locus \emph{PAH} codificano una \emph{PAH} funzionale che converte la fenilanalina in tirosina. 
		Alleli mutanti invece codificano una \emph{PAH} difettosa che non compie la conversione e alti livelli di fenilanalina causano ritardo mentale, eczema e cute chiara. 

		\subsubsection{Differenze tra omozigoti ed eterozigoti}

			\paragraph{Omozigoti}
			Gli omozigoti mostrano un'attivit\`a molto bassa dell'enzima \emph{PAH} e alte concentrazioni di fenilanalina.
			
			\paragraph{Eterozigoti}
			Gli eterozigoti mostrano livelli intermedi do attivit\`a enzimatica ma una bassa concentrazione plasmatica di fenilanalina. 

			\paragraph{Conclusione}
			La \emph{PKU} esibisce dominanza incompleta riguardo l'attivit\`a enzimatica, ma in riferimento alla concentrazione plasmatica di fenilanalina si mostra recessivado attivit\`a enzimatica ma una bassa concentrazione plasmatica di fenilanalina. 
			
			\paragraph{Conclusione}
			La \emph{PKU} esibisce dominanza incompleta riguardo l'attivit\`a enzimatica, ma in riferimento alla concentrazione plasmatica di fenilanalina si mostra recessiva.
	
		\subsubsection{Eterozigoti compositi}
		A livello del locus \emph{PAH} si incontrano molti alleli diversi e alleli mutati mostrano ampia variabilit\`a di difetti nell'attivit\`a enzimatica con effetti sulla gravit\`a dei sintomi.

		\subsubsection{Caratteristiche cliniche}
		\`E importante una diagnosi precoce in quanto il ritardo mentale \`e curabile attraverso una dieta particolare. 
		Altre caratteristiche sono pigmentazione leggera, postura particolare e epilessia. 

		\subsubsection{Fenilchetonuria materna}
		La nascita di ritardo mentale nella progenie di madri omozigoti \`e un esempio di malattia genetica basata sul genotipo della madre. 
		Il danno \`e aggravato da processi placentari che funzionano per mantenere un alto livello di amminoacidi nel feto. 
		Si \`e notato come la frequenza di anormalie congenitali aumenta aumentando i livelli materni di fenilanalina. 

	\subsection{Piebaldismo}
	Il piebaldismo \`e un fenotipo raro presente nell'uomo con variabilit\`a dell'espressione fenotipica.
	L'allele dominante interferisce con la migrazione dei melanociti: il \emph{c-Kit} protooncogene.

	\subsection{Incapacit\`a di sentire l'amaro}
	La variante polimorfica che determina l'incapacit\`a di sentire l'amaro \`e recessiva ma il carattere non salta generazioni.
	
	\subsection{Famiglia cinese colpita da corea di Hungtinton}
	La penetranza della patologia \`e legata all'et\`a e raggiunge il $100\%$ a $80$ anni.
	La patologia \`e legata a un allele, ma non \`e una mutazione puntiforme.
	Il fenomeno legato \`e l'espansione di un elemento ripetuto di $CAG$, che essendo in una porzione codificante codifica per glutammina.
	Nel gene dell'hungtintin un esone contiene uno stretch di ripetizioni.
	La malattia \`e dominante. 
	Questo studio unisce analisi genetica con analisi molecolare.

		\subsubsection{Elettroforesi}
		L'elettroforesi viene utilizzata per determinare il numero di ripetizioni di $CAG$ presenti nel gene dopo \emph{PCR}. 
		Si notano due bande sugli eterozigoti, mentre una sugli omozigoti.
		La posizione delle bande indica il numero di ripetizioni presenti. 

	\subsection{Complementazione genica nell'uomo - sordit\`a}
	In famiglie diverse affette da sordit\`a segregano mutazioni non alleliche: ci\`o rende possibile il fenomeno della complementazione genica quando individui di famiglie diverse si incrociano.
	Genitori con sordit\`a profonda hanno uditi di figlio normale.

	\subsection{Sindrome di Beckwith-Wiedemann}
	La sindrome di Beckwith-WIedemann \emph{BWS} \`e causta da mutazioni o delezioni di geni imprinted nel cromosoma $11p15.5$. 
	Geni coinvolti sono \emph{P57}, \emph{H19} e \emph{LIT1}.
	La malattia pu\`o anche essere causata da mutazioni puntiformi nel gene \emph{CDKN1C}.
	\`E un disordine della crescita pediatrico con predisposizione nello sviluppo dei tumori. 
	La presentazione clinica \`e variabile.

		\subsubsection{Caratteristiche cliniche}
		Individui con \emph{BWS} possono crescere pi\`u rapidamente durante la seconda met\`a della gravidanza e nei primi anni di vita, ma l'altezza degli adulti rimane normale.
		La crescita anormale si pu\`o mainfestare come emiipertrofia o macroglossia.
		L'ipoglicemia \`e riportata nel $50\%$ dei bambini. 
		Aumenta la frequenza di malformazioni e complicazioni mediche come difetti della parete addominale, e visceromegalia.

		\subsubsection{Ereditariet\`a}
		Il metodo di eredit\`a della sindrome \`e complesso: pattern possibili includono:
		\begin{itemize}
			\item Ereditariet\`a autosomale dominante con espressivit\`a variabile.
			\item Imprinting genomico risultante da una copia difettiva o assente del gene della madre.
		\end{itemize}


