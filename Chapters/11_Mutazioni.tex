\chapter{Mutazioni}

\section{Panoramica}
Una mutazione \`e un cambiamento permanente di informazione in uno o pi\`u punti del genoma trasmissibile e non riparabile.

	\subsection{Tipologie di mutazione}

		\subsubsection{Mutazioni puntiformi}
		Le mutazioni puntiformi sono mutazioni di una base del genoma.
		Si dividono:
		\begin{multicols}{2}
			\begin{itemize}
				\item Transizione: purina con purina o pirimidina con pirimidina.
				\item Trasversione: da purina a pirimidina o viceversa.
			\end{itemize}
		\end{multicols}

		\subsubsection{Mutazioni frameshift}
		Si intendono per mutazioni frameshift mutazioni causate da un'inserzione o delezione del modulo di lettura dovuto all'inserzione o delezione di una base.

		\subsubsection{Mutazioni indotte}
		Le mutazioni indotte nascono a seguito dell'esposizione ad agenti mutageni conosciuti.

		\subsubsection{Mutazioni adattive}
		Le mutazioni adattive insorgono nei geni che quando mutati conferiscono un vantaggio evolutivo.
		Insorgono a causa di fenomeni di ipermutabilit\`a non equivalenti nel genoma.

		\subsubsection{Mutazioni de novo}
		Si intende per mutazioni de novo mutazioni non riconducibili a una trasmissione di ceppi parentali, limitata ad un solo figlio e correlata ad un errore durante gametogenesi o delle prime fasi dello sviluppo embrionale.

		\subsubsection{Mutazioni a mosaico}
		Le mutazioni a mosaico sono mutazioni che avvengono solo in una parte di un organismo multicellulare.

		\subsubsection{Mutazioni reversibili}
		Le mutazioni reversibili sono mutazioni che portano alla reversione del fenotipo allo stato precedente alla mutazione.

		\subsubsection{Mutazioni condizionali}
		Le mutazioni condizionali sono mutazioni che portano ad un fenotipo alternativo solo a determinate condizioni restrittive, come la sensibilit\`a alla temperatura.

		\subsubsection{Auxotrofie}
		Si intende per auxotrofie mutazioni che portano a difetti nel percorso biochimico che porta alla sintesi di un elemento necessario all'organismo.

		\subsubsection{Mutazioni letali}
		Si intende per mutazioni letali mutazioni in geni essenziali che portano alla morte dell'individuo.

	\subsection{Agenti mutageni}
	L'insorgenza delle mutazioni \`e dovuta a un agente mutageno, generatore della mutazione.

	\subsection{Luogo di mutazioni}
	Le mutazioni possono avvenire in diverse porzioni funzionali del genoma.
	\begin{multicols}{2}
		\begin{itemize}
			\item Promotori: aumentano o riducono l'espressione del gene.
			\item \emph{$3'$-UTR} alterazione della regolazione genica.
			\item Introni: difetto di splicing.
			\item Le ripetizioni nelle \emph{Ori} mitigano gli effetti di mutazioni.
		\end{itemize}
	\end{multicols}

	\subsubsection{Saggio di reversione}
	Il saggio di reversione o controselezione identifica eventi di mutazioni rari.
	Utilizza reverenti per geni reporter.

	\subsubsection{Saggio di soppressione di mutazione non senso}
	Il saggio di soppressione di mutazione nonsenso si serve di un gene target mutato con una sequenza di stop anomala per selezionare i mutanti che revertano allo stato wild type con il gene funzoinante.
	Pu\`o essere quello per la $\beta$-galattosidasi con la mutazione amber.
	Mutazioni che sopprimono il fenotipo mutante in secondo luogo vengono dette soppressor.

\section{Sorgenti di danno al DNA}
Le mutazioni indotte hanno diverse cause in quanto gli acidi nucleici sono suscettibili sia a danno endogeno che indotto.

	\subsection{Tipologie di danno}

		\subsubsection{Danno idrolitico}
		Il danno idrolitico causa diverse mutazioni.

			\paragraph{Deaminazione della citosina}
			La deaminazione della citosina la trasforma in uracile.

			\paragraph{Depurinazione della guanina}
			La depurinazione della guanina lascia uno zucchero con \emph{OH} terminale, un sito abasico riparabile tramite \emph{BER}, con endonucleasi e polimerasi $\beta$.

			\paragraph{Deaminazione della $5$-metilcitosina}
			La deaminazione della $5$-metilcitosina la trasforma in una timina causando un malappaiamento.
			Viene riparato da un sistema di mismatch repair e da glicolasi.

		\subsubsection{Alchilazione}
		L'alchilazione \`e una modifica dei doppi legami all'interno delle basi azotate in legami singoli aggiungendo gruppi metilici.
		Colpiscono principalmente le cellule in replicazione.

			\paragraph{Sorgente endogena}
			La sorgente endogena di alchilazione \`e \emph{SAM} S-adenosil metionina che dona gruppi metili per alcuni processi e raramente su DNA.

			\paragraph{Risoluzione}
			Il processo viene risolto grazie alla \emph{$O^6$-metil-guanina metiltransferasi} che viaggia sul DNA e quando trova una \emph{$O^6$-metil guanina} coniuga su di s\`e il gruppo metile attraverso una cisteina in un processo non reversibile.

	\subsection{Composti bifunzionali}
	I composti bifunzionali sono molecole in grado di danneggiare il DNA attraverso sostituzione nucleofila e attacco di due gruppi alchilici in contemporanea.
	Questo forma un doppio legame covalente tra DNA e le molecole.

		\subsubsection{Procarbazina}
		La procarbazina \`e in grado di compiere due alchilazioni in contemporanea legandosi covalentemente a due basi azotate della doppia elica creando un crosslinl.
		Tale legame pu\`o collegare due filamenti in posizione $n$ e $n+1$.
		Questo si pu\`o riparare unicamente tramite \emph{NER}.

	\subsection{Raggi UV}
	I raggi UV possiedono tre bande $A$, $B$ e $C$.
	La terza \`e sufficientemente energetica per stimolare le basi azotate adiacenti e formare legami covalenti tra di loro provocando una distorsione della doppia elica e un aumento della rigidit\`a.
	
	\subsection{Metabolismo}
	Il metabolismo di un organismo pu\`o detossificare mutageni o attivare pro-mutagene.

	\subsection{Test di Ames}
	Nel test di Ames colonie batteriche di salmonella auxotrofe per l'istidina vengono piastrate su terreno selettivo con un agente mutageno.
	La differenza di tasso di revertanti indica la capacit\`a della sostanza mutagena di creare mutazioni.
	Nel terreno viene aggiunto un estratto di enzimi del fegato in quanto agenti mutageni vengono aggiunti in presenza di citocromi.
	Pertanto l'attivit\`a epatica pu\`o essere causa di un aumento dell'insorgenza delle mutazioni.
	Vengono usati tre ceppi di Salmonella in quanto permettono di identificare il tipo di mutazioni che portano alla reversione.

\section{Identificare la DNA polimerasi che replica il DNA}

	\subsection{Mutanti}
	Polimerasi $\delta$. 
	Inverte $C$ e $G$.
	Sensore con posizione di mutazioni su \emph{ura3} vicino a \emph{Ori}.
	Per il wild type la frequenza di mutazioni \`e uguale nei filamenti.
	Senza MMR aumenta in entrambi.
	Polimerasi mutata e MMR diminuiscono ma elevate.
	Con Polimreasi mutata e no MMR aumentando dastricamente.
	Cambia lo spettro di mutazione non la frequenza, in quanto la polimerasi mutata interviene nel filamento lagging.
	Si formano hotspot sul lagging.
	Sul filamento principale polimerasi $\epsilon$.

\section{Mutazioni dnella riparazione del DNA}

	\subsection{Xeroderma Pigmentosum}
	Sensibilit\`a a danni di raggi UV.

		\subsubsection{Diverse mutazioni}
		Diverse mutazioni che complementano possono portare a diversa sensibilit\`a.
		Modifiche nella riparazione di altri danni, neurologici.
		Effetto sulla trascrizione.

		\subsubsection{Risposta al danno}
		Modifica di endonucleasi.
		Incapacit\`a di stallo sul danno.
		Mutazioni sulla polimerasi translesione.

\section{Triplette ripetute}

	\subsection{Caratteristiche}
	Suscettibili a mutazioni.
	Causano Corea di Huntington.

	\subsection{Fenomeno soglia}
	Range di variazione di ripetizioni no fenotipo.

	\subsection{Meccanismo}
	Errori durante la replicazione in sequenze ripetute di trinucleotidi, strutture secondarie a causa di legami tra filamenti erroneo o slippage.

	\subsection{Lievito sensore}
	Fragilit\`a cromosomica $GAA$ orientamento e richiede MMR.

		\subsubsection{Ceppo}
		Cromosoma $t$ metionina marcaotre, centromero.
		Ultima Ori \emph{ARS507}, capire il filamento leading e quello lagging.
		\emph{lys2} modificato con inserzione di triplette $GAA$.
		Geni lys2 a valle formano struttrue instabili e rottura cromosomica, colorazione anomala.
		Copia lys2-8 su chr 3 che fa ricombinazione con il cromosoma rotto.

		\subsubsection{Orientamento}
		L'orientamento influisce in quanto sul lagging strand pi\`u probabile che si formino strutture secondarie.

		\subsubsection{MMR}
		Il MMR quando coinvolto sulla struttura cruciforme pu\`o creare DSB a causa di endonucleasi, stallo della forca.
		MMR produce una rottura che porta a danni e fonte di ricombinogenicit\`a.

