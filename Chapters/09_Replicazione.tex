\chapter{Replicazione del DNA}

\section{Processo}
La replicazione del DNA \`e un processo bidirezionale e segue un modello semiconservativo.

	\subsection{Origini di replicazione}
	Le origini di replicazione sono siti specifici del DNA ricchi di $AT$ dove inizia la replicazione.

		\subsubsection{Origin recognition complex}
		Gli origin recognition complex \emph{ORC} sono proteine che riconoscono le \emph{Ori} e svi si associano durante la $G_1$ e vengono attivate in fase $S$ fosforilando attraverso \emph{CDK} \emph{Cdc6} e \emph{Cdt1}.
		Questo causa un distacco di \emph{ORC} e la sua attivazione.

	\subsection{Apertura del doppio filamento}
	L'apertura del doppio filamento avviene ad opera di elicasi \emph{MCM2-7}.
	Aprono una bolla da un alto provocando dall'altro un superavvolgimento risolto da topoisomerasi.
	\emph{SSBP} mantengono i due filamenti separati.

	\subsection{Polimerasi}
	Le polimerasi sintetizzano il DNA a partire da corti primer a RNA creando un legame fosfodiestere sul \emph{$3'$-OH} con liberazione di pirofosfato.

		\subsubsection{Primasi}
		La primasi sintetizza il primer a RNA necessario per l'inizio della replicazione.

		\subsubsection{\emph{PCNA}}
		\emph{PCNA} o pinza scorrevole rende la DNA polimerasi processiva sul DNA aumentando il tasso di sintesi.
		\`E composta da $\beta$-foglietti che aumentano l'affinit\`a della polimerasi con il DNA.
		La sua attivit\`a \`e controllata da modifiche post-traduzionali.

		\subsubsection{Polimerasi $\mathbf{\epsilon}$}
		La polimerasi $\epsilon$ si occupa di sintetizzare il lagging strand in maniera semi-discontinua.
		Una RNAasi rimuove i primer.
		\emph{FNE1} riconosce e rimuove sovrapposizioni di DNA in caso di sintesi troppo lunga di un frammento.

\section{Pettinatura del DNA}
La pettinatura del DNA \`e una tecnica che permette di visualizzare con alta risoluzione caratteristiche delle doppie eliche.
Il DNA in soluzione tende a raggomitolarsi e viene posto su un vetrino.
Sfruttando la tensione superficiale vengono pettinate le fibre di DNA rendendolo pi\`u accessibile e pi\`u adatto per colorazione attraverso nucleotidi trifosfati fluorescenti.
Aggiungendo colorante in un momento successivo all'inizio della sintesi si riescono ad individuare le origini di replicazione.

\section{DNA polimerasi}

	\subsection{Sito catalitico}
	
		\subsubsection{Discriminazione del nucleotide corretto}
		Il sito catalitico di una DNA polimerasi permette l'ingresso di tutti i nucleotidi.
		In caso di ingresso di un nucleotide non complementare la distanza tra il \emph{$3'$-OH} e il fosfato $\alpha$ \`e maggiore rendendo la sostituzione nucleofila meno efficace e riducendo la velocit\`a di catalisi.
		Nel caso entri un nucleotide con ribosio come uracile la distanza causata dal gruppo \emph{OH} sar\`a tale da impedire la formazione del legame.

		\subsubsection{Ioni divalenti}
		Ioni divalenti \emph{$Mg^{2+}$} partecipano alla polimerizzazione favorendo la sostituzione nucleofila.

		\subsubsection{Movimento}
		L'attivit\`a di proof-reading avviene grazie al movimento del sito catalitico che ruota valutando l'interazione.

	\subsection{Tipologie di DNA polimerasi}
	Le DNA polimerasi sono molte e diverse, possono essere processive o distributive, pi\`u o meno fedeli in base alla precisione con cui leggono  il DNA.
	L'utilit\`a delle DNA polimerasi distributive sta nei processi di riparazione di DNA o di copiatura di DNA danneggiato in modo da non bloccare la forcella di replicazione.

\section{Telomeri}
Il problema delle estremit\`a cromosomiche in overhang lasciate dal lagging strand viene risolta da telomerasi.
La telomerasi si accoppia ad esso e lo allunga ulteriormente permettendo una strand invasion con una sequenza upstream grazie alle sequenze ripetute.
In C. elegans si trovano \emph{G-strand C-strand} anche a $5'$ con due proteine simili a \emph{POT} nei mammiferi: \emph{CeOB1} e \emph{CeOB2}.
