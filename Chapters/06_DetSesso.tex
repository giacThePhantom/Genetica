\chapter{Determinazione del sesso ed eredit\`a legata ad esso}

\section{Teoria cromosomica dell'eredit\`a}

	\subsection{Scoperta dei comosomi}
	Sutton scopre che \`e possibile distinguere cromosomi individuali nelle cellule che stanno subendo meiosi durante test del grillo Brachystola magna.
	Nel $1902$ descrive pertanto le configurazioni dei cromosomi nella cellula durante i vari stadi della meiosi.
	Distingue $11$ paia di cromosomi in base alla loro dimensione e un singleton accessorio che presume essere un cromosoma sessuale.

		\subsubsection{Teoria cromosomica dell'eredit\`a}
		I cromosomi materni e paterni che si associano in coppie e poi si separano durante la divisione riduzionale della meiosi pongono le basi fisiche per la legge mendeliana dell'ereditariet\`a.

	\subsection{Indipendenza dei cromosomi}
	Sutton nota anche che la posizione di ogni cromosoma sulla piastra metafasica era casuale e non si nota un lato paterno o materno consistente.
	Pertanto ogni cromosoma doveva essere indipendente dagli altri.
	Durante la separazione in gameti, l'insieme di cromosomi in ogni cellula figlia potrebbe contenere una mescolanza di tratti parentali ma non la stessa di altre cellule figlie.

		\subsubsection{Combinatoria}
		L'indipendenza cromosomica impone che il numero di combinazioni cromosomiche per ogni gamete potesse essere calcolato in base al numero di cromosomi nell'organismo.
		Considerando $2^n$ possibili combinazioni di cromosomi, dove $n$ \`e il loro numero, e il loro accoppiamento con altri la variazione negli zigoti si calcola come:
		\[i{(2^n)}^2\]

	\subsection{Relazione tra il comportamento dei cromosomi e l'eredit\`a dei caratteri secondo Mendel}
	Il meccanismo con cui si ereditano i caratteri \`e spiegabile con le modalit\`a di trasmissione dei cromosomi durante la gametogenesi e la fecondazione.
		
		\subsubsection{Principi fondamentali}
		\begin{itemize}
			\item I cromosomi contengono il materiale genetico che viene trasmesso dai genitori.
			\item I cromosomi vengono replicati e trasmessi di generazione in generazione.
			\item I nucleo contengono cromosomi che si presentano in coppie omologhe e un membro di ciascuna coppia viene ereditato dalla madre mentre l'altro dal padre.
			\item Durante la gametogenesi i cromosomi di tipo diverso segregano indipendentemente l'uno dall'altro.
		\end{itemize}

		\subsubsection{Prima legge di Mendel}
		La prima legge di Mendel si spiega con la segregazione dei cromosomi omologhi durante la meiosi.
		Le due coppie di un gene si trovano su cromosomi omologhi, per cui durante la divisione si ha la segregazione dei due alleli in gameti distinti.

		\subsubsection{Seconda legge di Mendel}
		La seconda legge di Mendel si spiega con l'allineamento casuale delle tetradi durante la profase della meiosi $I$.
		I due geni situati su due cromosomi differenti durante la metafase della meiosi $I$ si dividono casualmente e le disposizioni possibili dei cromosomi omologhi nelle tetradi possono condurre a combinazioni diverse negli alleli nei gameti risultanti.

\section{Esclusione del \emph{mtDNA} paterno}
Nonostante lo scopo della riproduzione sessuale sia di mischiare i genomi, si nota come il \emph{mtDNA} paterno viene escluso.

	\subsection{\emph{OXPHOS}}
	Il \emph{mtDNA} codifica per le subunit\`a centrali del complessi a multipli peptidi \emph{OXPHOS I, III, IV, V}.
	Essendo la sequenza dei geni di \emph{mtDNA} altamente variabile il mescolamento di due diverse variazioni di \emph{mtDNA} degli stessi elementi \emph{OXPHOS} potrebbe essere deleteria.
	Una congettura predice che se due \emph{mtDNA} normali ma diversi verrebbero mescolati nello stesso animale avverrebbe dell'incopatibilit\`a.
	Questa incompatibilit\`a renderebbe instabile lo stato eteroplasmatico e avere effetti negativi sul fenotipo dell'animale.

	\subsection{Endonucleasi mitocondriale \emph{G}}
	I mitocondri sono ereditati maternalmente nella maggior parte degli animali, ma i meccanismi di eliminazione elettiva mitocondriale paterna \emph{PME} sono sconosciuti. 
	Analizzando fertilizzazione in Caenorhabditis elegans si osserva che mitocondri paterni perdono rapidamente l'integrit\`a della membrana interna. 
	\emph{CSP-6} un'endonucleasi mitocondriale \emph{G} si riloca dallo spazio intermembrana alla matrice dopo la fertilizzazione per degradare il DNA mitocondriale.
	Agisce con autofagia materna e macchinari di proteosomi per promuovere \emph{PME}.
	Rimozione ritardata di mitocondri paterni causa un aumento nella letalit\`a embrionale.

	\subsection{Eteroplasmia del \emph{mtDNA} nel topo \`e geneticamente instabile e causa cognizione e comportamento alterati}
	L'eredit\`a materna del \emph{mtDNA} \`e la norma e per investigare le conseguenze si generano topi contenenti una mescolanza di 
	\emph{mtDNA} \emph{NZB} e \emph{123S6} nella presenza di un congenito background nucleare \emph{C57BL/6J}.
	L'analisi della segregazione dei due \emph{mtDNA} sivela che la porzione di \emph{NBZ} era ridotta preferenzialmente.
	La segregazione produce topi \emph{NZB-129} eteroplasmici e le controparti omoplasmiche. 
	Comparando i fenotipi si dimostra come i topi eteroplasmici presentano attivit\`a ridotta di intake di cibo, di tasso respiratorio, risposta allo stressa accentuata e impedimenti cognitivi.
	Un'insieme dei due \emph{mtDNA} diversi \`e geneticamente instabile e pu\`o produrre effetti fisiologici avversi. 

\section{Sistemi di determinazione del sesso}
	
	\subsection{I sistemi cromosomici di determinazione del sesso}
	
		\subsubsection{Scoperte}

			\paragraph{Henking}
			Nel $1891$ Henking scopre una struttura peculiare nel nucleo delle cellule di insetti maschi e la chiama corpo $X$.
	
			\paragraph{Stevens e Wilson}
			Nel $1905$ Stevens e Wilson dimostrano che nelle cavallette e altri insetti le cellule della femmina hanno due cromosomi $X$, mentre in quelle del maschio solo $1$.
			In alcuni insetti contano lo stesso numero di cromosomi nei maschi e nelle femmine, ma notano che una coppia di questi era diversa: nelle femmine si trovano due $X$, mentre nei maschi compare un solo $X$ e uno pi\`u piccolo detto $Y$. 
			Mostrano come i due cromosomi $X$ e $Y$ si separano in cellule diverse durante la formazione dello sperma: met\`a riceve un $X$ e l'altra un $Y$.
			Tutte le cellule uovo invece ricevono un $X$.
			Se uno spermatozoo con un $X$ incontra un uovo nasce una femmina $XX$, mentre se contiene un $Y$ nasce un maschio $XY$.
			Questa distribuzione spiega come mai il rapporto fra sessi \`e $1:1$.
			Scoprono pertanto che il sesso \`e associato all'eredit\`a di una coppia di cromosomi sessuali, diversi in maschi e femmine.
			I cromosomi uguali non sessuali vengono invece detti autosomi.
	
		\subsubsection{Determinazione del sesso $\mathbf{XX}$-$\mathbf{X0}$}
		Il sistema di cavallette studiato da McClung \`e uno dei meccanismi pi\`u semplici di determinazione del sesso attraverso i cromosomi.
		Nel sistema $XX$-$X0$ le femmine hanno due cromosomi $XX$ mentre i maschi ne possiedono uno solo $X0$.
		Il numero zero indica infatti l'assenza di un cromosoma sessuale.
		Il sesso di un singolo organismo \`e determinato dal tipo di gamete maschile.
		
			\paragraph{Meiosi}
			Nella meiosi delle femmine i due cromosomi $X$ si appaiano e si separano in modo che un solo $X$ entra nell'uovo.
			Nei maschi l'unico $X$ segrega in met\`a delle cellule, mentre la rimanente non riceve alcun cromosoma sessuale.
		
			\subparagraph{Eterogameti e omogameti}
			I maschi che producono due tipi diversi di gameti sono detti sesso eterogametico, mentre le femmine che ne producono solo uno sono dette sesso omogametico.
	
		\subsubsection{Determinazione del sesso $\mathbf{XX}$-$\mathbf{XY}$}
		In molte specie le cellule di maschi e femmine hanno lo stesso numero di cromosomi, ma quelle delle femmine hanno due cromosomi $X$ e quelle dei maschi un solo cromosoma $X$ e uno sessuale pi\`u piccolo $Y$. 
		Nell'uomo il cromosoma $Y$ \`e acrocentrico 
		Il maschio \`e etrogametico, mentre la femmina omogametica.
		Piante, insetti, rettili e animali hanno questo sistema, mentre molti altri varianti di questo.
		Durante la meiosi $X$ e $Y$ si accoppiano e segregano in cellule diverse.
		Questo \`e possibile dal fatto che i cromosomi sono omologhi in aree dette regioni pseudoautosomiche nelle quali partano gli stessi geni. 
		Nell'essere umano sono presenti ad entrambe le estremit\`a e sono dette \emph{PAR1} e \emph{PAR2}.
		Pu\`o avvenire crossing over in queste regioni durante la meiosi. 

		\subsubsection{Determinazione del sesso $\mathbf{ZZ}$-$\mathbf{ZW}$}
		In questo sistema la femmina \`e eterogametica e il maschio omogametico.
		Le femmine sono $ZW$ e i maschi $ZZ$, dopo la meiosi met\`a delle uova hanno il cromosoma $Z$ e l'altra met\`a il cromosoma $W$.
		Questo sistema presiede alla determinazione del sesso in uccelli, serpenti, farfalle, anfibi e pasci.

	\subsection{Determinazione genica del sesso}
	In alcuni organismi il sesso \`e determinato geneticamente ma non ci sono differenze evidenti nei cromosomi di maschi e femmine.
	Non esistono cromosomi sessuali e sono i genotipi su uno o pi\`u loci a determinare il sesso.
	Avviene in piante, funghi, protozoi e pesci.
	Si noti come anche nel sistema cromosomico il sesso \`e determinato da geni individuali.

	\subsection{Determinazione del sesso legata all'ambiente}
	In un certo numero di organismi il sesso \`e determinato da fattori ambientali. 

		\subsubsection{Esempi}

			\paragraph{Crepidula fornicata}
			Nella crepidula fornicata o patella comune \`e un mollusco che vive in colonie costituite da pi\`u individui sovrapposti. 
			Ogni patella nasce come larva: la prima larva che trova un substrato disponibile si sviluppa come femmina.
			Produce sostanze chimiche che producono altre patelle che si insediano sopra di lei.
			Queste si sviluppano come maschi che si accoppiano con la patella sottostante.

				\subparagraph{Ermafroditismo sequenziale}
				Nell'ermafroditismo sequenziale ogni mollusco pi\`o essere sia maschio che femmina anche se non nello stesso momento.

			\paragraph{Tartarughe}
			Nelle tartarughe temperature elevate durante il periodo di cova producono pi\`u femmine.
	
			\paragraph{Alligatori}
			Negli alligatori temperature elevate durante il periodo di cova producono pi\`u maschi.

			\paragraph{Lucertole drago barbute}
			Nelle lucertole drago barbute i fattori ambientali possono prevalere sulla determinazione cromosomica.
			Tipicamente i maschi sono $ZZ$ e le femmine $ZW$, ma a temperature elevate individui $ZZ$ si sviluppano fenotipicamente come femmine.

	\subsection{Determinazione del sesso nella Drosophila melanogaster}
	La drosophila melanogaster possiede $8$ cromosomi: tre coppie di autosomi e una di cromosomi sessuali.
	Le femmine sono $XX$ e i maschi $XY$.
	Il sesso della Drosophila non \`e determinato dal numero di cromosomi $X$ e $Y$ ma dal bilanciamento fra geni di determinazione femminile localizzati sul cromosoma $X$ e geni di determinazione maschile presenti sugli autosomi.
	
		\subsubsection{Rapporto $\mathbf{X:A}$}
		Il sesso dei moscerini \`e determinato dal rapporto $X:A$, il numero di cromosomi diviso per il numero di assetti aploidi dei cromosomi autosomici.
		I moscerini possiedono solitamente due assetti apolidi di autosomi e due cromosomi $X$ se femmine, mentre un $X$ e un $Y$ se maschi.
		Un rapporto $X:A$ pari a $1$ d\`a origine a una femmina, mentre apri a $0.5$ a un maschio.
	
		\subsubsection{Meccanismo molecolare}
		Sono presenti sul cromosoma $X$ diversi geni che influiscono sul fenotipo sessuale, ma ne esistono alcuni autosomici che lo fanno.
		I principali geni responsabili della determinazione del sesso sono situati sul cromosoma $X$.
		L'influenza dei caratteri autosomici \`e indiretta e influenza la tempistica degli eventi di sviluppo.

	\subsection{La determinazione del sesso nell'uomo}
	L'uomo ha una determinazione del sesso $XX$-$XY$, ma la presenza del gene \emph{SRY} sul cromosoma $Y$ determina i caratteri sessuali maschili.
	I fenotipi che risultano da un numero anomalo di cromosoma sessuali dimostra l'importanza del $Y$.
	
		\subsubsection{Patologie legate al sesso}
	
			\paragraph{Sindrome di Turner}
			La sindrome di Turner colpisce donne con caratteristiche sessuali secondarie non pienamente sviluppate.
			Le donne colpite sono di bassa statura, attaccatura di capelli bassa, torace ampio e pieghe cutanee sul collo.
			Le cellule con possono avere un solo cromosoma $X$.
			Non essendo noti casi in cui un individuo abbia perso entrambi i cromosomi $X$ si dimostra come questo sia essenziale.

			\paragraph{Sindrome di Klinefelter}
			Le persone che soffrono della sindrome di Klinefelter hanno cellule con uno o pi\`u cromosomi $Y$ e $X$ multipli.
			Gli uomini colpiti dalla sindrome hanno testicoli ridotti e scarsa peluria sul volto e pube, sono pi\`u alti del normale e sterili.
	
			\paragraph{Femmine poli-$\mathbf{X}$}
			Le cellule femminili in alcuni casi presentano $3$ $X$ o sindrome della tripla $X$: non presentano caratteristiche particolari se non una tendenza ad essere alte e magre.
			Alcune sono sterili ma la maggior parte ha un ciclo regolare ed \`e fertile.
			L'incidenza di ritardo mentale \`e leggermente superiore alla norma.
			La gravit\`a del ritardo mentale aumenta linearmente con il numero di $X$.
	
			\paragraph{Sindrome da insensibilit\`a agli androgeni}
			Le donne affette da sindrome da insensibilit\`a agli androgeni possiedono una vagina a fondo ceco e al posto di utero, ovidotti e ovaie sono presenti dei testicoli che producono livelli di testosterone simili a quelli dei maschi.
			Le cellule contengono un cromosoma $X$ e un $Y$.
			In un embrione umano con un cromsoma $Y$ \emph{SRY} trasforma che le gonadi si trasformino in testicoli e producano il testosterone.
			Questo poi stimola i tessuti embrionali a sviluppare caratteri maschili.
			Per farlo deve per\`o legarsi a un recettore assente nelle femmine con questa sindrome.
			Le cellule sono pertanto insensibili al testosterone e sviluppano caratteristiche femminili.
			Questo gene si trova sul cromosoma $X$.
	

		\subsubsection{Il numero dei cromosomi sessuali}
		\begin{multicols}{2}
			\begin{itemize}
				\item Il cromosoma $X$ contiene informazioni geniche essenziali per entrambi i sessi, per uno sviluppo corretto \`e necessaria la presenza di almeno una copia del cromosoma $X$.
				\item Il gene che determina il sesso maschile \`e localizzato sul cromosoma $Y$.
					Un'unica copia anche in presenza di diversi cromosomi $X$ determina un fenotipo maschile.
				\item La mancanza del cromosoma $Y$ di solito produce un fenotipo femminile.
				\item I geni che influiscono sulla fertilit\`a sono sul cromosoma $X$ e $Y$.
					Per essere fertile, una femmina di solito ha bisogno di almeno due copie del cromosoma $X$.
				\item Copie eccedenti del cromosoma $X$ possono alterare lo sviluppo normale dei maschi e delle femmine producendo problemi fisici e mentali.
			\end{itemize}
		\end{multicols}
	
		\subsubsection{il gene che determinali fenotipo maschile nell'uomo}
		Il cromosoma $Y$ nell'uomo \`e fondamentale per determinare il fenotipo maschile, ma esistono rari maschi $XX$.
		Si nota come in questi casi una parte di $Y$ si era attaccata a un altro cromosoma. 
		Durante gli stadi precoci dello sviluppo tutti gli esseri possiedono gonadi indifferenziate.
		A $6$ settimane dalla fecondazione si attiva un gene sul cromosoma $Y$: questo fa tasformare le gonadi in testicoli che secernono due ormoni, testosterone e anti-muelleriano, che determinano la regressione dei dotti riproduttivi femminili.
		Il gene della regione $Y$ di determinazione del sesso \emph{SVY} codifica per una proteina fattore di trascrizione che stimola la trascrizione di altri geni legandosi al DNA.
		Altri geni presenti sul $X$, altri sul $Y$ e sugli autosomici hanno una funzione nella fertilit\`a e differenze fra i sessi.

\section{Le caratteristiche legate al sesso sono determinate da geni presenti su cromosomi sessuali}
I geni sul cromosoma $X$ determinano caratteristiche legate al $X$ e quelli sul $Y$ caratteristiche legate al $Y$.
Essendo $X$ molto piccolo e contiene una quantit\`a di informazioni limitata molte caratteristiche legate al sesso sono legate al $X$.

	\subsection{Occhi bianchi legati al cromosoma $\mathbf{X}$ nella Drosophila}
	Morgan studiando Drosophila scopre fra i moscerini della colonia del suo laboratorio un maschio con occhi bianchi diverso dai normali con occhi rossi.
	Per indagare sull'ereditariet\`a del carattere conduce una serie di incroci: comincia con una linea pura di femmine con gli occhi rossi con il maschio con gli occhi bianchi, producendo una $F_1$
	a occhi rossi.
	Reincrociando i moscerini in $F_1$ nota come tutte le femmine di $F_2$ avevano occhi rossi, mentre met\`a dei maschi aveva occhi bianchi.
	Morgan ipotizza che il locus per il colore degli occhi si trovasse sul cromosoma $X$.
	I maschi sono pertanto definiti emizigoti per i loci legati al cromosoma $X$.

	\subsection{Meccanismo di non-disgiunzione e teoria cromosomica dell'ereditariet\`a}
	Incrociando il maschio a occhi bianchi e femmine omozigoti a occhi rossi tra gli individui ne erano presenti tre con occhi bianchi.
	Morgan attribuisce la loro presenza a mutazioni casuali, ma la loro frequenza era troppo elevata.
	Bridges, un suo allievo allora nota come queste eccezioni avvengono solo in certi ceppi di moscerini a occhi bianchi.
	Incrociando una di queste femmine anomale a occhi bianchi con un maschio a occhi rossi il $5\%$ della progenie maschile aveva occhi rossi e il $5\%$ di quella femminile aveva occhi bianchi.
	Essendo che in questo incrocio ogni moscerino maschio edita un $X$ dalla madre con genotipo $X^WY$ e occhi bianchi, mentre ogni femmina eredita un allele dominante per gli occhi rossi dal $X$ del padre, tutta la progenie femminile dovrebbe essere $X^+X^W$ e avere occhi rossi.

		\subsubsection{Spiegazione di Bridges}
		Per spiegare la comparsa di occhi bianchi Bridges ipotizza che le femmine a occhi bianchi possiedano due $X$ e un cromosoma $Y$: in Drosophila i moscerini $XXY$ sono femmine.
		Il $90\%$ delle volte i due $X$ si separano uno dall'altro con un $X$ e un $Y$ che entrano in un gamete e $X$ nell'altro.
		Questi gameti, quando fecondati da uno spermatozoo di maschio con occhi rossi producono maschi con occhi bianchi e femmine con occhi rossi, ma il $10\%$ delle volte i due cromosomi $X$ delle femmine non riescono a separarsi nel fenomeno di non-disgiunzione.
		In questo caso met\`a delle uova riceve due $X$ e l'altra met\`a solo $Y$.
		Si generano pertanto quattro combinazioni:
		\begin{multicols}{2}
			\begin{itemize}
				\item $X^+X^WX^W$ che di solito muore.
				\item $X^WX^WY$ che si sviluppa come femmina a occhi bianchi.
				\item $X^+Y$ che si sviluppa come maschio a occhi rossi.
				\item $YY$ che di solito muore.
			\end{itemize}
		\end{multicols}
		Questo esperimento fornisce una prova di come i geni legati al sesso sono situati sul cromosoma $X$.

	\subsection{Daltonismo legato al cromosoma $\mathbf{X}$ nell'uomo}
	L'occhio umano percepisce con chiarezza il colore grazie ai coni che ricoprono la retina.
	Ogni cono contiene uno dei tre pigmenti capace di assorbire la luce di una particolare lunghezza d'onda.
	Ognuno dei coni \`e codificato da un locus specifico.
	Quello per il blu si trova sul cromosoma $7$, mentre quelli per verde e rosso su $X$.
	La cecit\`a ai colori \`e causata da difetti dei pigmenti rosso e verde.
	Viene detta daltonismo.
	Il daltonismo \`e ereditato come carattere recessivo legato a $X$.

	\subsection{Caratteri legati al cromosoma $\mathbf{Z}$}
	Negli organismi con determinazione del sesso $ZZ$-$ZW$ i maschi sono omogametici e portano due alleli legati al sesso e possono essere omo od eterozigoti.
	Le femmine sono eterogametiche e hanno un solo allele legato a $Z$.
	L'eredit\`a dei caratteri legati a $Z$ \`e la stessa di quelli legati a $X$, tranne che il modello di eredit\`a \`e invertito.
	La femmina eredita il $W$ dalla madre e lo $Z$ dal padre, mentre il maschio eredita $Z$ dalla madre e dal padre.

		\subsubsection{Esempi}

			\paragraph{Cameo nel pavone blu indiano}
			Il fenotipo cameo nel pavone blu indiano \emph{Pavo cristatus} \`e il risultato di un allele legato a $Z$.
			Tipicamente il colore del piumaggio selvatico \`e blu metallico.
			Il piumaggio cameo caratterizzato da piume marrone \`e il risultato di un allele $Z^{ca}$ recessivo rispetto a blu $Z^{ca+}$.
			Se una femmina blu si incrocia con un maschio cameo tutte le femmine di $F_1$ saranno cameo e tutti i maschi blu.

	\subsection{Caratteri legati al cromosoma $\mathbf{Y}$}
	I caratteri legati al $Y$ o caratteri olandrici mostrano un modello di eredit\`a diverso: vengono ereditati solo nei maschi e sempre dal padre.
	Un maschio con un carattere legato a $Y$ lo trasmette a tutta la progenie maschile.

		\subsubsection{Evoluzione del cromosoma $\mathbf{Y}$}
		I cromosomi sessuali si evolvono da una coppia di autosomi.
		Il primo passo si verifica quando un membro di una copia acquisisce un gene che determina il fenotipo maschile.
		Ogni organismo con una copia del cromosoma diventa maschio.
		Su questi proto-$Y$ si verificano ulteriori mutazioni riguardanti caratteri vantaggiosi unicamente per il sesso maschile.
		Per evitare che compaiano anche nelle femmine si sopprime il crossing-over tra $X$ e $Y$.
		L'assenza di crossing-over porta a un accumulo di mutazioni e perdita di materiale genetico da parte di $Y$.

		\subsubsection{Caratteristiche del cromosoma $\mathbf{Y}$ nell'uomo}
		Due terzi del cromosoma $Y$ sono costituiti da brevi ripetizioni del DNA e prive di geni attivi.
		L'altro terzo \`e costituito da pochi geni: $350$, molti sembrano avere un ruolo nello sviluppo sessuale maschile e nella fecondit\`a.
		Contiene molti elementi che influiscono sull'espressione di numerosi geni autosomici e su $X$.
		Sono presenti otto sequenze palindromiche tra cui avviene ricombinazione impedendo la sua completa degradazione.
		In alcuni casi possono portare a delezioni del gene \emph{SRY} producendo femmine $XY$.

\section{La compensazione di dose uniforma i livelli di proteina prodotti dai geni $\mathbf{X}$-linked e dai geni autosomici}
Nelle specie $XX$-$XY$ la differenza nel numero di cromosomi $X$ rappresenta un problema durante lo sviluppo: i geni sui cromosomi $X$ e sugli autosomi nelle femmine non sono in equilibrio.
Essendo la quantit\`a di proteina prodotta funzione del numero di copie del gene, nei maschi ci sarebbero meno proteine codificati da geni legato a $X$.
Alcuni animali superano il problema attraverso meccanismi che rendono uguale il quantitativo di proteina prodotto dall'unico cromosoma $X$ e dai due autosomi o compensazione di dose.
Nei moscerini della frutta la compensazione \`e raggiunga raddoppiando l'attivit\`a dei geni su $X$ dei maschi ma non delle femmine.
Nei mammiferi plascentati l'espressione dei geni sensibili alla dose su $X$ \`e aumentata associandosi all'inattivazione di uno dei cromosomi $X$ nelle femmine in modo che l'espressione dei geni sia bilanciata.

	\subsection{Ipotesi di Lyon}
	nel $1949$ vengono osservati da Barr corpi compatti nei nuclei delle cellule di gatti femmina.
	Queste strutture vengono chiamate corpi di Barr e nel $1961$ Lyon ipotizza che fossero cromosomi $X$ inattivi.
	Viene suggerito che all'interno di ogni cellula femminile, in maniera casuale uno dei due cromosomi $X$ risultasse inattivo.
	Una conseguenza \`e che le femmine dei mammiferi placentati sono emizigoti per i geni legati al $X$: negli organismi eterozigoti per un carattere su $X$ met\`a delle cellule esprime un allele, mentre l'altra met\`a quello rimanente.
	Le femmine sono pertanto mosaici per l'espressione di questi geni.
	Essendo che l'inattivazione casuale ha luogo nelle prime fasi dello sviluppo e la linea somatica eredita lo stato le cellule vicine tendono a presentare lo stesso cromosoma $X$ inattivo e un modello a chiazze, come si nota nei gatti calico.
	Le patologie legate al numero di cromosomi $X$ suggeriscono che solo il $75\%$ dei geni legati a $X$ sono disattivati permanentemente, alcuni sfuggono completamente, mentre altri vengono disattivati solo in alcuni individui.

	\subsection{Meccanismo dell'inattivazione casuale del cromosoma $\mathbf{X}$}
	L'inattivazione casuale di $X$ si compie in due stadi:
	\begin{itemize}
		\item Primo stadio: la cellula stima quanti $X$ sono presenti.
		\item Secondo stadio: viene selezionato un cromosoma $X$ che deve rimanere attivo mentre gli altri vengono silenziati.
	\end{itemize}
	Numerosi geni partecipano a questo processo tra cui \emph{Xist} (trascritto specifico per l'inattivazione di $X$).
	Sui cromosomi $X$ destinati ad essere disattivati \emph{Xist} \`e attivo e produce una lunga molecola di RNA di \num{17000}\si{nt} che si avvolge intorno a $X$.
	Successivamente recluta proteine e porta al silenziamento dei geni modificando la struttura cromatinica.
	Sul cromosoma $X$ destinato a rimanere attivo altri geni reprimono \emph{Xist} in modo che non lo copra facendo rimanere attivi i suoi geni.
